\section{2015}
\begin{enumerate}[1~]
\renewcommand{\labelenumi}{\textbf{\theenumi. }}
\renewcommand{\Im}{\text{Im }}
\item[一、]
填空题 (每小题 4 分,共 20 分)
\begin{enumerate}[1.~]
\item
若 $\mathbb{P}$ 为包含 $\mathbb{Q}$ 和 $\sqrt{3}$ 的最小数域, 则 $\mathbb{P}$ 视为 $\mathbb{Q}$ 上的线性空间其维数是(\quad)。
\begin{solution}
$2$。过程可参考2005年填空题第一题。
\end{solution}

\item
若 $f(x)$ 为数域 $\mathbb{P}$ 上的不可约多项式, 则 $f(x)$ 与 $f'(x)$ 的关系是 (\quad)。
\begin{solution}
互素。过程可参考 2006 年填空题第二题。
\end{solution}

\item
若 $A$ 为奇数阶反对称阵,则 $|A| =$(\quad)。
\begin{solution}
0。过程可参考 2006 年填空题第三题。
\end{solution}

\item
设 $A$ 为方阵,且$A^3=0$,则$(E-A)^{-1}=$(\quad)。
\begin{solution}
$E+A+A^2$。过程可参考 2006 年填空题第四题。
\end{solution}

\item
向量组$\boldsymbol{\boldsymbol{\alpha}} _ { 1 } , \boldsymbol{\alpha} _ { 2 } , \boldsymbol{\alpha} _ { 3 } , \boldsymbol{\alpha} _ { 4 } , \boldsymbol{\alpha} _ { 5 } \in \mathbb { R } ^ { 5 }$线性无关,则向量组$\boldsymbol{\alpha} _ { 1 } + \boldsymbol{\alpha} _ { 2 } , \boldsymbol{\alpha} _ { 2 } + \boldsymbol{\alpha} _ { 3 } , \boldsymbol{\alpha} _ { 3 } + \boldsymbol{\alpha} _ { 4 } , \boldsymbol{\alpha} _ { 4 } + \boldsymbol{\alpha} _ { 5 } , \boldsymbol{\alpha} _ { 5 } + \boldsymbol{\alpha} _ { 1 }$的线性相关性是(\quad)。
\begin{solution}
线性相关。过程可参考 $2006$ 年填空题第五题。
\end{solution}

\item
设$A$ ,$B$ 为 $n$ 阶方阵,且 $AB=0$,则$r(A)+r(B)\le$(\quad)。
\begin{solution}
由 Sylvester 不等式知\[
r(A)+r(B)\le n+r(AB)=n.
\]
\end{solution}

\item
设$\mathscr { A }$为 $n$ 维线性空间$V$ 的线性变换,$\Ker\mathscr { A }=0$,则$\mathscr { A }$为(\quad)线性变换
\begin{solution}
可逆。过程可参考2006年填空题第七题。
\end{solution}

\item
设$A$,$B$ 为$n$ 阶方阵,且$A$ 可逆,则 $AB$ 与 $BA$ 的关系是(\quad)。
\begin{solution}
相似,过程可参考2006年填空题第八题。
\end{solution}

\item
若 $A$,$B$ 为同阶正交阵,且$|AB|=-1$,则$|A+B|=$(\quad)。
\begin{solution}
0。过程可参考2006年填空题第九题。
\end{solution}

\item
设$A$ 为$m\times n$ 实矩阵,$r(A)=n$,则$n$元二次型$X^T(A^T A)X$正定性为(\quad)。
\begin{solution}
正定。
\end{solution}
\end{enumerate}

\item[二、](15分)
设 $V$ 为 $\mathbb{R}$ 上的三维线性空间,$\mathscr { A }$为$V$ 的一个线性变换,且$\mathscr { A }$在 $V$ 的基$\boldsymbol{\alpha} _ { 1 } , \boldsymbol{\alpha} _ { 2 } , \boldsymbol{\alpha} _ { 3 }$下的矩阵为\[
A = \left( \begin{array} { c c c } { 1 } & { 4 } & { 2 } \\ { 0 } & { - 3 } & { 4 } \\ { 0 } & { 4 } & { 3 } \end{array} \right).
\]
(1) 求 $V$ 的另一个基 $\boldsymbol{\beta} _ { 1 } , \boldsymbol{\beta} _ { 2 } , \boldsymbol{\beta} _ { 3 }$ 使 $\mathscr{A}$ 在此基下的矩阵 $B$ 为对角阵。\\
(2) 求 $A^k$。
\begin{solution}
重复。
\end{solution}

\item[三、](15分)
对齐次线性方程组\[
\left\{ \begin{array} { l } { x _ { 1 } + x _ { 2 } + x _ { 3 } + x _ { 4 } + x _ { 5 } = 0 ,} \\ { 3 x _ { 1 } + 2 x _ { 2 } + x _ { 3 } + x _ { 4 } - 3 x _ { 5 } = 0 ,} \\ { x _ { 2 } + 2 x _ { 3 } + 2 x _ { 4 } + 6 x _ { 5 } = 0 ,} \\ { 5 x _ { 1 } + 4 x _ { 2 } + 3 x _ { 3 } + 3 x _ { 4 } - x _ { 5 } = 0. } \end{array} \right.
\]
(1) 求其一个基础解系。\\
(2) 求其向量形式的通解。
\begin{proof}
(1) 齐次线性方程组的系数矩阵为\[
A=\left( \begin{matrix}
	1&		1&		1&		1&		1\\
	3&		2&		1&		1&		-3\\
	0&		1&		2&		2&		6\\
	5&		4&		3&		3&		-1\\
\end{matrix} \right) 
\]
对矩阵$A$作初等行变换得\[
A\rightarrow \left( \begin{matrix}
	1&		0&		-1&		-1&		-5\\
	0&		1&		2&		2&		6\\
	0&		0&		0&		0&		0\\
	0&		0&		0&		0&		0\\
\end{matrix} \right) 
\]
由此可得原线性方程组的一个基础解系\[
\eta_1= (1, -2, 1, 0, 0)', \eta_2=(1, -2, 0, 1, 0)', \eta_3=(5, -6, 0, 0, 1)'.
\]
(2) 通解为\[
\left( \begin{array}{c}
	x_1\\
	x_2\\
	x_3\\
	x_4\\
	x_5\\
\end{array} \right) =k\left( \begin{array}{c}
	1\\
	-2\\
	1\\
	0\\
	0\\
\end{array} \right) +l\left( \begin{array}{c}
	1\\
	-2\\
	0\\
	1\\
	0\\
\end{array} \right) +m\left( \begin{array}{c}
	5\\
	-6\\
	0\\
	0\\
	1\\
\end{array} \right) , k, l, m\in \mathbb{Z}.
\]
\end{proof}

\item[四、](15分)
设 $\mathscr { A } , \mathscr { B }$ 为 $n$ 维线性空间 $V$ 上的线性变换,\[
( \mathscr { A } + \mathscr { B } ) ^ { 2 } = \mathscr { A } + \mathscr { B } , \mathscr { A } ^ { 2 } = \mathscr { A } , \mathscr { B } ^ { 2 } = \mathscr { B }.
\]
求证:$\mathscr { A }  \mathscr { B }=0$。
\begin{proof}
重复。
\end{proof}

\item[五、](15分)
 设 $\mathscr{A}$ 为数域 $\mathbb{P}$ 上 $n$ 维线性空间 $V$ 上的线性变换,$f_1(x), f_2(x)$ 为 $\mathbb{P} [x]$ 中两个互素的多项式,$f(x) =
f_1(x)f_2(x)$,求证:\[
\operatorname { Ker } f ( \mathscr { A } ) = \operatorname { Ker } f _ { 1 } ( \mathscr { A } ) \oplus \operatorname { Ker } f _ { 2 } ( \mathscr { A } ).
\]
\begin{proof}
重复。
\end{proof}

\item[六、](15分)
设 $V$ 为数域 $\mathbb{F}$ 上的 $n$ 维线性空间,$\mathscr{A}$ 为 $V$ 的线性变换\[
\mathscr { A } ^ { 2 } = \mathscr { A }.
\]
求证:\\
(1) $V = \mathscr { A } ( V ) \oplus \operatorname { Ker } \mathscr { A }$。\\
(2) 存在$V$ 的一个基 $\varepsilon _ { 1 } , \varepsilon _ { 2 } , \cdots , \varepsilon _ { n }$,在此基下 $\mathscr{A}$ 的矩阵为 $A = \operatorname { diag } \{ 1 , \dotsc , 1,0 , \dotsc , 0 \}$ (对角线为
$1 , \dotsc , 1 , 0 , \dotsc , 0$ 的对角阵)。
\begin{proof}
(1) 任取$\boldsymbol{\alpha} \in V$,则$\mathscr { A } { \boldsymbol{\alpha} } \in \mathscr{A}V$。由于\[
\mathscr { A } ( \boldsymbol{\alpha} - \mathscr { A } \boldsymbol{\alpha} ) = \mathscr { A } \boldsymbol{\alpha} - \mathscr { A } ^ { 2 } \boldsymbol{\alpha} = \mathscr { A } \boldsymbol{\alpha} - \mathscr { A } \boldsymbol{\alpha} = 0,
\]
因此$\boldsymbol{\alpha} - \mathscr { A } { \boldsymbol{\alpha} } \in \Ker \mathscr { A } $。由于$\boldsymbol{\alpha} = \mathscr { A }  { \boldsymbol{\alpha} } + \left( \boldsymbol{\alpha} - \mathscr { A } { \boldsymbol{\alpha} } \right)$,因此\[
V =  \mathscr { A }V + \Ker \mathscr { A }.
\]
任取$\boldsymbol{\beta} \in \mathscr { A }V \cap \Ker \mathscr { A }$,由于$\boldsymbol{\beta} \in\mathscr { A }V$,因此存在$\boldsymbol{\gamma} \in V$,使得$\boldsymbol{\beta} = \mathscr { A } \boldsymbol{\gamma}$。由于$\boldsymbol{\beta} \in \Ker \mathscr { A }$,因此$\mathscr { A } \boldsymbol{\beta} = 0$。从而\[
0 = \mathscr { A } \boldsymbol{\beta} = \mathscr { A } ( \mathscr { A } \boldsymbol{\gamma} ) = \mathscr { A } ^ { 2 } \boldsymbol{\gamma} = \mathscr { A } \boldsymbol{\gamma} = \boldsymbol{\beta}.
\]
于是\[
\mathscr { A }V \cap \Ker \sigma = 0.
\]
所以$V=\mathscr { A }V \oplus \Ker \mathscr { A }$。\\
(2) 在$\mathscr { A }V$和$\Ker \mathscr { A }$中分别取一个基,它们合起来是$V$的一个基 $\varepsilon _ { 1 } , \varepsilon _ { 2 } , \dotsc , \varepsilon _ { n }$,$\mathscr { A }$在此基下的矩阵为\[
\left( \begin{array} { l l } { I _ { r } } & { 0 } \\ { 0 } & { 0 } \end{array} \right),
\]
其中$r={\rm rank}(A)$。
\end{proof}

\item[七、](15分)
设 $A$ 为 $n$ 阶方阵,$\lambda_0$ 为 $A$ 的特征值。此时我们称 $n -  { r } \left( \lambda _ { 0 } E - A \right)$ 为 $\lambda_0$ 的几何重数,$\lambda_0$ 作为 $A$ 的特征
多项式 $| \lambda E - A |$ 之根的重数称为 $\lambda_0$ 的代数重数。求证:$\lambda_0$ 的几何重数不超过其代数重数。
\begin{proof}
重复。
\end{proof}

\item[八、](15 分)
设 $A, B$ 为 $n$ 阶实对称阵,且 $A$ 正定,求证:存在一个可逆矩阵 $P$ 使得 $P^T AP$ 和 $P^T BP$ 同时为对角阵.
\begin{proof}
重复。
\end{proof}
\end{enumerate}
\endinput