\section{2012}
\begin{enumerate}[1~]
\renewcommand{\labelenumi}{\textbf{\theenumi. }}
\renewcommand{\Im}{\text{Im }}
\item[一、]
填空题 (每小题 4 分,共 20 分)
\begin{enumerate}[1.~]
\item
三阶行列式有两个元素为 $4$,其余为 $\pm 1$,则此行列式可能的最大值为(\quad)。 
\begin{solution}
$$\left| \begin{matrix}
	4&		1&		1\\
	-1&		4&		-1\\
	-1&		-1&		1\\
\end{matrix} \right|=19.$$
\end{solution}

\item
设 $\boldsymbol{\gamma} _ { 1 } , \boldsymbol{\gamma} _ { 2 } , \boldsymbol{\alpha} , \boldsymbol{\beta}$皆为三维列向量,$A = \left( \boldsymbol{\alpha} , 2 \boldsymbol{\gamma} _ { 1 } , 3 \boldsymbol{\gamma} _ { 2 } \right) , B = \left( \boldsymbol{\beta} , \boldsymbol{\gamma} _ { 1 } , 2 \boldsymbol{\gamma} _ { 2 } \right)$ 且 $| A | = 18 , | B | = 4$,则 $|A - B| =$(\quad)。 
\begin{solution}
$|A-B|=|(\boldsymbol{\alpha}-\boldsymbol{\beta}, \boldsymbol{\gamma}_1, \boldsymbol{\gamma}_2)|=|(\boldsymbol{\alpha}, \boldsymbol{\gamma}_1, \boldsymbol{\gamma}_2)|-|(\boldsymbol{\beta}, \boldsymbol{\gamma}_1, \boldsymbol{\gamma}_2)|=\frac16 |A|- \frac12 |B|=3-2=1$。
\end{solution}

\item
三阶方阵 $A$ 的特征值为 $1 , - 1, 2$,则 $A ^ { 2 } + 4 A ^ { - 1 }$ 的特征值(\quad)。 

\begin{solution}
显然,$5, -3, 6$。
\end{solution}

\item
若不可约多项式 $ p(x) $ 是 $ f^{(k)}(x)$ 的 $s$ 重因子,且 $p ( x ) | f ( x )$,那么 $ p(x) $( \quad )$f(x)$ 的 $s + k$ 重因子。

\begin{solution}
设 $p(x)$ 是 $f(x)$ 的 $t$ 重因式,则 $p(x)$ 是 $f^{(k)}(x)$ 的 $t - k$ 重因式,因此 $t - k =s$,所以 $p(x)$ 是 $f(x)$ 的 $s+ k$ 重因式。
\end{solution}

\item
设 
$$
A = \left( \begin{matrix} { 0 } & { a } & { 9 } \\ { 0 } & { 6 } & { 0 } \\ { 4 } & { 2 b } & { 0 } \end{matrix} \right)
$$
相似于对角阵,则 $a$ 与 $b$ 的关系式为(\quad)。

\begin{solution}
矩阵 $A$ 的特征多项式为\[
f(\lambda) = (\lambda + 6) (\lambda - 6)^2.
\]
因为 $A$ 相似于对角阵,所以 $A$ 的最小多项式为 \[
m(\lambda) = (\lambda + 6) (\lambda - 6).
\]
因此 $(A + 6 I) (A - 6 I) = 0$,即\[
\left( \begin{matrix}
	0&		6a+18b&		0\\
	0&		0&		0\\
	0&		4a+12b&		0\\
\end{matrix} \right) =0.
\]
于是 $a + 3b = 0$。
\end{solution}

\item
设 $\mathbb{R}^2$ 中的内积为 
$$
(\boldsymbol{\alpha} , \boldsymbol{\beta} ) = \boldsymbol{\alpha} ^ { \prime } A \boldsymbol{\beta} , A = \left( \begin{array} { l l } { 2 } & { 1 } \\ { 1 } & { 2 } \end{array} \right),
$$
则
$$
\left(\begin{array} { l } { 1 } \\ { 0 } \end{array} \right) , \left( \begin{array} { l } { 0 } \\ { 1 } \end{array} \right)
$$
在此内积之下的度量矩阵为 (\quad)。 

\begin{solution}
设所求的度量矩阵为 $B$,记 $\boldsymbol{\alpha}_1=(1, 0)', \boldsymbol{\alpha}_2=(0, 1)'$,则\[
A=
\left( \begin{matrix}
	\left( \boldsymbol{\alpha} _1,\boldsymbol{\alpha} _1 \right)&		\left( \boldsymbol{\alpha} _1,\boldsymbol{\alpha} _2 \right)\\
	\left( \boldsymbol{\alpha} _2,\boldsymbol{\alpha} _1 \right)&		\left( \boldsymbol{\alpha} _2,\boldsymbol{\alpha} _2 \right)\\
\end{matrix} \right) =\left( \begin{matrix}
	\boldsymbol{\alpha} _1'A\boldsymbol{\alpha} _1&		\boldsymbol{\alpha} _1'A\boldsymbol{\alpha} _2\\
	\boldsymbol{\alpha} _2'A\boldsymbol{\alpha} _1&		\boldsymbol{\alpha} _2'A\boldsymbol{\alpha} _2\\
\end{matrix} \right) =\left( \begin{matrix}
	2&		1\\
	1&		2\\
\end{matrix} \right) .
\]
\end{solution}
\item
令$A \in \mathbb { R } ^ { 4 \times 4 }$ 的特征值为 $1, 2, 3, 4$, 则$\operatorname { tr } \left( A ^ { 2 } \right) =$(\quad)。 
\begin{solution}
$1+4+9+16=30$。
\end{solution}
\item
设$A , B \in \mathbb { R } ^ { m \times n }$,在矩阵方程 $AX = B$ 有解的充要条件为 (\quad)。
 \begin{solution}
${\rm rank}(A)={\rm rank}(A\ B)$。
\end{solution}
\item
设 $A$ 为正交矩阵, 且 $|A| = -1$,则 $A$ 必有特征值为(\quad)。 
\begin{solution}
$-1$。若全为$1$,则 $|A|=1$。
\end{solution}
\item
 向量组 $\boldsymbol{\alpha} _ { 1 } = ( 1,1 , k ) , \boldsymbol{\alpha} _ { 2 } = ( 1 , k , 1 ) , \boldsymbol{\alpha} _ { 3 } = ( k , 1,1 )$ 是线性无关的,则 $k$(\quad)。 
 \begin{solution}
\[|(\boldsymbol{\alpha}_1', \boldsymbol{\alpha}_2', \boldsymbol{\alpha}_3')|=\left|
\begin{array}{ccc}
 1 & 1 & k \\
 1 & k & 1 \\
 k & 1 & 1 \\
\end{array}
\right|=(k-1)^2(k+2).\]
所以 $k\ne1$且$k\ne -2$。
\end{solution}
\end{enumerate}

\item[二、](15分)
设
\[
A = \left( \begin{array} { c c c } { a } & { 1 } & { 2 } \\ { 1 } & { b } & { 1 } \\ { 1 } & { 3 b } & { 1 } \end{array} \right).
\]
$B$ 是三阶非零方阵,且 $AB = O$,求 $a, b$ 以及 $B$ 的秩。
\begin{solution}
对矩阵$A$作初等行变换得\[
A\rightarrow \left( \begin{matrix}
	1&		0&		1\\
	0&		1&		2-a\\
	0&		0&		-b\left( 2-a \right)\\
\end{matrix} \right) 。
\]
所以 ${\rm rank}(A)\ge 2$。\\
又 $B$ 是三阶非零方阵,所以  ${\rm rank}{B}\ge 1$,于是由 Sylvester 不等式得\[
{\rm rank}(A) + {\rm rank}(B) \le 3.
\]
因此 ${\rm rank}(A)\le 2$,所以 ${\rm rank}(A) = 2$,即 $b=0$ 且 $a=2$,此时${\rm rank}(B) = 1$。
\end{solution}
\item[三、](15分)
设 $A$ 是 $n$ 阶正定矩阵,$B$ 为 $n$ 阶实方阵,证明:\\
(1) 若 $B$ 正定,则 $AB$ 的特征值皆大于 $0$。\\
(2) 若 $B$ 正定,且 $AB = BA$,则 $AB$ 正定。
\begin{proof}
(1) 因为 $A$,$B$都是 $n$ 阶正定矩阵,所以存在正定矩阵 $ S $,$ T $,使得\[
S^{-1} (AB) S = S^{-1} S^2 T^2 S = S T^2 S = (TS)'(TS) =: C.
\]
因为 $C' = (TS)'(TS) = C$,所以 $C$ 是实对称矩阵。\\
又对任意 $ n $ 维非零实向量 $\boldsymbol{x}$,由 $TS$ 可逆知 $TS\boldsymbol{x}\ne0$,故\[
\boldsymbol{x}'C\boldsymbol{x} = \boldsymbol{x}' (TS)'(TS)\boldsymbol{x} = (TS\boldsymbol{x})'(TS\boldsymbol{x}) > 0.
\]
因此 $C$ 是正定矩阵,从而 $C$ 的特征值全大于零。因为 $AB$ 与 $C$ 相似,所以 $AB$ 的特征值全大于零。\\
(2) 因为 $AB = AB$,所以\[
(AB)'=B'A'=BA=AB.
\]
所以 $AB$ 是实对称矩阵。由 (1) 知 $AB$ 的特征值全大于零,故 $AB$ 是正定矩阵。
\end{proof}
\item[四、]
$A$ 为 $n$ 阶方阵,如果 $A^2 = A$,其中 $E$ 是 $n$ 阶单位矩阵。

\item[五、]
设\[
A = \left( \begin{array} { c c c } { 1 } & { 4 } & { 2 } \\ { 0 } & { - 3 } & { 4 } \\ { 0 } & { 4 } & { 3 } \end{array} \right).
\]
试求 $A^n$。
\begin{solution}
矩阵 $A$ 的特征多项式为\[
|\lambda I-A|=\left(
\begin{array}{ccc}
 \lambda-1 & -4 & -2 \\
 0 & \lambda+3 & -4 \\
 0 & -4 & \lambda-3 \\
\end{array}
\right) = (\lambda+5) (\lambda-1) (\lambda-5).
\]
所以 $A$ 的特征值为 $-5$,$1$,$5$。\\
解线性方程组$(-5I-A)x=0$,得一个基础解系:\[
\boldsymbol{\xi}_1=(1, -2, 1)'.
\]
解线性方程组$(1I-A)x=0$,得一个基础解系:\[
\boldsymbol{\xi}_1=(1, 0, 0)'.
\]
解线性方程组$(5I-A)x=0$,得一个基础解系:\[
\boldsymbol{\xi}_1=\left(1, \frac12, 1\right)'.
\]
令\[
P=\left( \begin{matrix}
	1&		1&		1\\
	-2&		0&		\frac{1}{2}\\
	1&		0&		1\\
\end{matrix} \right) ,
\]
则\[
P^{-1}AP={\rm diag}\{-5, 1, 5\}.
\]
因此\begin{align*}
A^n&=\left( P\left( \begin{matrix}
	-5&		0&		0\\
	0&		1&		0\\
	0&		0&		5\\
\end{matrix} \right) P^{-1}\right)^n=P\left( \begin{matrix}
	(-5)^n&		0&		0\\
	0&		1&		0\\
	0&		0&		5^n\\
\end{matrix} \right) P^{-1}\\
&=\left( \begin{matrix}
	1&		1&		1\\
	-2&		0&		\frac{1}{2}\\
	1&		0&		1\\
\end{matrix} \right)
\left( \begin{matrix}
	(-5)^n&		0&		0\\
	0&		1&		0\\
	0&		0&		5^n\\
\end{matrix} \right)
\left(
\begin{array}{ccc}
 0 & -\frac{2}{5} & \frac{1}{5} \\
 1 & 0 & -1 \\
 0 & \frac{2}{5} & \frac{4}{5} \\
\end{array}
\right) \\
&=\left(
\begin{array}{ccc}
 1 & 2\cdot 5^{n-1}-2 (-1)^n 5^{n-1} & -1+4\cdot 5^{n-1}+(-1)^n 5^{n-1} \\
 0 & 5^{n-1}+4 (-1)^n 5^{n-1} & 2\cdot 5^{n-1}-2 (-1)^n 5^{n-1} \\
 0 & 2\cdot 5^{n-1}-2 (-1)^n 5^{n-1} & 4\cdot 5^{n-1}+(-1)^n 5^{n-1} \\
\end{array}
\right).
\end{align*}
\end{solution}
\item[六、]
设 $A$ 为 $n$ 阶实方阵,已知 $A$ 的特征值全为实数,且\[
A A ^ { T } = A ^ { T } A.
\]
证明:$A$ 必为对称矩阵。
\begin{proof}
\begin{lemma}\label{shijuzhenzhengjiaoxiangsishangsanjiao}
$n$ 级实矩阵 $A$ 正交相似于一个上三角矩阵的充分必要条件是:$A$ 的特征多项式在复数域中的根都是实数。
\end{lemma}
\begin{subproof}
必要性。设 $n$ 级实矩阵 $A$ 正交相似于一个上三角矩阵 $B = (b_{ij})$,则 \[
|x I - A| = |I - B| = (\lambda - b_1) (\lambda - b_2) \cdots (\lambda - b_m).\]
这表明 $|A I - A|$ 的根 $b_{11}, b_{22}, \dots, b_{nn}$ 都是实数。

充分性。对实矩阵的级数作数学归纳法。$n = 1$ 时,显然命题为真。假设对于 $n - 1$ 级实矩阵命题为真,现在来看 $n$ 级实矩阵 $A$。由于 $A$ 的特征多项式在复数域中的根都是实数,因此可以取 $A$ 的一个特征值 $\lambda_1$。设 $\boldsymbol{\eta}_1$ 是 $A$ 的属于 $\lambda_1$ 的一个特征向量,且 $|\lambda| = 1$。把 $\boldsymbol{\eta}_1$ 扩充成 $\mathbb{R}^n$ 的一个基,然后经过施密特正交化和单位化,得到 $\mathbb{R}^n$ 的一个标准正交基:
$\boldsymbol{\eta}_{1}, \boldsymbol{\eta}_{2}, \dots, \boldsymbol{\eta}_{n}$。令 $T_1 = (\boldsymbol{\eta}_{1}, \boldsymbol{\eta}_{2}, \dots, \boldsymbol{\eta}_{n})$,则 $T_1$ 是正交矩阵。
\[
T_1^{-1} A T_1 = T_1^{-1}(A\boldsymbol{\eta}_{1}, A\boldsymbol{\eta}_{2}, \dots, A\boldsymbol{\eta}_{n}) = (T_1^{-1} \lambda_1 \boldsymbol{\eta}_{1}, T_1^{-1} A\boldsymbol{\eta}_{2}, \dots, T_1^{-1}A\boldsymbol{\eta}_{n}).
\]
由于 $T_1^{-1}T_1 = I$,因此 $T_1^{-1} \boldsymbol{\eta}_1 = \boldsymbol{\varepsilon_1}$。从而
\[
T_1^{-1}AT_1=\left( \begin{matrix}
	\lambda _1&		\boldsymbol{\boldsymbol{\alpha} }\\
	0&		B\\
\end{matrix} \right) ,
\]
于是 $|\lambda I-A| = (\lambda - \lambda_1) |\lambda I - B|$。因此 $n - 1$ 级实矩阵 $B$ 的特征多项式在复数域中的根都是实数。从而对 $B$ 可用归纳假设:存在 $n - 1$ 级正交矩阵 $T_2$,使得 $T_2^{-1} B T_2$ 为上三角矩阵。

令
\[
T = T_1 \left( \begin{matrix}
	1&		\boldsymbol{\boldsymbol{\alpha} }\\
	0&		T_2\\
\end{matrix} \right) ,
\]
则 $T$ 是 $n$ 级正交矩阵,且
\begin{align*}
 T^{-1} A T 
 &=\left( \begin{array}{cc}{1} & {\mathbf{0}} \\ 
 {\mathbf{0}} & {T_{2}}\end{array}\right)^{-1} 
 T_{1}^{-1} A T_{1} 
 \left( \begin{array}{cc}{1} & {\mathbf{0}} \\ 
 {\mathbf{0}} & {T_{2}}\end{array}\right)
 =\left( \begin{array}{cc}{1} & {\mathbf{0}} \\ 
 {\mathbf{0}} & {T_{2}^{-1}}\end{array}\right) 
 \left( \begin{array}{cc}{\lambda_{1}} & {\boldsymbol{\boldsymbol{\alpha}}} \\ 
 {\mathbf{0}} & {B}\end{array}\right) 
 \left( \begin{array}{cc}{1} & {\mathbf{0}} \\ 
 {\mathbf{0}} & {T_{2}}\end{array}\right) \\ 
 &=\left( \begin{array}{cc}{\lambda_{1}} & {\boldsymbol{\boldsymbol{\alpha}} T_{2}} \\ 
 {\mathbf{0}} & {T_{2}^{-1} B T_{2}}\end{array}\right) 
\end{align*}
因此 $T^{-1} A T$ 是上三角矩阵。

据数学归纳法原理,对一切正整数 $n$,此命题为真。
\end{subproof}
由引理 \eqref{shijuzhenzhengjiaoxiangsishangsanjiao} 得,存在 $n$ 级正交矩阵 $T$,使得 $T^{-1} A T = B$,其中 $B = (b_{ij})$ 是上三角矩阵。从而 $T'A'(T^{-1})' = B'$,即 $T^{-1} A' T = B'$。由于 $AA' = A'A$,因此 $BB' = B' B$。于是\begin{equation}\label{bikbki}
\sum_{k = 1}^n  b_{ik}^2 = \sum_{k = 1}^n b_{ki}^2, i = 1, 2, \dots ,n.
\end{equation}
当 $i = 1$ 时, \eqref{bikbki} 式成为\begin{equation*}\label{b1k2b11}
\sum_{k = 1}^n b_{1k}^2 = b_{11}^2.
\end{equation*}
由此推出,$b_{12}^2 + b_{13}^2 + \cdots + b_{1n}^2 = 0$。从而 $b_{12} = b_{13} = \cdots = b_{1n} = 0$。

当 $i = 2$ 时, \eqref{bikbki} 式成为
\begin{equation*}
\sum_{k = 1}^n b_{2k}^2 = b_{12}^2 + b_{22}^2 = b_{22}^2.
\end{equation*}
由此推出,$b_{23} = b_{24} = \cdots = b_{2n} = 0$。

依次下去,可得\[
b_{34} = \cdots = b_{3n} = 0, \dots , b_{n-1, n} = 0.
\]
因此 $B$ 是对角矩阵。由于 $A$ 正交相似于对角矩阵 $B$,因此 $A$ 是对称矩阵。
\end{proof}

\item[七、]
设\[
A = \left( \begin{array} { c c c } { a } & { 1 } & { 2 } \\ { 1 } & { b } & { 1 } \\ { 1 } & { 3 b } & { 1 } \end{array} \right).
\]
$B$ 是三阶非零方阵,且 $AB = O$,求 $a, b$ 以及 $B$ 的秩。
\begin{solution}
重复。
\end{solution}
\item[八、](15 分)
设$A、B$ 为 $n$ 元实对称矩阵,且 $B$ 正定,求证:存在一个实可逆阵 $P$ 使得 $P^TAP$ 和 $P^TBP$ 同时为对角阵。
\begin{proof}
由于 $B$ 是 $n$ 元正定矩阵。因此 $A \simeq I$。从而存在 $n$ 元实可逆矩阵 $P_1$,使得 $P_1^TBP_1 = I$。\\
由于 $(P_1^T AP_1)^T = P_1^T A^T P_1 = P_1^T AP_1$,因此 $P_1^T AP_1$ 是 $n$ 元实对称矩阵。于是存在 $n$ 级正交矩阵 $T$,使得\[
T ^ T \left( P _ { 1 } ^ T A P _ { 1 } \right) T = T ^ { - 1 } \left( P _ { 1 } ^ T A P _ { 1 } \right) T = \operatorname { diag } \left\{ \mu _ { 1 } , \mu _ { 2 } , \dots , \mu _ { n } \right\}.
\]
令$P=P_1 T$,则$P$是实可逆矩阵,且使得\begin{align*}
P ^ T B P &= \left( P _ { 1 } T \right) ^ T B \left( P _ { 1 } T \right) = T ^ T \left( P _ { 1 } ^ T B P _ { 1 } \right) T = T ^ T I T = I,\\
P ^ T A P &= T ^ T \left( P _ { 1 } ^ T A P _ { 1 } \right) T = \operatorname { diag } \left\{ \mu _ { 1 } , \mu _ { 2 } , \dots , \mu _ { n } \right\}.
\end{align*}
\end{proof}

\end{enumerate}
\endinput