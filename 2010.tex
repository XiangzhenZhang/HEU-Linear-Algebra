\section{2010}
\begin{enumerate}[1~]
\renewcommand{\labelenumi}{\textbf{\theenumi. }}
\renewcommand{\Im}{\text{Im }}
\item[一、]
填空题 (每小题 4 分,共 20 分)
\begin{enumerate}[1.~]
\item
设 $f ( x ) = x ^ { 5 } - 2 x ^ { 4 } + \frac { 3 } { 2 } x ^ { 3 } - 3 x ^ { 2 } + \frac { 1 } { 2 } x - 1$,则 $f(x)$ 的有理根是(\quad)。

\begin{solution}
\begin{theorem}\label{duoxiangshixishu}
设$f ( x ) = a _ { n } x ^ { n } + a _ { n - 1 } x ^ { n - 1 } + \dots + a _ { 1 } x + a _ { 0 }$是一个次数 $n$ 大于 $0$ 的整系数多项式,如果$\frac{q}{p}$是$f(x)$的一个有理根,其中$p$,$q$是互素的整数,那么$p|a_n$,$q|a_0$。
\end{theorem}

由定理 \eqref{duoxiangshixishu} 知, $2f(x)$ 的有理根只可能是 $\pm\frac12$ 或 $\pm1$ 或 $\pm2$。直接验算可知 $2f(x)$ 的有理根为 $1$,从而 $f(x)$ 的有理根为$1$。
 \end{solution}
 
\item
多项式$ f(x) = x^3 + 5x − 10 $在有理数域上是(\quad)(可约或不可约)的。

\begin{solution}
因为存在素数5,使得\begin{align*}
&5|5, 5|-10;\\
&5\nmid 1;\\
&5^2\nmid -10.
\end{align*}
所以由 Eisenstein 判别法知,$f(x)$在有理数域上式不可约的。
\end{solution}

\item
设
$$
D = \left| \begin{array} { c c c c } { 1 } & { 2 } & { 3 } & { 4 } \\ { 3 } & { 2 } & { 4 } & { 1 } \\ { 0 } & { 2 } & { 3 } & { 1 } \\ { 0 } & { 2 } & { 4 } & { 3 } \end{array} \right|,
$$
$A_{ij}$ 表示元素 $a_{ij}$ 的代数余子式,则 $2A_{14} + A_{24} + A_{34} + A_{44} = $(\quad)。
\begin{solution}
$2A_{14} + A_{24} + A_{34} + A_{44} = \left| \begin{matrix}
	1&		2&		3&		2\\
	3&		2&		4&		1\\
	0&		2&		3&		1\\
	0&		2&		4&		1\\
\end{matrix} \right|=\left| \begin{matrix}
	1&		2&		3&		4\\
	3&		2&		4&		1\\
	0&		2&		3&		1\\
	0&		2&		4&		3\\
\end{matrix} \right|-\left| \begin{matrix}
	1&		2&		3&		2\\
	3&		2&		4&		0\\
	0&		2&		3&		0\\
	0&		2&		4&		2\\
\end{matrix} \right|=-22-(-16)=-6.$
\end{solution}

\item
 对称多项式$f \left( x _ { 1 } , x _ { 2 } , x _ { 3 } \right) = x _ { 1 } ^ { 2 } + x _ { 2 } ^ { 2 } + x _ { 3 } ^ { 2 }$表示为初等对称多项式是(\quad)。
 \begin{solution}
$f \left( x _ { 1 } , x _ { 2 } , x _ { 3 } \right)$的首项为$x_1^2$,首项的幂指数组为$(2, 0, 0)$。$f \left( x _ { 1 } , x _ { 2 } , x _ { 3 } \right)$是二次齐次对称多项式,$f_i (i=1, 2, \dots, s)$也是二次齐次对称多项式,它们的首项幂指数组
$(p_1, p_2, p_3)$应当满足:\[
p _ { 1 } + p _ { 2 } + p _ { 3 } = 2 , \quad 2 \geqslant p _ { t } \geqslant p _ { 2 } \geqslant p _ { 3 }.
\]
满足这两个条件的非负整数3元组$(p_1, p_2, p_3)$只可能是:\[
( 2, 0, 0 ) , (1,1,0) .
\]
它们分别是$f$,$f_1$的首项幂指数组,于是$f_2=0$,且
\begin{align*}
& \Phi _ { 1 } \left( x _ { 1 }, x_2 , x _ { 3 } \right) = \sigma _ { 1 } ^ { 2 - 0 } \sigma _ { 2 } ^ { 0 } \sigma _ { 3 } ^ { 0 }  = \sigma _ { 1 } ^ { 2 },  \\ 
&{ \Phi _ { 2 } \left( x _ { 1 }, x_2 , x _ { 3 } \right) = a \sigma _ { 1 } ^ { 1 - 1 } \sigma _ { 2 } ^ { 1 - 0 } \sigma _ { 3 } ^ { 0 } = a \sigma _ { 2 } }.
\end{align*}
于是\[
f(x_1, x_2, x_3)=\Phi_2+\Phi_1=a\sigma_2+\sigma_1^2.
\]
为了确定$a$,$b$的值,$x_1, x_2, x_3$分别用 $x_1=x_2=x_3=1$ 代入,得\[
3=3a+9.
\]
因此 $a=-2$,故 $f(x_1, x_2, x_3)=\sigma_1^2-2\sigma_2$。
 \end{solution}
 
\item
将矩阵 $\left( \begin{smallmatrix} { 2 } & { 1 } \\ { 4 } & { 3 } \end{smallmatrix} \right)$ 写成初等矩阵之积(\quad)。

\begin{solution}
$\left( \begin{matrix}
	1&		0\\
	2&		1\\
\end{matrix} \right) \left( \begin{matrix}
	1&		1\\
	0&		1\\
\end{matrix} \right) \left( \begin{matrix}
	2&		0\\
	0&		1\\
\end{matrix} \right) $.

\end{solution}

\item
设 $\boldsymbol{\alpha} _ { 1 } = ( 1, 1, k ) , \boldsymbol{\alpha} _ { 2 } = ( 1 , k , 1 ) , \boldsymbol{\alpha} _ { 3 } = ( k, 1, 1 )$ 是线性无关的,则 $k$ 的取值为 (\quad)。
\begin{solution}
由题意,\[
\left| \begin{matrix}
	1&		1&		k\\
	1&		k&		1\\
	k&		1&		1\\
\end{matrix} \right|\ne 0.
\]
因此,$-k^3+3k-2\ne 0$,解得$k\ne -2$且$k\ne 1$。
\end{solution}

\item
二次型 $f \left( x , x _ { 2 } , x _ { 3 } \right) = x _ { 1 } ^ { 2 } + 2 x _ { 2 } ^ { 2 } + 3 x _ { 3 } ^ { 2 } + 2 t x _ { 2 } x _ { 3 }$ 是正定的,则 $t$ 的取值范围为(\quad)。
\begin{solution}
二次型$f(x_1, x_2, x_3)$的矩阵为:\[
A=\left( \begin{matrix}
	1&		0&		0\\
	0&		2&		t\\
	0&		t&		3\\
\end{matrix} \right) .
\]
因为二次型$f(x_1, x_2, x_3)$是正定的,所以矩阵$A$是正定矩阵,所以$|A|>0$,由此得$-\sqrt{6}<t<\sqrt{6}$。
\end{solution}

\item
$\mathbb{R}^3$中的向量 $\boldsymbol{\alpha} = \left( a _ { 1 } , a _ { 1 } , a _ { 3 } \right)$ 在基 $\boldsymbol{\alpha} _ { 1 } = ( 1, 1, 1 ) , \boldsymbol{\alpha} _ { 2 } = ( 0, 1, 1 ) , \boldsymbol{\alpha} _ { 3 } = ( 0, 0, 1 )$ 下的坐标是(\quad)。

\begin{solution}
设$\boldsymbol{\alpha}$在基$\boldsymbol{\alpha}_1, \boldsymbol{\alpha}_2, \boldsymbol{\alpha}_3$下的坐标是$(x_1, x_2, x_3)$。则\[
\left\{ \begin{array}{l}
	a_1=x_1\\
	a_2=x_1+x_2\\
	a_3=x_1+x_2+x_3\\
\end{array} \right. 
\]
因此坐标为 $(a_1, a_2-a_1, a_3-a_2)$。
\end{solution}

\item
在 $\mathbb{R}^3$ 中与向量 $(1, 1, 2)$ 和 $(−1, 1, 0)$ 都正交的单位向量是(\quad)。
\begin{solution}
设满足题设条件的单位向量为$(\boldsymbol{\eta}_1, \boldsymbol{\eta}_2, \boldsymbol{\eta}_3)$,则\[
\left\{ \begin{array}{l}
	\boldsymbol{\eta} _1+\boldsymbol{\eta} _2+2\boldsymbol{\eta} _3=0,\\
	-\boldsymbol{\eta} _1+\boldsymbol{\eta} _2=0,\\
	\sqrt{\boldsymbol{\eta}_1^2+\boldsymbol{\eta}_2^2+\boldsymbol{\eta}_3^2}=1.\\
\end{array} \right. 
\]
解得$(\boldsymbol{\eta}_1, \boldsymbol{\eta}_2, \boldsymbol{\eta}_3)=\pm(\frac{\sqrt{3}}{3}, \frac{\sqrt{3}}{3}, -\frac{\sqrt{3}}{3})$。
\end{solution}

\item
令 $A \in \mathbb { R } ^ { 4 \times 4 }$ 的特征值为 1,2,3,4,则 ${\rm tr}(A^2) =$(\quad)。

\begin{solution}
由2014年的填空题的第九题下的Remarks知,$A^2$的特征值为1, 4, 9, 16,所以${\rm tr}(A^2) =1+4+9+16=30$。
\end{solution}
\end{enumerate}

\item[二、](15分)
在 $\mathbb{R}^3$ 中,线性变换 $\mathscr{A}$ 定义为
\[
\left\{ \begin{array} { l } { \mathscr { A } \boldsymbol{\alpha} _ { 1 } = ( 1,0,0 ) } ,\\ { \mathscr { A } \boldsymbol{\alpha} _ { 2 } = ( 3,3,2 ) } ,\\ { \mathscr { A } \boldsymbol{\alpha} _ { 3 } = ( 3,3,1 ) }. \end{array} \right.
\]
其中
\[
\left\{ \begin{array} { l } { \boldsymbol{\alpha} _ { 1 } = ( 1,0,0 ) }, \\ { \boldsymbol{\alpha} _ { 2 } = ( 1,1,0 ) }, \\ { \boldsymbol{\alpha} _ { 3 } = ( 1,1,1 ) }. \end{array} \right.
\]

(1) 求$\mathscr{A}$在基$\boldsymbol{\alpha}_1, \boldsymbol{\alpha}_2, \boldsymbol{\alpha}_3$下的矩阵$B$。

(2) 求$\mathscr{A}$的特征值与特征向量。

\begin{solution}
(1) 因为\[
\left( \begin{array}{c}
	\mathscr{A}\boldsymbol{\alpha} _1\\
	\mathscr{A}\boldsymbol{\alpha} _2\\
	\mathscr{A}\boldsymbol{\alpha} _3\\
\end{array} \right) =B\left( \begin{array}{c}
	\boldsymbol{\alpha} _1\\
	\boldsymbol{\alpha} _2\\
	\boldsymbol{\alpha} _3\\
\end{array} \right) .
\]
所以\[
B=\left( \begin{array}{c}
	\mathscr{A}\boldsymbol{\alpha} _1\\
	\mathscr{A}\boldsymbol{\alpha} _2\\
	\mathscr{A}\boldsymbol{\alpha} _3\\
\end{array} \right) \left( \begin{array}{c}
	\boldsymbol{\alpha} _1\\
	\boldsymbol{\alpha} _2\\
	\boldsymbol{\alpha} _3\\
\end{array} \right) ^{-1}=\left(
\begin{array}{ccc}
 1 & 0 & 0 \\
 0 & 1 & 2 \\
 0 & 2 & 1 \\
\end{array}
\right).
\]
(2) $\mathscr{A}$ 的特征多项式为\[
|\lambda I-B|=-(x+1) (x-1) (x-3).
\]
所以 $\mathscr{A}$ 的特征值为 $-1$, $1$ , $3$。\\
解线性方程组 $(-I-B)x=0$,得一个基础解系:\[
\boldsymbol{\xi}_1=\left( \begin{array}{c}
	0\\
	-1\\
	1\\
\end{array} \right) .
\]
解线性方程组 $(I-B)x=0$,得一个基础解系:\[
\boldsymbol{\xi}_1=\left( \begin{array}{c}
	1\\
	0\\
	0\\
\end{array} \right) .
\]
解线性方程组 $(3I-B)x=0$,得一个基础解系:\[
\boldsymbol{\xi}_1=\left( \begin{array}{c}
	0\\
	1\\
	1\\
\end{array} \right) .
\]
因此 $\mathscr{A}$ 的对应于 $-1$ 的特征向量为 $k(0, -1, 1)', k\in \mathbb{R}, k\ne 0$,$\mathscr{A}$ 的对应于 $-、1$ 的特征向量为 $l(1, 0, 0)', l\in \mathbb{R}, l\ne 0$,$\mathscr{A}$ 的对应于 $-1$ 的特征向量为 $m(0, 1, 1)', m\in \mathbb{R}, m\ne 0$。
\end{solution}

\item[三、](15分)
设\[
V = \left\{ A \in \mathbb { R } ^ { n \times n } | \operatorname { tr } ( A ) = 0 \right\} , W = \{ a E | a \in \mathbb { R } \}.
\]

(1) 求证$V$为$\mathbb{R}^{n\times n}$的子空间,并求$\dim V$。

(2) 求证$\mathbb{R}^{n\times n}=V\oplus W$。

\begin{proof}

(1) 易知$V$是$\mathbb{R}^{n\times n}$的子集。因为$0  \in V$,所以$V$非空。又
\begin{align*}
&B , C \in V \quad \Longrightarrow {\rm tr}(A)={\rm tr}(B)=0 \Longrightarrow {\rm tr}(A+B)={\rm tr}(A)+{\rm tr}(B)=0\Longrightarrow \quad A + B \in V,\\
&A \in V , k \in \mathbb{R}\Longrightarrow{\rm tr}(A)=0\Longrightarrow{\rm tr}(kA)=k{\rm tr}(A)=0 \quad \Longrightarrow k \boldsymbol{\alpha} \in V.
\end{align*}

因此$V$对于矩阵的加法和纯量乘法封闭,于是$V$是$\mathbb{R}^{n\times n}$的子空间。

\begin{align*}
X&=(x_{ij})\in V\\
&\Longleftrightarrow x _ { 11 } + x _ { 22 } + \dots + x _ { n n } = 0\\
& \begin{aligned} \Longleftrightarrow X  = & x _ { 11 } E _ { 11 } + x _ { 12 } E _ { 12 } + \cdots + x _ { 1 n } E _ { 1 n } \\ & + x _ { 21 } E _ { 11 } + x _ { 22 } E _ { 22 } + \cdots + x _ { 2 n } E _ { 2 n } \\ & + \cdots \\ & + x _ { n 1 } E _ { n 1 } + x _ { n 2 } E _ { n 2 } + \cdots - \left( x _ { 11 } + x _ { 22 } + \cdots + x _ { n - 1 , n - 1 } \right) E _ { n n } \end{aligned}\\
& \begin{aligned}\Longleftrightarrow X = & x _ { 11 } \left( E _ { 11 } - E _ { n n } \right) + x _ { 12 } E _ { 12 } + \cdots + x _ { 1 n } E _ { 1 n } \\ & + x _ { 21 } E _ { 21 } + x _ { 22 } \left( E _ { 22 } - E _ { n n } \right) + \cdots + x _ { 2 n } E _ { 2 n } \\ & + \cdots \\ & + x _ { n 1 } E _ { n 1 } + x _ { n 2 } E _ { n 2 } + \cdots + x _ { n , n - 1 } E _ { n , n - 1 } \end{aligned}
\end{align*}

又容易验证 $E _ { 11 } - E _ { n n } , E _ { 12 } , \dots , E _ { 1 n } , E _ { 21 } , E _ { 22 } - E _ { n n } , E _ { 23 } , \dots , E _ { 2 n } , \dots , E _ { n - 1,1 } , \dots , E _ { n - 1 , n - 1 } - E _ { n n }, E _ { n - 1 , n } , E _ { n 1 } , E _ { n 2 } , \dots , E _ { n , n - 1 }$ 线性无关。因此它们就是 $V$ 的一个基,从而
\[
\dim V=n^2-1.
\]

(2) 第 1 步,证明$\mathbb{R}^{n\times n}=V+W$。任取$A=(a_{ij})\in \mathbb{R}^{n\times n}$,想把$A$表示成$A_1+A_2$,其中$A_1\in V, A_2\in W$。设$A_2=kI$,令$A_1=A-A_2$,它应满足${\rm tr}(A_1)=0$,即
$\left( a _ { 11 } + a _ { 22 } + \dots + a _ { n n } \right) - n k = 0$。于是有$k=\frac{1}{n}(a_{11}+a_{22} +\cdots+a+{nn})$。由此构造的$A_1$与$A_2$就满足$A=A_1+A_2$。于是$A\in V+W$。从而得出\[
\mathbb{R}^{n\times n}=V+W.
\]
第2步,证明和$V\oplus W$是直和。因为\[
\dim V+\dim W=n^2-1+1=n^2=\dim \mathbb{R}^{n\times n}=\dim(V+W).\]
所以和$V\oplus W$是直和。
\end{proof}

\item[四、](15分)
设 $4$ 元二次型\[
f \left( x _ { 1 } , x _ { 2 } , x _ { 3 } , x _ { 4 } \right) = 2 x _ { 1 } x _ { 2 } + 2 x _ { 3 } x _ { 4 }.
\]

(1) 写出二次型 $f(x_1, x_2, x_3, x_4)$ 的矩阵表达式$ f(x_1, x_2, x_3, x_4) = X^T AX$。

(2) 求 $A$ 的特征值和特征向量。

(3) 求正交矩阵 $P$,使得 $P^{-1}AP = \Lambda$,其中$\Lambda$ 是对角阵。

(4) 写出二次型 $f(x_1, x_2, x_3, x_4)$ 的标准型。

\begin{solution}
(1) 
$$
X^T A X=\left( x_1,x_2,x_3,x_4 \right) \left( \begin{matrix}
	0&		1&		0&		0\\
	1&		0&		0&		0\\
	0&		0&		0&		1\\
	0&		0&		1&		0\\
\end{matrix} \right) \left( \begin{array}{c}
	x_1\\
	x_2\\
	x_3\\
	x_4\\
\end{array} \right).
$$

(2) 矩阵$A$的特征多项式为:\[
f(\lambda)=|\lambda I-A|=\left| \begin{matrix}
	\lambda&		-1&		0&		0\\
	-1&		\lambda&		0&		0\\
	0&		0&		\lambda&		-1\\
	0&		0&		-1&		\lambda\\
\end{matrix} \right|=\left( \lambda +1 \right) ^2\left( \lambda -1 \right) ^2.
\]
所以$A$的特征值为-1(二重),1(二重)。\\
解线性方程组$(-I-A)X=0$,得一个基础解系:\[
\boldsymbol{\xi}_1=(1, -1, 0, 0)', \boldsymbol{\xi}_2=(0, 0, 1, -1)'.
\]
解线性方程组$(I-a)X=0$,得一个基础解系:\[
\boldsymbol{\xi}_3=(1, 1, 0, 0)', \boldsymbol{\xi}_4=(0, 0, 1, 1)'.
\]
因此,$A$的属于特征值$-1$的特征向量为$k\boldsymbol{\xi}_1+l\boldsymbol{\xi}_2, k, l\in \mathbb{Z}$,$A$的属于特征值$1$的特征向量为$m\boldsymbol{\xi}_3+n\boldsymbol{\xi}_4, m, n\in \mathbb{Z}$。\\
(3) 注意到$\boldsymbol{\xi}_1, \boldsymbol{\xi}_2, \boldsymbol{\xi}_3, \boldsymbol{\xi}_4$两两正交。下面将$\boldsymbol{\xi}_1, \boldsymbol{\xi}_2, \boldsymbol{\xi}_3, \boldsymbol{\xi}_4$单位化。\begin{align*}
\boldsymbol{\eta}_1&=\frac{\boldsymbol{\xi}_1}{|\boldsymbol{\xi}_1|}=\left(\frac{\sqrt{2}}{2}, -\frac{\sqrt{2}}{2}, 0, 0\right)';\\
\boldsymbol{\eta}_2&=\frac{\boldsymbol{\xi}_2}{|\boldsymbol{\xi}_2|}=\left(0, 0, \frac{\sqrt{2}}{2}, -\frac{\sqrt{2}}{2}\right)';\\
\boldsymbol{\eta}_3&=\frac{\boldsymbol{\xi}_3}{|\boldsymbol{\xi}_3|}=\left(\frac{\sqrt{2}}{2}, \frac{\sqrt{2}}{2}, 0, 0\right)';\\
\boldsymbol{\eta}_4&=\frac{\boldsymbol{\xi}_4}{|\boldsymbol{\xi}_4|}=\left(0, 0, \frac{\sqrt{2}}{2}, \frac{\sqrt{2}}{2}\right)'.\\
\end{align*}
令 $P=(\boldsymbol{\eta}_1, \boldsymbol{\eta}_2, \boldsymbol{\eta}_3, \boldsymbol{\eta}_4)$,则 $P$ 是正交矩阵,且\[
P^{-1}AP={\rm diag}\{-1, -1, 1, 1\}.
\]
(4) 标准型为\[
f(x_1, x_2, x_3, x_4)=-y_1^2-y_2^2+y_3^2+y_4^2.
\]
\end{solution}

\item[五、](15分)
设矩阵 $A$ 与 $B$ 没有公共的特征值,$f_A(x)$ 是 $A$ 的特征多项式,证明:\\
(1) 矩阵 $f_A(B)$ 可逆。\\
(2) 矩阵方程 $AX = XB$ 只有零解。
\begin{proof}
(1) 设\[
f_A(x)=(x-\lambda_1)^{l_1} (x-\lambda_2)^{l_2} \cdots (x-\lambda_s)^{l_s}.
\]
则\[
f_A(B)=(B-\lambda_1I)^{l_1} (B-\lambda_2I)^{l_2} \cdots (B-\lambda_sI)^{l_s}.
\]
因为矩阵 $A$ 与 $B$ 没有公共的特征值,所以\[
|B-\lambda_i I|\ne 0, i=1, 2, \dots, s.
\]
因此,\[
|f_A(B)|\ne 0.
\]
故矩阵$f_A(B)$可逆。\\
(2) 设$C$是矩阵方程$AX=XB$的一个解,则\[
f_A(A)C=Cf_A(B).
\]
因此\[
Cf_A(B)=0,
\]
因为矩阵$f_A(B)$可逆,所以$C=0$。因此矩阵方程$AX=XB$只有零解。
\end{proof}

\item[六、](15分)
设 $A$ 是 $n$ 阶正定阵,求证:存在唯一的正定阵 $B$,使得\[
A = B ^ { 2 }.
\]
\begin{solution}
存在性。设$A$是$n$级正定矩阵,则$A$的特征值$\lambda_1, \lambda_2, \dots, \lambda_n$全大于0。由于$A$是实对称矩阵,因此存在$n$级正交矩阵$T$,使得
\begin{align*} 
A & = T ^ { - 1 } \operatorname { diag } \left\{ \lambda _ { 1 } , \lambda _ { 2 } , \dots , \lambda _ { n } \right\} T \\ 
& = T ^ { - 1 } \operatorname { diag } \left\{ \sqrt { \lambda _ { 1 } } , \sqrt { \lambda _ { 2 } } , \dots , \sqrt { \lambda _ { n } } \right\} T \quad T ^ { - 1 } \operatorname { diag } \left\{ \sqrt { \lambda _ { 1 } } , \sqrt { \lambda _ { 2 } } , \dots , \sqrt { \lambda _ { n } } \right\} T \\ 
& = B ^ { 2 },
 \end{align*}
其中\[
B = T ^ { - 1 } \operatorname { diag } \left\{ \sqrt { \lambda _ { 1 } } , \sqrt { \lambda _ { 2 } } , \dots , \sqrt { \lambda _ { n } } \right\} T.
\]
显然$B$是正定矩阵。\\
唯一性。设还有一个$n$级正定矩阵$C$,使得$A=C$。设$C$的全部特征值是$v_1, v_2,\dots, v_n$,则$A$的全部特征值是$v_ { 1 }^2 , v_ { 2 }^2 , \dots , v_ { n }^2$。设$B$的全部特征值是$\mu _ { 1 } , \mu _ { 2 } , \dots , \mu _ { n }$,则$A$的全部特征值是$\mu _ { 1 } ^ { 2 } , \mu _ { 2 } ^ { 2 } , \dots  , \mu _ { n } ^ { 2 }$。于是适当调换$v_1, v_2,\dots, v_n$的下标可以使$\mu _ { i } ^ { 2 } = v _ { i } ^ { 2 } , i = 1,2 , \dots, n$。由于$\mu_i, v_i$都大于0,因此$\mu_i=v_i, i=1, 2, \dots,n$。\\
由于$B$和$C$都是$n$级实对称矩阵,因此存在$n$级正交矩阵$T$,$T_1$,使得\begin{align*}
 B & = T ^ { - 1 } \operatorname { diag } \left\{ \mu _ { 1 } , \mu _ { 2 } , \dots , \mu _ { n } \right\rangle T, \\ C & = T ^ { - 1 } \operatorname { diag } \left\{ v _ { 1 } , v _ { 2 } , \dots , v _ { n } \right\} T _ { 1 }. \end{align*}
由于$B^2=A=C^2$,且$\mu _ { i } = v _ { i } , i = 1,2 , \dots , n$,因此\[
T ^ { - 1 } \operatorname { diag } \left\{ \mu _ { 1 } ^ { 2 } , \mu _ { 2 } ^ { 2 } , \dots , \mu _ { n } ^ { 2 } \right\} T = T _ { 1 } ^ { - 1 } \operatorname { diag } \left\{ \mu _ { 1 } ^ { 2 } , \mu _ { 2 } ^ { 2 } , \dots , \mu _ { n } ^ { 2 } \right\} T _ { 1 }.
\]
两边左乘$T_1$,右乘$T^{-1}$,得\begin{equation}\label{T1Tmu}
T _ { 1 } T ^ { - 1 } \operatorname { diag } \left\{ \mu _ { 1 } ^ { 2 } , \mu _ { 2 } ^ { 2 } , \dots , \mu _ { n } ^ { 2 } \right\} = \operatorname { diag } \left\{ \mu _ { 1 } ^ { 2 } , \mu _ { 2 } ^ { 2 } , \dots , \mu _ { n } ^ { 2 } \right\} T _ { 1 } T ^ { - 1 }.
\end{equation}
记 $T _ { 1 } T ^ { - 1 } = \left( t _ { i j } \right)$。比较 \eqref{T1Tmu} 式两边的 $(i, j)$ 元,得
\begin{equation}\label{tmu2}
t _ { i j } \mu _ { j } ^ { 2 } = \mu _ { i } ^ { 2 } t _ { i j }
\end{equation}
若 $t_{ij}\ne0$,则从 \eqref{tmu2} 式得,$\mu_j^2=\mu_i^2$,由于$\mu_j, \mu_i$都是正数,因此$\mu_j=\mu_i$,从而有\begin{equation}\label{tmu}
t_{ij}\mu_j=\mu_i t_{ij}.
\end{equation}
若 $t_{ij}=0$,则显然 \eqref{tmu} 式也成立。由于 \eqref{tmu} 式对于$1\le i, j\le n$ 都成立,因此
\begin{equation}\label{T1Tmu}
T _ { 1 } T ^ { - 1 } \operatorname { diag } \left\{ \mu _ { 1 } , \mu _ { 2 } , \dots , \mu _ { n } \right\} = \operatorname { diag } \left\{ \mu _ { 1 } , \mu _ { 2 } , \dots , \mu _ { n } \right\} T _ { 1 } T ^ { - 1 }
\end{equation}
从而
\[
T ^ { - 1 } \operatorname {diag} \left\{ \mu _ { 1 } , \mu _ { 2 } , \dots , \mu _ { n } \right\} T = T _ { 1 } ^ { - 1 } \operatorname {diag} \left\{ \mu _ { 1 } , \mu _ { 2 } , \dots , \mu _ { n } \right\} T _ { 1 }.
\]
即 $B=C$。这就证明了唯一性。
\end{solution}

\item[七、](15分)
设$\boldsymbol{\varepsilon} _ { 1 } , \boldsymbol{\varepsilon} _ { 2 } , \dots , \boldsymbol{\varepsilon} _ { n }$ 是 $n$ 维线性空间 $V$ 的一组基,$A$ 是 $V$ 上的线性变换。证明:$A$ 是可逆的,当且仅当
$\mathscr { A } \left( \boldsymbol{\varepsilon} _ { 1 } \right) , \mathscr { A } \left( \boldsymbol{\varepsilon} _ { 2 } \right) , \dots , \mathscr { A } \left( \boldsymbol{\varepsilon} _ { n } \right)$也是 $V$ 的基。
\begin{proof}
因为$\dim V=n$,所以
\begin{align*}
&\text{$A$ 是可逆的,当且仅当$\mathscr { A } \left( \boldsymbol{\varepsilon} _ { 1 } \right) , \mathscr { A } \left( \boldsymbol{\varepsilon} _ { 2 } \right) , \dots , \mathscr { A } \left( \boldsymbol{\varepsilon} _ { n } \right)$也是 $V$ 的基。}\\
&\Longleftrightarrow \text{$A$ 是可逆的,当且仅当$\mathscr { A } \left( \boldsymbol{\varepsilon} _ { 1 } \right) , \mathscr { A } \left( \boldsymbol{\varepsilon} _ { 2 } \right) , \dots , \mathscr { A } \left( \boldsymbol{\varepsilon} _ { n } \right)$线性无关。}\\
&\begin{aligned}\Longleftrightarrow \text{$A$ 是可逆的,}&\text{当且仅当对任意的一组常数$k_1, k_2, \dots, k_n\in \mathbb{K}$,}\\
&\text{若$k_1\mathscr { A } \left( \boldsymbol{\varepsilon} _ { 1 } \right) +k_2 \mathscr { A } \left( \boldsymbol{\varepsilon} _ { 2 } \right) + \cdots +k_n \mathscr { A } \left( \boldsymbol{\varepsilon} _ { n } \right)=0$,则$k_1=k_2=\cdots=k_n=0$。}
\end{aligned}\\
&\begin{aligned}\Longleftrightarrow \text{$A$ 是可逆的,}
&\text{当且仅当对任意的一组常数$k_1, k_2, \dots, k_n\in \mathbb{K}$,}\\
&\text{若$\mathscr { A } \left(k_1 \boldsymbol{\varepsilon} _ { 1 }  +k_2  \boldsymbol{\varepsilon} _ { 2 }  + \cdots +k_n \boldsymbol{\varepsilon} _ { n } \right)=0$,则$k_1=k_2=\cdots=k_n=0$。}
\end{aligned}\\
&\Longleftrightarrow \text{$A$ 是可逆的,当且仅当对任意的向量$\boldsymbol{\alpha}\in V$,若$\mathscr { A } \left(\boldsymbol{\alpha} \right)=0$,则$\boldsymbol{\alpha}=0$。}\\
&\Longleftrightarrow \text{$A$ 是可逆的,当且仅当$\mathscr{A}$是单射。}
\end{align*}
因为有限维空间之间的单射必是满射,所以\begin{align*}
&\text{$A$ 是可逆的,当且仅当$\mathscr{A}$是单射。}\\
&\Longleftrightarrow \text{$A$ 是可逆的,当且仅当$\mathscr{A}$是单射且$\mathscr{A}$是满射。}\\
&\Longleftrightarrow\text{$A$ 是可逆的,当且仅当$
\mathscr{A}$是可逆的线性变换。}
\end{align*}
这是显然的。
\end{proof}
\item[八、](15 分)
设 $\mathscr{A}$ 与 $\mathscr{B}$ 为 $n$ 维欧式空间 $V$ 
上的两个线性变换。若对任意的 $\boldsymbol{\alpha} \in V$,有\[
( \mathscr { A } \boldsymbol{\alpha} , \mathscr { A } \boldsymbol{\alpha} ) = ( 
\mathscr { B } \boldsymbol{\alpha} , \mathscr { B } \boldsymbol{\alpha} )
\]
则 $\mathscr{A} V$ 与 $\mathscr{B}V$ 作为欧式空间是同构
的。
\begin{proof}
设欧式空间$V$是数域$\mathbb{K}$上的欧式空间,$
\boldsymbol{\varepsilon}_1, \boldsymbol{\varepsilon}_2, \dots, \boldsymbol{\varepsilon}_n$是$V
$的一个基,由第七题知,$\mathscr{A}\boldsymbol{\varepsilon}_1, 
\mathscr{A}\boldsymbol{\varepsilon}_2, \dots, \mathscr{A}
\boldsymbol{\varepsilon}_n$是$\mathscr{A}V$的一个基,$\mathscr{B}
\boldsymbol{\varepsilon}_1, \mathscr{B}\boldsymbol{\varepsilon}_2, \dots, 
\mathscr{B}\boldsymbol{\varepsilon}_n$是$\mathscr{B}V$的一个基。令
\[
\sigma(t_1\mathscr{A}\boldsymbol{\varepsilon}_1+t_2\mathscr{A}
\boldsymbol{\varepsilon}_2+\cdots+t_n\mathscr{A}
\boldsymbol{\varepsilon}_n)=t_1\mathscr{B}
\boldsymbol{\varepsilon}_1+t_2\mathscr{B}\boldsymbol{\varepsilon}_2+\cdots+t_n
\mathscr{B}\boldsymbol{\varepsilon}_n\quad(t_i\in \mathbb{K}).
\]
我们来证明$\sigma$是$\mathscr{A}V$到$\mathscr{B}V$的
同构映射。\\
(i) 显然$\mathscr{A}V$中任一向量$t_1\mathscr{A}
\boldsymbol{\varepsilon}_1+t_2\mathscr{A}\boldsymbol{\varepsilon}_2+\cdots+t_n
\mathscr{A}\boldsymbol{\varepsilon}_n$被$\sigma$映为$\mathscr{B}V
$中的向量$t_1\mathscr{B}\boldsymbol{\varepsilon}_1+t_2\mathscr{B}
\boldsymbol{\varepsilon}_2+\cdots+t_n\mathscr{B}\boldsymbol{\varepsilon}_n$;而且
$\mathscr{B}V$中的任何向量$t_1\mathscr{B}
\boldsymbol{\varepsilon}_1+t_2\mathscr{B}\boldsymbol{\varepsilon}_2+\cdots+t_n
\mathscr{B}\boldsymbol{\varepsilon}_n $都是$\mathscr{A}V$中的向量
$t_1\mathscr{A}\boldsymbol{\varepsilon}_1+t_2\mathscr{A}
\boldsymbol{\varepsilon}_2+\cdots+t_n\mathscr{A}\boldsymbol{\varepsilon}_n$在$
\sigma$下的象,因此$\mathscr{A}V$到$\mathscr{B}V$的映
射$\sigma$是映上的。\\
(ii) 设$\boldsymbol{\alpha}$,$\boldsymbol{\beta} \in \mathscr{A}V$,则
\begin{align*}
&{ \boldsymbol{\alpha} } = \sum _ { k = 1 } ^ { n } c _ { k }  {\mathscr{A}\boldsymbol{\varepsilon} } _ { k } , { \boldsymbol{\beta} } = \sum _ { k = 1 } ^ { n } d _ { k }  { \mathscr{A}\boldsymbol{\varepsilon} } _ { k } \\ 
&{ \sigma ( { \boldsymbol{\alpha} } ) = \sum _ { k = 1 } ^ {n } c _ { 
k }  { \mathscr{B}\boldsymbol{\varepsilon} } _ { k } , \sigma (  
{ \boldsymbol{\beta} } ) = \sum _ { k = 1 } ^ { n } d _ { k }  { 
\mathscr{B}\boldsymbol{\varepsilon} } _ { k } }. 
\end{align*}
其中$c _ { k } , d _ { k } \in \mathbb{K}$。若$\sigma (  { \boldsymbol{\alpha} } ) = \sigma (  { \boldsymbol{\beta} } )$,则由上式得到
\[
 { \delta } = \sigma (  { \boldsymbol{\alpha} } ) - \sigma (  { \boldsymbol{\beta} } ) = \sum _ { k = 1 } ^ { n } \left( c _ { k } - d _ { k } \right)  { \mathscr{B}\boldsymbol{\varepsilon} } _ { k } =  { 0 }.
 \]
于是\[
( \delta , \delta ) = \sum _ { 1 \leq i , j \leqslant n } \left( c _ { i } - d _ { i } \right) \left( c _ { j } - d _ { j } \right) \left(  { \mathscr{B}\boldsymbol{\varepsilon} } _ { i } ,  { \mathscr{B}\boldsymbol{\varepsilon} } _ { j } \right) = 0.
\]
记
\[
 { \theta } =  { \boldsymbol{\alpha} } -  { \boldsymbol{\beta} } = \sum _ { k = 1 } ^ { n } c _ { k }  { \mathscr{A}\boldsymbol{\varepsilon} } _ { k } - \sum _ { k = 1 } ^ { n } d _ { k }  { \mathscr{A}\boldsymbol{\varepsilon} } _ { k } = \sum _ { k = 1 } ^ { n } \left( c _ { k } - d _ { k } \right)  { \mathscr{A}\boldsymbol{\varepsilon} } _ { k }.
\]
依题设条件及内积性质,我们有
\begin{align*}
 (  { \theta } ,  { \theta } ) 
& = \left( \sum _ { k = 1 } ^ { n } \left( c _ { k } - d _ { k } \right)  { \mathscr{A}\boldsymbol{\varepsilon} } _ { k } , \sum _ { k 
= 1 } ^ { n } \left( c _ { k } - d _ { k } \right)  
{\mathscr{A}\boldsymbol{\varepsilon} } _ { k } \right) \\ 
& = \sum _ { 1 \leq i , j \leqslant n } \left( c _ { i } - d _ { i } \right) \left( c _ { j } - d _ { j } \right) \left(  { 
\mathscr{A}\boldsymbol{\varepsilon} } _ { i } ,  { \mathscr{A}\boldsymbol{\varepsilon} } _ { j } \right) \\ 
& = \sum _ { 1 \leq i , j \leqslant n } \left( c _ { i } - d _ { i } \right) \left( c _ { j } - d _ { j } \right) \left(  { 
\mathscr{B}\boldsymbol{\varepsilon} } _ { i } ,  { \mathscr{B}
\boldsymbol{\varepsilon} } _ { j } \right) \\ 
& = ( \delta , \delta ) \\ 
& = 0 
\end{align*}
因此$\theta = 0$,即$\boldsymbol{\alpha}=\boldsymbol{\beta}$,故$\sigma$是单射,从而$\sigma$是$\mathscr{A}V$到$\mathscr{B}V$的一一映射。\\
(iii) 此外,依$\sigma$的定义知
\begin{align*} 
\sigma (  { \boldsymbol{\alpha} } +  { \boldsymbol{\beta} } ) 
 & = \sigma \left( \sum _ { k = 1 } ^ { n } \left( c _ { k } + d _ { k } \right)  { \mathscr{A}\boldsymbol{\varepsilon} } _ { k } \right) \\
 & = \sum _ { k = 1 } ^ { n } \left( c _ { k } + d _ { k } \right)  { \mathscr{B}\boldsymbol{\varepsilon} } _ { k } \\ 
 & = \sum _ { k = 1 } ^ { n } c _ { k }  { \mathscr{B}\boldsymbol{\varepsilon} } _ { k } + \sum _ { k = 1 } ^ { n } d _ { k }  { \mathscr{B}\boldsymbol{\varepsilon} } _ { k } \\
 & = \sigma (  { \boldsymbol{\alpha} } ) + \sigma (  { \boldsymbol{\beta} } ) .
\end{align*}
类似地可验证\[
\sigma ( a  { \boldsymbol{\alpha} } ) = a \sigma (  { \boldsymbol{\alpha} } )\text{,其中$a\in\mathbb{K}$},
\]
以及\[
( \sigma (  { \boldsymbol{\alpha} } ) , \sigma (  { \boldsymbol{\beta} } ) ) = (  { \boldsymbol{\alpha} } ,  { \boldsymbol{\beta} } ).
\]
由(1)、(2)、(3)得,$\sigma$是$\mathscr{A}V$到$\mathscr{B}V$的同构映射。于是$\mathscr{A}V$与 $\mathscr{B}V$ 作为欧式空间是同构的。
\end{proof}

\end{enumerate}
\endinput