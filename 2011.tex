\section{2011}
\begin{enumerate}[1~]
\renewcommand{\labelenumi}{\textbf{\theenumi. }}
\renewcommand{\Im}{\text{Im }}

\item[一、]
填空题 (每小题 4 分,共 20 分)
\begin{enumerate}[1.~]
\item
多项式$f(x) = 2x^3 +4x^2 +6x+1$在有理数域上是(\ \ \ \ )。(可约或不可约)
\begin{solution}
因为$f(x)$是一个次数大于0的本原多项式,且存在素数2,使得\[
\left\{ \begin{array}{l}
	2\mid 2,\ 2\mid 4,\ 2\mid 6,\\
	2\nmid 1,\\
	2^2\nmid 2.\\
\end{array} \right.  \]
所以,由 Eisenstein 判别法知,$f(x)$ 在有理数域上不可约。
\end{solution}

\item
设$A$和$B$均为$n$阶方阵,$A^*$与$B^*$分别为他们的伴随矩阵,$|A|=2$,$|B| = -3$,则$|A^{-1}B^* -A^*B^{-1}|$ = (\ \ \ \ )。
\begin{solution}
\begin{align*}
\left| A^{-1} \left|B\right| B^{-1} \right| &= \left| \left( \left|B \right| - \left|A \right| \right) A^{-1} B^{-1}\right|\\  &=\left( \left|B \right| -\left|A \right| \right)^n \left|A \right|^{-1}\left|B \right|^{-1} \\ &=\frac{-\left( -5 \right)^n}{6}.
\end{align*}
\end{solution}

\item
设 $3$ 阶方阵 $A$ 按下列分块为 $A=(A_1, A_2, A_3)$,且 $|A|=5$,又设 $B=(A_1+2A_2, 3A_1+4 A_3)$,则$|B|=$(\quad)。
\begin{solution}
\begin{align*}
|B|&=|(A_1+2A_2, 3A_1+4 A_3)|=|(A_1, 3A_1+4 A_3)|+|(2A_2, 3A_1+4 A_3)|\\
&=0+24 |(A_2, A_1, A_3)|=-24 |(A_2, A_1, A_3)|\\
&=-24\times 5=-120.
\end{align*}
\end{solution}

\item
已知向量组$\boldsymbol{\alpha}_1 = (1, 2, -1, 1), \boldsymbol{\alpha}_2 =(2, 0, t, 0), \boldsymbol{\alpha}_3 = (0, -4, 5, -2)$的秩为2,则$t=$ (\ \ \ \ )。
\begin{solution}
对
$$\left( \begin{matrix}
\boldsymbol{\alpha}_1\\
\boldsymbol{\alpha}_2\\
\boldsymbol{\alpha}_3  
\end{matrix} \right)
$$
作初等行变换:\[
\left( \begin{matrix}
\boldsymbol{\alpha}_1\\
\boldsymbol{\alpha}_2\\
\boldsymbol{\alpha}_3  
\end{matrix} \right) =  \left( \begin{matrix}
	1&		2&		-1&		1\\
	2&		0&		t&		0\\
	0&		-4&		5&		-2
\end{matrix} \right) \\
\rightarrow  \left( \begin{matrix}
	1&		2&		-1&		1\\
	0&		-4&		t+2&		-2\\
	0&		0&		3-t&		0
\end{matrix} \right) \]
因此,若向量组 $\boldsymbol{\alpha}_1, \boldsymbol{\alpha}_2, \boldsymbol{\alpha}_3$ 的秩为 2,则 $t=3$。
\end{solution}

\item
若实对称矩阵$A$与矩阵$B = \left( \begin{smallmatrix}
1&  0&  0\\
0&  0&  2\\
0&  2&  0
\end{smallmatrix} \right)$合同,则二次型$X^TAX$的规范型为(\ \ \ \ )。

\begin{solution}
先求实对称矩阵$B$的特征值。因为\[
\left|\lambda I -B\right| = \left| \begin{matrix}
	\lambda-1&		0&		0\\
	0&		\lambda&			-2\\
	0&		-2&		\lambda
\end{matrix} \right| = (\lambda+2) (\lambda-1) (\lambda-2).\]
所以,矩阵 $B$ 的特征值为 $-2$ ,$1$,$2$。\\
所以,二次型 $X^TBX$ 的秩为 $3$ ,正惯性指数为 $2$ 。因为实对称矩阵 $A$ 与实对称矩阵 $B$ 合同,所以二次型 $XA^TX$ 的秩为 $3$,正惯性指数为 $2$。\\
所以,二次型$X^TAX$的规范型是$y_1^2+y_2^2-y_3^2$。
\end{solution}

\item
$\mathbb{R}^3$中的向量$\boldsymbol{\alpha} = (x_1, x_2, x_3) $在基$\boldsymbol{\alpha}_1 = (1, 1, 1)$,$\boldsymbol{\alpha}_2 = (0, 1, 1)$,$\boldsymbol{\alpha}_3 = (0, 0, 1)$下的坐标是(\ \ \ \ )。
\begin{solution}
设$\boldsymbol{\alpha} = k_1 \boldsymbol{\alpha}_1+k_2 \boldsymbol{\alpha}_2+k_3 \boldsymbol{\alpha}_3$,则向量$(k_1, k_2, k_3)$满足\begin{equation}\label{1}
\left( \begin{matrix}
	1&		0&		0\\
	1&		1&		0\\
	1&		1&		1\\
\end{matrix} \right) \left( \begin{array}{c}
	k_1\\
	k_2\\
	k_3\\
\end{array} \right) =\left( \begin{array}{c}
	x_1\\
	x_2\\
	x_3\\
\end{array} \right).\end{equation}
解线性方程组 \eqref{1} 得,\[
\left\{ \begin{array}{l}
	k_1=x_1,\\
	k_2=x_2-x_1,\\
	k_3=x_3-x_2.\\
\end{array} \right. \]
因此,$\boldsymbol{\alpha}$在$\boldsymbol{\alpha}_1, \boldsymbol{\alpha}_2, \boldsymbol{\alpha}_3$下的坐标为$(x_1, x_2-x_1, x_3-x_2)$。
\end{solution}

\item
设3阶方阵$A$的三个特征值为 $1, 2, -2$,矩阵$B$与$A$相似,则$B$的伴随矩阵$B^*$的三个特征值为(\ \ \ \ )。
\begin{solution}
先研究下述命题:\\
设$A$是复数域上的$n$级矩阵,$\lambda_1, \lambda_2, \dots, \lambda_n$是$A$的全部特征值,求$A^*$的全部特征值。
\begin{subsolution}
情形1。由于$A A ^ { * } = | A | I$,因此$ A ^ { * } = | A | A^{-1}$,所以$A^{*}$的全部特征值是\[
\lambda _ { 2 } \lambda _ { 3 } \cdots \lambda _ { n } ,  \lambda _ { 1 } \lambda _ { 3 } \cdots \lambda _ { n } ,  \dots ,  \lambda _ { 1 } \lambda _ { 2 } \cdots \lambda _ { n - 1 }. \]
情形2。$A$不可逆。此时0是$A$的一个特征值,不妨设$
\lambda_n = 0$。由于当${\rm rank}(A) <n-1$时,
$A^*=0$。此时0是$A^*$的$n$重特征值。当${\rm rank}(A) 
=n-1$时,${\rm rank}(A^*)=1$,从而$(0I-A^*)X = 0$的
解空间的维数等于$n-1$,于是0是$A^*$的至少$n-1$重特征值。
设$\mu$也是$A^*$的一个特征值,则$\left| \lambda I - A ^ 
{ * } \right| = \lambda ^ { n - 1 } ( \lambda - \mu )$。从
而$\mu = {\rm tr}(A^*) = A_{11} +A_{22}+ \cdots 
+A_{nn}$。由于$\lambda I-A|$的一次项系数等于$
(-1)^{n-1}$乘以$A$的所有$n-1$阶主子式的和,从而等于$
(-1)^{n-1} \sum_{i=1}^n A_n$。又$A$的一次项系数等于$
(-1)^{n-1} \lambda_1\lambda_2 \cdots \lambda_{n-1}
$。于是$A^*$的全部特征值是$\lambda_1\lambda_2\cdots
\lambda_{n-1}$, $0$(至少$n-1$重)。
\end{subsolution}
回到原题。$B$的特征值为 $1, 2, -2$,所以$B^*$的特征值为 $-4, -2, 2$。
\end{solution}

\begin{remark}
一个更一般的命题是:\\
设 $A$是数域$\mathbb{K}$上$n$级矩阵,$\lambda_1, \lambda_2, \dots,\lambda_n$是$A$的特征多项式在复数域上的全部根,则$A$的伴随矩阵$A^*$的特征多项式在复数域中的全部根为\[
\lambda _ { 2 } \lambda _ { 3 } \cdots \lambda _ { n } ,  \lambda _ { 1 } \lambda _ { 3 } \cdots \lambda _ { n } ,  \dots ,  \lambda _ { 1 } \lambda _ { 2 } \cdots \lambda _ { n - 1 }. \]
先证明下面三个引理:
\begin{lemma}\label{fujuzhenxiangsiyuduijiaozhen}
任一$n$级复矩阵一定相似于一个上三角矩阵。
\end{lemma}
\begin{subproof}
对复矩阵的级数$n$作数学归纳法。$n=1$时,显然命题为真,假设$n-1$级复矩阵一定相似于一个上三角矩阵。现在来看$n$级复矩阵$A$。设$\lambda_1$是$n$级复矩阵$A$的一个特征值,$\boldsymbol{\alpha}_1$是属于$\lambda_1$的一个特征向量。把$\boldsymbol{\alpha}_1$扩充成$\mathbb{C}^n$的一个基:$\boldsymbol{\alpha}_1, \boldsymbol{\alpha}_2, \dots, \boldsymbol{\alpha}_n$。令$P_1 = \left(\boldsymbol{\alpha}_1, \boldsymbol{\alpha}_2, \dots, \boldsymbol{\alpha}_n \right)$,则$P_1$是$n$级可逆矩阵,且\[
P _ { 1 } ^ { - 1 } A P _ { 1 } = P _ { 1 } ^ { - 1 } \left( A \boldsymbol { a } _ { 1 } , A \boldsymbol { \boldsymbol{\alpha} } _ { 2 } , \dots , A \boldsymbol { a } _ { n } \right) = \left( P _ { 1 } ^ { - 1 } \lambda _ { 1 } \boldsymbol { \boldsymbol{\alpha} } _ { 1 } , P _ { 1 } ^ { - 1 } A \boldsymbol { \boldsymbol{\alpha} } _ { 2 } , \dots , P _ { 1 } ^ { - 1 } A \boldsymbol { \boldsymbol{\alpha} } _ { n } \right). \]
由于$P_1^{-1}P_1 = I$,因此$P_1^{-1} \boldsymbol{\alpha}_1 =\epsilon_1$。从而\[
P _ { 1 } ^ { - 1 } A P _ { 1 } = \left( \begin{array} { l l } { \lambda _ { 1 } } & { \boldsymbol{\alpha} } \\ { 0 } & { B } \end{array} \right). \]
对$n-1$级复矩阵$B$用归纳假设,由$n-1$级可逆矩阵$P_2$,使得$P_2^{-1}BP_2$为上三角矩阵。令\[
P = P _ { 1 } \left( \begin{array} { l l } { 1 } & { 0 } \\ { 0 } & { P _ { 2 } } \end{array} \right) \]
则 $P$ 是 $n$ 级可逆矩阵,且\[
P ^ { - 1 } A P = \left( \begin{array} { c c } { 1 } & { 0 } \\ { 0 } & { P _ { 2 } } \end{array} \right) ^ { - 1 } \left( \begin{array} { l l } { \lambda _ { 1 } } & { \boldsymbol{\alpha} } \\ { 0 } & { B } \end{array} \right) \left( \begin{array} { l l } { 1 } & { 0 } \\ { 0 } & { P _ { 2 } } \end{array} \right)  = \left( \begin{array} { c c } { \lambda _ { 1 } } & { \boldsymbol { \boldsymbol{\alpha} } P _ { 2 } } \\ { \mathbf { 0 } } & { P _ { 2 } ^ { - 1 } B P _ { 2 } } \end{array} \right). \]
因此 $P^{-1}AP$ 是上三角矩阵。

据数学归纳法原理,对一切正整数 $n$,此命题为真。
\end{subproof}

\begin{lemma}\label{bansuixiangsi}
设$A, B$都是数域$\mathbb{K}$上的$n$级矩阵($n \ge 2$)。$A^*$,$B^*$分别是$A, B$的伴随矩阵,如果$ A \sim B$,那么$ A^* \sim B^*$。
\end{lemma}

\begin{subproof}
如果$ A \sim B$,则存在$\mathbb{K}$上$n$级可逆矩阵$P$,使得$P^{-1}AP= B$,所以$P^*A^*\left(P^{-1}\right)^*= B^*$,$\left( P ^ { - 1 } \right) ^ { * } = \left( P ^ { * } \right) ^ { - 1 }$,从而\[
P ^ { * } A ^ { * } \left( P ^ { * } \right) ^ { - 1 } = B ^ { * }. \]
因此$A ^ { * } \sim B ^ { * }$。
\end{subproof}

\begin{lemma}\label{bansuiduijiaohua}
设$A$是数域$\mathbb{K}$上的$n$级矩阵,如果$A$可对角化,那么$A$的伴随矩阵$A^*$也可对角化。
\end{lemma}

\begin{subproof}
若$A$可对角化,则$A \sim D$,其中$D = \operatorname { diag } \left\{ \lambda _ { 1 } , \lambda _ { z } , \dots , \lambda _ { n } \right\}$。据 \eqref{bansuixiangsi} 得,$A ^ { * } \sim D ^ { * }$。直接计算可得\[
D ^ { * } = \left( \begin{array} { c c c c } { \lambda _ { 2 } \lambda _ { 3 } \cdots \lambda _ { n } } & { 0 } & { \cdots } & { 0 } \\ { 0 } & { \lambda _ { 1 } \lambda _ { 3 } \cdots \lambda _ { n } } & { \cdots } & { 0 } \\ { \vdots } & { \vdots } & { } & { \vdots } \\ { 0 } & { 0 } & { \cdots } & { \lambda _ { 1 } \lambda _ { 2 } \cdots \lambda _ { n - 1 } } \end{array} \right). \]
因此$A^*$可对角化。
\end{subproof}

接下来回到以上命题:

\begin{proof}
把$A$看成复矩阵,据 \eqref{fujuzhenxiangsiyuduijiaozhen} 得,$A \sim B$,其中$B$是上三角矩阵,$B$的主对角元为$\lambda_1, \lambda_2, \dots, \lambda_n$。据 \eqref{bansuiduijiaohua} 得,$A^* \sim B^*$。直接计算可得\[
B ^ { * } = \left( \begin{array} { c c c c } { \lambda _ { 2 } \lambda _ { 3 } \cdots \lambda _ { n } } & { c _ { 12 } } & { \cdots } & { c _ { 1 n } } \\ { 0 } & { \lambda _ { 1 } \lambda _ { 3 } \cdots \lambda _ { n } } & { \cdots } & { c _ { 2 n } } \\ { \vdots } & { \vdots } & { } & { \vdots } \\ { 0 } & { 0 } & { \cdots } & { \lambda _ { 1 } \lambda _ { 2 } \cdots \lambda _ { n - 1 } } \end{array} \right). \]
由于$|\lambda I -A^*| = |\lambda I -B^*|$,因此$A^*$的特征多项式在复数域中的全部根是:\[
\lambda _ { 2 } \lambda _ { 3 } \cdots \lambda _ { n } ,  \lambda _ { 1 } \lambda _ { 3 } \cdots \lambda _ { n } ,  \dots ,  \lambda _ { 1 } \lambda _ { 2 } \cdots \lambda _ { n - 1 }. \]
\end{proof}
\end{remark}

\item
设矩阵
$$
A = \left( \begin{matrix}
1&  1&  1\\
1&  1&  1\\
1&  1&  1\\
\end{matrix} \right),
$$
则 $A$ 的最小多项式为(\ \ \ \ )。

\begin{solution}
$A$的特征多项式为\[
f(\lambda)=|\lambda I-A|=\left| \begin{matrix}
	\lambda -1&		-1&		-1\\
	-1&		\lambda -1&		-1\\
	-1&		-1&		\lambda -1\\
\end{matrix} \right|=\lambda ^2\left( \lambda -3 \right).
\]
因为$A(A-3I)=0$,所以$A$的最小多项式为$m(\lambda)=\lambda(\lambda-3)$。
\end{solution}

\item
$\mathbb{R}^3$中的子空间$V_1 =L(\boldsymbol{\alpha})$,其中$\boldsymbol{\alpha} =\left(1,1,1\right)$,则$V_1^{\bot} = $(\ \ \ \ )。
\begin{solution}
解线性方程组$\boldsymbol{\alpha} X=0$,得一个基础解系:\[
\boldsymbol{\eta}_1=(1, -1, 0)', \boldsymbol{\eta}_2=(1, 0, -1)'.
\]
因此$V_1^{\bot}=\{\boldsymbol{\eta}_1, \boldsymbol{\eta}_2\}$。
\end{solution}

\item
特征值为 1,1,1,1 的一切 $4\times4$ 复数矩阵在复数域内按相似可分为(\quad)类。

\begin{solution}
$5$。
\end{solution}
\end{enumerate}

\item[二、]
已知实矩阵
$$
A = \left( \begin{matrix}
2&  2\\
2&  x\\ 
\end{matrix} \right),
B = \left( \begin{matrix}
4&  y\\
3&  1\\ 
\end{matrix} \right).
$$
问:

(1) $x, y$ 为何值时,$A$ 合同于 $B$?

(2) $x, y$ 为何值时,$A$ 相似于 $B$?

\begin{solution}
(1) 若 $A$ 与 $B$ 合同,则 $A$ 与 $B$ 有相同的秩与正惯性指数。对 $A$ 和 $B$ 作初等行变换得\begin{align*}
A&=\left( \begin{matrix}
	2&		2\\
	2&		x\\
\end{matrix} \right) \rightarrow \left( \begin{matrix}
	2&		0\\
	0&		x-2\\
\end{matrix} \right) \\
B&=\left( \begin{matrix}
	4&		y\\
	3&		1\\
\end{matrix} \right) \rightarrow \left( \begin{matrix}
	4&		0\\
	0&		1-\frac{3}{4}y\\
\end{matrix} \right) .
\end{align*}
因此,应当有\[
\left\{ \begin{array}{l}
	x-2=0\\
	1-\frac{3}{4}y=0\\
\end{array}\text{或}\left\{ \begin{array}{l}
	x-2>0\\
	1-\frac{3}{4}y>0\\
\end{array}\text{或}\left\{ \begin{array}{l}
	x-2<0\\
	1-\frac{3}{4}y<0\\
\end{array} \right. \right. \right. 
\]
即\[
x=2,y=\frac{4}{3}\text{或}x>2,y<\frac{4}{3}\text{或}x<2,y>\frac{4}{3}.
\]
(2) $A$ 的特征多项式为\[
|\lambda I-A|=\lambda ^2-\left( 2+x \right) \lambda +2x-4.
\]
$B$ 的特征多项式为\[
|\lambda I-B|=\lambda ^2 - 5 \lambda + 4 - 3 y.
\]
若 $A$ 与 $B$ 相似,则 $|\lambda I-A|=|\lambda I-B|$,因此\[
\left\{ \begin{array}{l}
	2+x=5\\
	2x+4=4-3y\\
\end{array} \right. 
\]
解得$x=3, y=\frac{3}{2}$。
\end{solution}

\item[三、]
设
$$
A = \left( \begin{matrix}
1&  2&  1\\
{\alpha}&  1&  {\beta}\\
1&  {\beta}&  1\\
\end{matrix} \right), B = \left( \begin{matrix}
0&  0&  0\\
0&  1&  0\\
0&  0&  2\\
\end{matrix} \right),
$$
且 $A$ 与 $B$ 相似。

(1)求${\alpha}$和${\beta}$的值;

(2)求可逆阵 $P$ 使得 $P^{-1}AP=B$。

\begin{solution}
(1) $A$ 的特征多项式为:\[
|\lambda I-A|=(\lambda-1)^3-(1+2{\alpha}+{\beta}^2)(\lambda-1)-(2+{\alpha}){\beta}.
\]
$B$的特征多项式为:\[
|\lambda I-B|=\lambda(\lambda-1)(\lambda-2).
\]
解\[
\left\{ \begin{array}{l}
	-1-\left( 1+2{\alpha} +{\beta} ^2 \right) \left( -1 \right) -\left( 2+{\alpha} \right) {\beta} =0,\\
	-\left( 2+{\alpha} \right) {\beta} =0,\\
	1-\left( 1+2{\alpha} +{\beta} ^2 \right) -\left( 2+{\alpha} \right) {\beta} =0.\\
\end{array} \right. 
\]
得\[
{\alpha}=0, {\beta}=0 \text{\ 或\ } {\alpha}=-2, {\beta}=\pm 2.
\]
(2) 
\end{solution}

\item[四、]
在 $\mathbb { R } ^ { 3 }$ 中定义线性变换 $\mathscr { A }$ 为\[
\mathscr { A } \left( x _ { 1 } , x _ { 2 } , x _ { 3 } \right) = \left( 2 x _ { 1 } - x _ { 2 } , x _ { 2 } + x _ { 3 } , x _ { 1 } \right).
\]
(1) 求 $\mathscr { A }$ 在基 $\boldsymbol{\varepsilon} _ { 1 } = ( 1,0,0 ) , \boldsymbol{\varepsilon} _ { 2 } = ( 0,1,0 ) , \boldsymbol{\varepsilon} _ { 3 } = ( 0,0,1 )$ 下的矩阵。\\
(2) 设 $\boldsymbol{\alpha} = ( 1,0 , - 2 )$,求 $\mathscr { A } \boldsymbol{\alpha}$ 在基 $\boldsymbol{\alpha} _ { 1 } = ( 2,0,1 ) , \boldsymbol{\alpha} _ { 2 } = ( 0 , - 1,1 ) , \boldsymbol{\alpha} _ { 3 } = ( - 1,0,2 )$ 下的坐标。\\
(3) $\mathscr { A }$ 是否可逆?若可逆,求 $\mathscr{A}^{-1}$,若不可逆,说明原因。

\begin{solution}
(1) 由已知,\[
\mathscr{A} \left( \begin{array}{c}
	\boldsymbol{\varepsilon} _1\\
	\boldsymbol{\varepsilon} _2\\
	\boldsymbol{\varepsilon} _3\\
\end{array} \right) = \left( \begin{array}{c}
	\mathscr{A}\boldsymbol{\varepsilon} _1\\
	\mathscr{A}\boldsymbol{\varepsilon} _2\\
	\mathscr{A}\boldsymbol{\varepsilon} _3\\
\end{array} \right) =\left( \begin{array}{c}
	\boldsymbol{\varepsilon} _1\\
	\boldsymbol{\varepsilon} _2\\
	\boldsymbol{\varepsilon} _3\\
\end{array} \right) \left( \begin{matrix}
	2&		0&		1\\
	-1&		1&		0\\
	0&		1&		0\\
\end{matrix} \right) .
\]
因此, $\mathscr { A }$ 在基 $\boldsymbol{\varepsilon} _ { 1 } = ( 1,0,0 ) , \boldsymbol{\varepsilon} _ { 2 } = ( 0,1,0 ) , \boldsymbol{\varepsilon} _ { 3 } = ( 0,0,1 )$ 下的矩阵为 
$$
\left( \begin{matrix}
	2&		0&		1\\
	-1&		1&		0\\
	0&		1&		0\\
\end{matrix} \right).
$$

(2) 由已知\[
\mathscr{A} \left( \begin{array}{c}
	\boldsymbol{\alpha} _1\\
	\boldsymbol{\alpha} _2\\
	\boldsymbol{\alpha} _3\\
\end{array} \right) = \left( \begin{array}{c}
	\mathscr{A}\boldsymbol{\alpha} _1\\
	\mathscr{A}\boldsymbol{\alpha} _2\\
	\mathscr{A}\boldsymbol{\alpha} _3\\
\end{array} \right) =\left( \begin{array}{c}
	\boldsymbol{\alpha} _1\\
	\boldsymbol{\alpha} _2\\
	\boldsymbol{\alpha} _3\\
\end{array} \right) \left( \begin{matrix}
	2&		0&		1\\
	-1&		1&		0\\
	0&		1&		0\\
\end{matrix} \right) .
\]
解线性方程组 
$$
X \left( \begin{array}{c}
	\boldsymbol{\alpha} _1\\
	\boldsymbol{\alpha} _2\\
	\boldsymbol{\alpha} _3\\
\end{array} \right) = \boldsymbol{\alpha},
$$
得一个基础解系
$$
\boldsymbol{\xi}=(0,0,-1).
$$
因此
$$
\mathscr{A}\boldsymbol{\alpha}=\mathscr{A} \boldsymbol{\xi} \left( \begin{array}{c}
	\boldsymbol{\alpha} _1\\
	\boldsymbol{\alpha} _2\\
	\boldsymbol{\alpha} _3\\
\end{array} \right) = ( 0,-1,0 ) \left( \begin{array}{c}
	\boldsymbol{\alpha} _1\\
	\boldsymbol{\alpha} _2\\
	\boldsymbol{\alpha} _3\\
\end{array} \right) = ( 0,-1,0 ) \left( \begin{array}{c}
	\boldsymbol{\alpha} _1\\
	\boldsymbol{\alpha} _2\\
	\boldsymbol{\alpha} _3\\
\end{array} \right).
$$
故坐标为 $(0,-1,0)$。

(3) 因为 $\mathscr{A}$ 在基 $\boldsymbol{\varepsilon}_1, \boldsymbol{\varepsilon}_2, \boldsymbol{\varepsilon}_3$ 下的矩阵为
$$
A=\left( \begin{matrix}
	2&		0&		1\\
	-1&		1&		0\\
	0&		1&		0\\
\end{matrix} \right),
$$
$|A|\ne 0$,所以 $\mathscr{A}$ 可逆。又因为 \[
A^{-1}=\left( \begin{matrix}
	0&		-1&		1\\
	0&		0&		1\\
	1&		2&		-2\\
\end{matrix} \right).
\]
所以 $\mathscr{A}^{-1}$ 为
\[
\mathscr{A}^{-1} (x_1, x_2, x_3) = (x_3, -x_1 + 2 x_3, x_1 + x_2 - 2 x_3 ).
\]
\end{solution}

\item[五、]
分块矩阵\[
M=\left( \begin{array} { c c } { A } & { B } \\ { B ^ { T } } & { D } \end{array} \right).
\]
为正定矩阵,其中 $B$ 是 $B$ 的转置。证明:\\
(1) $A$ 可逆。\\
(2) $D - B^T A^{-1}B$ 也是正定。
\begin{proof}
(1) 由于 $M$ 正定,因此 $M$ 的所有主子式全大于 $0$。而 $A$ 是 $M$ 的 $r$ 阶顺序主子式,因此 $|A|\ne 0$,从而 $A$ 可逆。\\
(2) 引理:设\[
A = \left( \begin{array} { l l } { A _ { 1 } } & { A _ { 2 } } \\ { A _ { 3 } } & { A _ { 4 } } \end{array} \right)
\]
是一个 $ n $ 级对称矩阵,且 $A_1$ 是 $r$ 级可逆矩阵。证明:\[
A \simeq \left( \begin{array} { c c } { A _ { 1 } } & { 0 } \\ { 0 } & { A _ { 4 } - A _ { 2 } ^ { \prime } A _ { 1 } ^ { - 1 } A _ { 2 } } \end{array} \right).
\]
\begin{proof}
由于 $ A $ 是对称矩阵,因此 $A'=A$,即\[
\left( \begin{array} { c c } { A _ { 1 } ^ { \prime } } & { A _ { 3 } ^ { \prime } } \\ { A _ { 2 } ^ { \prime } } & { A _ { 4 } ^ { \prime } } \end{array} \right) = \left( \begin{array} { l l } { A _ { 1 } } & { A _ { 2 } } \\ { A _ { 3 } } & { A _ { 4 } } \end{array} \right).
\]
从而 $A_1$,$A$ 都是对称矩阵,且 $A_3=A'_2$。由于$A_1$可逆,因此\[
\left( \begin{array} { l l } { A _ { 1 } } & { A _ { 2 } } \\ { A _ { 2 } ^ { \prime } } & { A _ { 4 } } \end{array} \right)\rightarrow \left( \begin{matrix}
	A_1&		A_2\\
	0&		A_4-A_{2}^{'}A_{1}^{-1}A_2\\
\end{matrix} \right) \rightarrow \left( \begin{array} { c c } { A _ { 1 } } & { 0 } \\ { 0 } & { A _ { 4 } - A _ { 2 } ^ { \prime } A _ { 1 } ^ { - 1 } A _ { 2 } } \end{array} \right).
\]
从而\[
\left( \begin{array} { c c } { I _ { r } } & { 0 } \\ { - A _ { 2 } ^ { \prime } A _ { 1 } ^ { - 1 } } & { I _ { n r } } \end{array} \right) \left( \begin{array} { c c } { A _ { 1 } } & { A _ { 2 } } \\ { A _ { 3 } } & { A _ { 4 } } \end{array} \right) \left( \begin{array} { c c } { I _ { r } } & { - A _ { 1 } ^ { - 1 } A _ { 2 } } \\ { 0 } & { I _ { m r } } \end{array} \right) = \left( \begin{array} { c c } { A _ { 1 } } & { 0 } \\ { 0 } & { A _ { 1 } - A _ { 2 } ^ { \prime } A _ { 1 } ^ { - 1 } A _ { 2 } } \end{array} \right).
\]
由于 $ \left( - A _ { 1 } ^ { - 1 } A _ { 2 } \right) ^ { \prime } = - A _ { 2 } ^ { \prime } \left( A _ { 1 } ^ { - 1 } \right) ^ { \prime } = - A _ { 2 } ^ { \prime } \left( A _ { 1 } ^ { \prime } \right) ^ { - 1 } = - A _ { 2 } ^ { \prime } A _ { 1 } ^ { - 1 } $,因此从上式推出\[
\left( \begin{array} { l l } { A _ { 1 } } & { A _ { 2 } } \\ { A _ { 3 } } & { A _ { 4 } } \end{array} \right) \simeq \left( \begin{array} { c c } { A _ { 1 } } & { 0 } \\ { 0 } & { A _ { 4 } - A _ { 2 } ^ { \prime } A _ { 1 } ^ { - 1 } A _ { 2 } } \end{array} \right).
\]
\end{proof}
回到原题,由引理得\[
\left( \begin{array} { c c } { A } & { B } \\ { B ^ { \prime } } & { D } \end{array} \right) \simeq \left( \begin{array} { c c } { A } & { 0 } \\ { 0 } & { D - B ^ { \prime } A ^ { - 1 } B } \end{array} \right).
\]
从而上式右边的矩阵也是正定矩阵。于是根据刚才证得的结论,$D-B'A^{-1}B$ 是正定矩阵。

\end{proof}

\item[六、]
给定数域 $\mathbb{P}$ 上的分块矩阵 
$$
M = \left( \begin{array} { c c } { A } & { C } \\ { 0 } & { B } \end{array} \right),
$$
其中 $A$ 为 $m \times n$ 的矩阵,$B$ 为 $k \times l$ 的矩阵, 证明:\[
\operatorname { rank } ( A ) + \operatorname { rank } ( B ) \leq \operatorname { rank } ( M ).
\]
注: ${\rm rank}(A)$ 表示矩阵 $A$ 的秩。
\begin{solution}
设 $\operatorname{rank}(A)=r, \operatorname{rank}(B)=t$。则 $A$ 有一个 $r$ 级子矩阵 $A_1$,使得 $|A_1|\ne 0$;$B$ 有一个 $t$ 级子矩阵 $B_1$,使得 $|B_1| \ne 0$。从而 $\left( \begin{smallmatrix}
	A&		C\\
	0&		B\\
\end{smallmatrix} \right) $ 有一个 $r + t$ 阶子式:\[
\left| \begin{array}{cc}{A_{1}} & {C_{1}} \\ {0} & {B_{1}}\end{array}\right|=\left|A_{1}\right|\left|B_{1}\right| \ne 0.
\]
因此 \[
\operatorname{rank} (M) = \operatorname{rank} \left( \begin{array}{cc}{A} & {C} \\ {0} & {B}\end{array}\right) \ge r+t=\operatorname{rank}(A)+\operatorname{rank}(B).
\]
\end{solution}

\item[七、]
设 $A$ 是半正定矩阵, 证明存在唯一的半正定矩阵 $B$ 使得\[
A = B ^ { 2 }.
\]
\begin{solution}
设 $ A $ 是 $ n $ 级半正定矩阵,则存在 $ n $ 级正交矩阵 $ T $,使得 \[
A = T ^ { - 1 } \operatorname { diag } \left\{ \lambda _ { 1 } , \lambda _ { 2 } , \dots , \lambda _ { n } \right\} T,
\]
其中 $\lambda _ { 1 } , \lambda _ { 2 } , \dots , \lambda _ { n }$ 是 $ A $ 的全部特征值,它们全非负。于是有\[
\begin{aligned} A & = T ^ { - 1 } \operatorname { diag } \left\{ \sqrt { \lambda _ { 1 } } , \sqrt { \lambda _ { 2 } } , \dots , \sqrt { \lambda _ { n } } \right\} T T ^ { - 1 } \operatorname { diag } \left\{ \sqrt { \lambda _ { 1 } } , \sqrt { \lambda _ { 2 } } , \dots , \sqrt { \lambda _ { n } } \right\} T \\ & = C ^ { 2 } \end{aligned},
\]
其中\[
C = T ^ { - 1 } \operatorname { diag } \left\{ \sqrt { \lambda _ { 1 } } , \sqrt { \lambda _ { 2 } } , \dots , \sqrt { \lambda _ { n } } \right\} T.
\]
显然 $ C $ 是半正定矩阵。
\end{solution}

\item[八、]
设 $V$ 为 $n$ 维欧式空间,求证:\\
(1) 对 $V$ 中每个线性变换 $\mathscr{A}$,都存在唯一的共轭变换 $\mathscr { A } ^ { * }$,即存在唯一的线性变换 $\mathscr { A } ^ { * }$,使得对任意的 $ \boldsymbol{\alpha} , \boldsymbol{\beta} \in
V$,有\[
( \mathscr { A } \boldsymbol{\alpha} , \boldsymbol{\beta} ) = \left( \boldsymbol{\alpha} , \mathscr { A } ^ { * } \boldsymbol{\beta} \right).
\]
(2) $\mathscr { A }$ 为对称变换当且仅当 $\mathscr { A } ^ { * } = \mathscr { A }$。\\
(3) $\mathscr { A }$ 为正交变换当且仅当\[
\mathscr { A } \mathscr { A } ^ { * } = \mathscr { A } ^ { * } \mathscr { A } = \mathscr { E }.
\]
其中 $\mathscr { E }$ 是 $V$ 上的恒等变换。
\begin{proof}
\begin{lemma}
设 $f$ 是域 $\mathbb{F}$ 上 $n$ 维线性空间 $V$ 上的一个双线性函数。则

$f$ 是非退化的当且仅当映射 $R_f: \boldsymbol{\beta}\longmapsto \boldsymbol{\beta}_R$ 是线性空间 $V$ 到 $V^*$ 的一个同构映射。
\end{lemma}
\begin{subproof}
看\cite{qiugaoxia} 第 435 页。
\end{subproof}
(1) 任给 $\boldsymbol{\beta}\in V$,据已知条件和引理得,$R_f: \boldsymbol{\beta}\longmapsto \boldsymbol{\beta}_R$ 是线性空间 $V$ 到 $V^*$ 的一个同构映射。由于$\boldsymbol{\beta}_R \mathscr{A} \in V^*$,因此存在唯一的向量$\boldsymbol{\beta}' \in V$,使得 $\boldsymbol{\beta}_R\mathscr{A} = \boldsymbol{\beta}'_R$,从而$\boldsymbol{\beta}_R (\mathscr{A}a) = \boldsymbol{\beta}'_R(a), \forall a\in V$。于是有\begin{equation}\label{mathscrAalphabeta}
\left(\mathscr{A}{{\boldsymbol{\alpha}}}, \boldsymbol{\beta}\right)=\left(\boldsymbol{\alpha}, \boldsymbol{\beta}^{\prime}\right), \quad \forall \boldsymbol{\alpha} \in V.
\end{equation}
于是我们得到 $V$ 到自身的一个映射 $\mathscr{A}^* : \boldsymbol{\beta} \longmapsto \boldsymbol{\beta}'$。由 \eqref{mathscrAalphabeta} 式得
\begin{equation}\label{alphaA*beta}
\left(\mathscr{A}{\boldsymbol{\alpha}}, \boldsymbol{\beta}\right) = \left(\boldsymbol{\alpha}, \mathscr{A}^{*} \boldsymbol{\beta}\right), \quad \forall \boldsymbol{\alpha}, \boldsymbol{\beta} \in V.
\end{equation}
现在来验证 $A^*$ 是 $V$ 上的线性变换。任取 $\boldsymbol{\alpha}, \boldsymbol{\beta}, \boldsymbol{\gamma} \in V, k\in \mathbb{R}$,有
\begin{align*}
\left(\boldsymbol{\alpha}, \mathscr{A}^{*}(k \boldsymbol{\beta}+\boldsymbol{\gamma})\right)
&=\left(\mathscr{A}{\boldsymbol{\alpha}}, k \boldsymbol{\beta}+\boldsymbol{\gamma}\right)=\left(\mathscr{A}{\boldsymbol{\alpha}}, k \boldsymbol{\beta}\right)+\left(\mathscr{A}{\boldsymbol{\alpha}}, \boldsymbol{\gamma}\right)\\
&=\overline{k}\left(\mathscr{A}{\boldsymbol{\alpha}}, \boldsymbol{\beta}\right)+\left(\mathscr{A}{\boldsymbol{\alpha}}, \boldsymbol{\gamma}\right)=\overline{k}(\boldsymbol{\alpha}, \mathscr{A}^* \boldsymbol{\beta})+(\boldsymbol{\alpha}, \mathscr{A}^*  \boldsymbol{\gamma})\\
&=(\boldsymbol{\alpha}, k \mathscr{A}^* \boldsymbol{\beta})+(\boldsymbol{\alpha}, \mathscr{A}  ^*\boldsymbol{\gamma})=(\boldsymbol{\alpha}, k \mathscr{A}^*\boldsymbol{\beta}+\mathscr{A} ^* \boldsymbol{\gamma}),
\end{align*}
因此\[
\mathscr{A}^*(k \boldsymbol{\beta}+\boldsymbol{\gamma})=k \mathscr{A}^{*} \boldsymbol{\beta}+\mathscr{A}^{*} \boldsymbol{\gamma}.
\]
从而 $\mathscr{A}^*$ 是 $V$ 上的一个线性变换。这证明了存在性。\\
唯一性。假设还有线性变换 $\mathscr{B}$ 使得
\begin{equation}\label{Aalphabeta}
\left(\mathscr{A}{\boldsymbol{\alpha}}, \boldsymbol{\beta}\right)=(\boldsymbol{\alpha}, \mathscr{B} \boldsymbol{\beta}), \quad \forall \boldsymbol{\alpha} , \boldsymbol{\beta} \in V.
\end{equation}
则从 \eqref{alphaA*beta} 和 \eqref{Aalphabeta} 式得\[
\left(\boldsymbol{\alpha}, \mathscr{A}^{*} \boldsymbol{\beta}\right)=(\boldsymbol{\alpha}, \mathscr{B} \boldsymbol{\beta}), \quad \forall \boldsymbol{\alpha}, \boldsymbol{\beta} \in V.
\]
由此得出,$\mathscr{A}^{*} \boldsymbol{\beta}=\mathscr{B} {\boldsymbol{\beta}}, \forall \boldsymbol{\beta} \in V$,因此 $\mathscr{A}^*=\mathscr{B}$。

(2) 由对称变换的定义与 (1) 的唯一性即得。

(3) 充分性。因为 $\mathscr{A}\mathscr{A}^*=\mathscr{E}$,所以 $\mathscr{A}$ 是可逆的线性变换,当然是满的。又\[
(\mathscr{A}\boldsymbol{\alpha}, \mathscr{A}\boldsymbol{\beta})=(\boldsymbol{\alpha}, \mathscr{A}^*\mathscr{A}\boldsymbol{\beta})=(\boldsymbol{\alpha}, \mathscr{E}\boldsymbol{\beta})=(\boldsymbol{\alpha}, \boldsymbol{\beta}), \forall \boldsymbol{\alpha} , \boldsymbol{\beta} \in V.
\]
因此 $\mathscr{A}$ 为正交变换。

必要性。因为 $\mathscr{A}$ 是正交变换,所以\[
(\mathscr{A}\boldsymbol{\alpha}, \boldsymbol{\beta}) = (\mathscr{A}\boldsymbol{\alpha}, \mathscr{A}\mathscr{A}^{-1}\boldsymbol{\beta}) = (\boldsymbol{\alpha}, \mathscr{A}^{-1}\boldsymbol{\beta}).
\]
由伴随变换的唯一性知,$\mathscr{A}^{-1}=\mathscr{A}^*$,所以\[
\mathscr { A } \mathscr { A } ^ { * } = \mathscr { A } ^ { * } \mathscr { A } = \mathscr { E }.
\]
\end{proof}
\end{enumerate}

% 
%\newenvironment{Quotation}[1]%
%{\newcommand\quotesource{#1}%
%\begin{quotation}}%
%{\par\hfill——\textit{\quotesource} %
%\end{quotation}}
%\footnotetext[1]{
%\begin{Quotation}{洪家男}
% 做鬼脸:只有帅到一定程度才可以做鬼脸。
%\end{Quotation}}
%
%
%\footnotetext[2]{\begin{Quotation}{熊森林}
% 数
%\end{Quotation}}
\endinput






