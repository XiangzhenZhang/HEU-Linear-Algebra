\section{2013}
\begin{enumerate}[1~]
\renewcommand{\labelenumi}{\textbf{\theenumi. }}
\renewcommand{\Im}{\text{Im }}
\item[一、]填空题 
\begin{enumerate}[1.~]
\item
设$f ( x ) = x ^ { 4 } - 10 x ^ { 2 } + 1 ,  g ( x ) = x ^ { 4 } - 4 \sqrt { 2 } x ^ { 3 } + 6 x ^ { 2 } + 4 \sqrt { 2 } x + 1$,则 $f(x)$ 和 $g(x)$ 的首一最大公因式为(\ \ \ \ \ )。
\begin{solution}
$(f(x), g(x)) = (f(x), f(x)-g(x)) = (x^4 - 10x^2 +1, 4\sqrt{2} x^3-16 x^2-4\sqrt{2}x)$。\\
显然$x \nmid x^4-10x^2 +1$,所以\[
(x^4 -10x^2+1, 4 \sqrt { 2 } x ^ { 3 } -1 6 x ^ { 2 } - 4 \sqrt { 2 } x ) = (x^4 -10x^2+1,  x ^ { 2 } - 2 \sqrt { 2 } x -1 ). \]
方程$x^2 -2\sqrt{2}x -1 = 0$的两个解为$x_1 = \sqrt{2}+\sqrt{3}$,$x_2 = \sqrt{2} -\sqrt{3}$。不难验证$x_1$和$x_2$都不满足方程$x^4-10x^2+1 = 0$,所以$x^4-10x^2+1$和$x^2-2\sqrt{2}x-1$没有相同的实数根,所以$(x^4-10x^2+1, x^2-2\sqrt{2}x-1) = 1$,因此$(f(x), g(x)) = 1$,即$f(x)$与$g(x)$的最大公因式为1。
\end{solution}

\item 
行列式$\left| \begin{array} { c c c c c } { 5 } & { 3 } & { 3 } & { 3 } & { 3 } \\ { 3 } & { 5 } & { 3 } & { 3 } & { 3 } \\ { 3 } & { 3 } & { 5 } & { 3 } & { 3 } \\ { 3 } & { 3 } & { 3 } & { 5 } & { 3 } \\ { 3 } & { 3 } & { 3 } & { 3 } & { 5 } \end{array} \right|$中,第一行元素的代数余子式之和为(\ \ \ \ )。
\begin{solution}
将题中行列式的第一行元素全换为1,则原行列式的第一行元素的代数余子式之和为$\left| \begin{matrix}
	1&		1&		1&		1&		1\\
	3&		5&		3&		3&		3\\
	3&		3&		5&		3&		3\\
	3&		3&		3&		5&		3\\
	3&		3&		3&		3&		5\\
\end{matrix} \right|=\left| \begin{matrix}
	1&		1&		1&		1&		1\\
	0&		2&		0&		0&		0\\
	0&		0&		2&		0&		0\\
	0&		0&		0&		2&		0\\
	0&		0&		0&		0&		2\\
\end{matrix} \right|=16$。
\end{solution}

\item
设 
$$
A = \left( \begin{matrix} { c c c } { 4 } & { 0 } & { 2 } \\ { 8 } & { 0 } & { 4 } \\ { - 10 } & { 0 } & { - 5 } \end{matrix} \right),
$$
则$A^{2015}=$(\ \ \ \ )。

\begin{solution}
先求矩阵$A$的特征多项式:

$$
\phi(\lambda) = \left| \begin{matrix}
	\lambda -4&		0&		-2\\
	-8&		\lambda&		-4\\
	10&		0&		\lambda+5\\
\end{matrix} \right| = \lambda^2 (\lambda+1).
$$
用 $\varphi(\lambda) $ 去除 $\lambda^n$,作带余除法,得\begin{equation}\label{20130103daiyuchufa}
\lambda^n = h(\lambda) \varphi(\lambda) +a \lambda^2+b \lambda +c .\end{equation}

其中$h(\lambda) \in \mathbb{R}[\lambda],a, b, c \in \mathbb{R}$。\\
由于 $0$ 是 $A$ 的 $2$ 重特征值,$-1$ 是 $A$ 的 $1$ 重特征值,因此在  \eqref{20130103daiyuchufa} 中分别令$\lambda = 0, \lambda = -1$有 \begin{equation}\label{20130103abc}
\left\{ \begin{array}{l}
0 = c,\\
(-1)^n = a-b+c.\\
\end{array} \right.\end{equation}
在  \eqref{20130103daiyuchufa} 式两端同时求导,得
\begin{equation} \label{20130103qiudaohou}
n \lambda^{n-1} = h'(\lambda) \phi(\lambda) +h(\lambda) \phi'(\lambda) + 2a\lambda + b.
\end{equation}

在 \eqref{20130103qiudaohou} 中令$\lambda = 0$,得
\begin{equation} \label{20130103b}
0 = b.
\end{equation}

由 \eqref{20130103abc} 和  \eqref{20130103b} 得
$$
\left\{ \begin{array}{l}
a = (-1)^n,\\
b = 0,\\
c = 0.\\
\end{array}\right.
$$

因此,
\begin{equation} \label{20130103lambda2}
\lambda^n = h(\lambda)\phi (\lambda) + (-1)^n \lambda^2.
\end{equation}

将 $\lambda$ 用 $A$ 代入  \eqref{20130103lambda2} 式,由 Hamilton--Cayley 定理得
$$
A^{2015} = - A^2 = \left( \begin{matrix}
4&  0&  2\\
8&  0&  4\\
-10&  0&  -5
\end{matrix} \right).
$$

因此,
$$
A^n =  \left( \begin{matrix}
4&  0&  2\\
8&  0&  4\\
-10&  0&  -5
\end{matrix} \right).
$$
\end{solution}

\item
设$A$ 为 $3$ 阶方阵,$|A|=2$,$A^*$ 为 $A$ 的伴随阵,若 $M = \left( \begin{smallmatrix}{ A ^ { 2 } + 3 A ^ { * } } & { 2 A ^ { * } } \\ { A } & { 0 } \end{smallmatrix} \right)$,则 $(M^{-1})^*=$(\ \ \ \ )。
\begin{solution}
$(M^{-1})^{*} = |M^{-1}| (M^{-1})^{-1} = |M^{-1}| M = \frac{1}{|M|} M$。\\
因为$|M| = -|A||2A^*| = -|A| 2^3 |A|^2 = -64$,\\
所以\[
(M^{-1})^* = -\frac{1}{64}M = -\frac{1}{64} \left( \begin{matrix}
A^2+3A^*&  2A^*\\
A&  0
\end{matrix} \right).
\]
\end{solution}

\item
多项式空间 $\mathbb{R}[x]_2$ 上定义内积 $(f(x), g(x))=\int_0^1 f(x)g(x)\,\mathrm{d}x$,则$\mathbb{R}[x]_2$的一组标准正交基$f_1(x) = 1,f_2 (x)=$ (\ \ \ \ )。
\begin{solution}
取$\mathbb{R}[x]_2$的一个基是$1, x$。先用Schmidt正交化法求出与$f_1(x)$正交的向量$g_2(x)$,\[
g_2(x) = x-\frac{\int_0^1 x \,{\rm d}x}{\int_0^1 1 \,{\rm d}x} = x-\frac12. \]
将$g_2(x)$单位化,得\[
f_2(x) = \frac{g_2(x)}{\sqrt{(g_2(x), g_2(x))}} = \frac{g_2(x)}{\sqrt{\int_0^1 (x-t)^2 \ {\rm dx}}} = 2 \sqrt{3} x-\sqrt{3}. \]
因此,$f_2(x) = 2 \sqrt{3}x -\sqrt{3}$。
\end{solution}

\item
线性空间 $\mathbb{R}^{2\times2}$ 中,基 (1):
$$
A _ { 1 } = \left( \begin{array} { c c } { 1 } & { 0 } \\ { 0 } & { 0 } \end{array} \right) , A _ { 2 } = \left( \begin{array} { c c } { 1 } & { 1 } \\ { 0 } & { 0 } \end{array} \right) , A _ { 3 } = \left( \begin{array} { c c } { 1 } & { 1 } \\ { 1 } & { 0 } \end{array} \right) , A _ { 4 } = \left( \begin{array} { c c } { 1 } & { 1 } \\ { 1 } & { 1 } \end{array} \right);
$$
基 (2):
$$
B _ { 1 } = \left( \begin{array} { c c } { 1 } & { 0 } \\ { 1 } & { 1 } \end{array} \right) , B _ { 2 } = \left( \begin{array} { c c } { 0 } & { 1 } \\ { 1 } & { 1 } \end{array} \right) , B _ { 3 } = \left( \begin{array} { c c } { 1 } & { 1 } \\ { 1 } & { 0 } \end{array} \right) , B _ { 4 } = \left( \begin{array} { c c } { 1 } & { 1 } \\ { 0 } & { 1 } \end{array} \right).
$$
则在基(1)与基(2)下有相同坐标的矩阵为 (\ \ \ \ )。
\begin{solution}
设 $A$ 在基(1)与基(2)下有相同的坐标 $(x_1, x_2, x_3, x_4), x_1, x_2, x_3, x_4\in \mathbb{R}$,则\[
x_1 A_1 +x_2 A_2 +x_3 A_3 +x_4 A_4 = x_1 B_1 +x_2 B_2 +x_3 B_3 +x_4 B_4.
\]
即
\[
\left( \begin{matrix}
	x_1+x_2+x_3+x_4&		x_2+x_3+x_4\\
	x_3+x_4&		x_4\\
\end{matrix} \right) =\left( \begin{matrix}
	x_1+x_3+x_4&		x_2+x_3+x_4\\
	x_1+x_2+x_3&		x_1+x_2+x_4\\
\end{matrix} \right) .
\]
由此得\[
\left\{ \begin{array}{l}
	x_1=x_2=x_4=0,\\
	x_3=x_3.\\
\end{array} \right. 
\]
因此,在基(1)与基(2)下有相同坐标的矩阵为 $\left( \begin{smallmatrix}
	k&		k\\
	k&		0\\
\end{smallmatrix} \right) $,其中 $k \in \mathbb{R}$。
\end{solution}

\item
设$A$为 $3$ 阶半正定阵,向量$\boldsymbol{\alpha}, \boldsymbol{\beta}$线性无关,若$\boldsymbol{\alpha}^T A \boldsymbol{\alpha} = \boldsymbol{\beta}^T A \boldsymbol{\beta} = 0$,且${\rm tr} A = 2$,则二次型$f(x_1, x_2, x_3) = \x^TA\x$经正交变换$x = Py$化成的标准形为 (\ \ \ \ )。
\begin{solution}
由向量$\boldsymbol{\alpha}, \boldsymbol{\beta}$线性无关,$\boldsymbol{\alpha} ^TA\boldsymbol{\alpha} =\boldsymbol{\beta} ^TA\boldsymbol{\beta} =0$ 得,$A\text{的秩为}1$,又 ${\rm tr}(A)=2$,所以 $A$ 的特征值为 $2$ (一重),$0$(二重)。因此,标准形为 $2y_1^2$。
\end{solution}

\item
设 $A = \left( \begin{smallmatrix}
1&  3\\
0&  2
\end{smallmatrix} \right)$,则 $3A^8-9A^7+6A^6+A^5-3A^4+2A^3+2A+E =$ (\ \ \ \ )。
\begin{solution}
$A$ 的特征多项式为\[
f(\lambda) = |\lambda I -A| = \left| \begin{matrix}
\lambda-1&  -3\\
0&  \lambda-2
\end{matrix} \right| = (\lambda-1)(\lambda-2). \]
用 $f(\lambda)$ 去除 $3 \lambda^8 -9 \lambda^7 +6 \lambda^6 +\lambda^5 - 3\lambda^4 +2\lambda^3 +2\lambda +1 =: g(\lambda)$,作带余除法,得\begin{equation} \label{20130108daiyuchufa}
g(\lambda) = h(\lambda) f(\lambda) +a \lambda +b. 
\end{equation}
其中,$h(\lambda) \in \mathbb{R}[\lambda], a,b \in \mathbb{R}.$\\
在  \eqref{20130108daiyuchufa} 式中分别令 $\lambda=1 ,\lambda =2$,得\[
\left\{ \begin{array}{l}
3 = a+b,\\
5 = 2a+b. \end{array}\right.
\]
解得 \[
\left\{ \begin{array}{l}
a =2,\\
b =1. \end{array}\right.
\]
因此 \begin{equation} \label{20130108daiyuchufadingxishu}
g(\lambda) = h(\lambda) f(\lambda) +2\lambda +1.
\end{equation}
在  \eqref{20130108daiyuchufadingxishu} 式中将 $\lambda$ 用$A$ 代入,由 Hamilton-Cayley 定理,得\[
g(A)  = 2A+E =\left( \begin{matrix}
3&  6\\
0&  5 
\end{matrix} \right). \]
因此,\[
3 A^8 -9 A^7 +6 A^6 +A^5 - 3A^4 +2A^3 +2A+E = \left( \begin{matrix}
3&  6\\
0&  5 
\end{matrix} \right). \]
\end{solution}

\item
设3阶方阵$A$的特征值为 $-1, -2, -2$,则$|(\frac12 A)^{*}| = $(\ \ \ \ )。
\begin{solution}
因为$|A| = \lambda_1 \lambda_2 \lambda_3 = 4$,所以\begin{align*}
\left|\left(\frac12 A \right)^*\right| &= \left|\left|\frac12 A\right| \cdot \left(\frac12 A\right)^{-1}\right| = \left|\frac12 A\right|^2 \\
&= \left(\left(\frac12\right)^3 |A|^2\right) = \left(\frac12\right)^6 \cdot 4^2 \\
&=\frac14.
\end{align*}
\end{solution} 

\end{enumerate}

\end{enumerate}
\endinput