\section{2008}
\begin{enumerate}[1~]
\renewcommand{\labelenumi}{\textbf{\theenumi. }}
\renewcommand{\Im}{\text{Im }}
\item[一、]
填空题 (每小题 4 分,共 20 分)
\begin{enumerate}[1.~]
\item
令 $\mathbb{F}$ 为 $\varepsilon=\cos \frac{2 \pi}{5}+i\sin \frac{2\pi}{5}$ 的一切有理式的集合所构成的数域,则 $\mathbb{F}$ 中元素的简约形式为(\quad)。
\begin{solution}
由题意知 $\mathbb{F} = \mathbb{Q}(\varepsilon)$,显然 $\varepsilon$ 在 $\mathbb{Q}$ 上的极小多项式为 $x^5 - 1$,因此 $\mathbb{F}$ 中元素的简约形式为 $c_0 + c_1 \varepsilon + c_2 \varepsilon^2 + c_3 \varepsilon^3 + c_4 \varepsilon^4$,其中 $c_i \in \mathbb{Q}, i = 0, 1, 2, 3, 4$。(相关知识可看\cite{qiujin})
\end{solution}

\item
多项式 $x^{21} - 1$ 和 $x^4 - 1$ 的最高公因式为(\quad)。
\begin{solution}
(把最高公因式改为首一最高(或最大)公因式)因为 $x^{21} - 1$ 和 $x^4 - 1$ 的首一最大公因式不随数域的扩大而改变,所以我们只需求 $x^{21} - 1$ 和 $x^4 - 1$ 在复数域上的首一最大公因式。\\
$x^4-1$ 在复数域上的全部根为:$1$,$-1$,$i$,$-i$。不难验证 $x^{21} - 1$ 和 $x^4 - 1$ 在复数域上的公共根是 $1$,因此$x^{21} - 1$ 和 $x^4 - 1$ 在复数域上的首一最大公因式是 $x - 1$,从而 $x^{21} - 1$ 和 $x^4 - 1$ 在任一数域上的首一最大公因式是 $x - 1$。
\end{solution}

\item
$n$ 阶行列式 
$$
D _ { n } = \left| \begin{array} { c c c c } { 2 } & { - 1 } & { } & { } \\ { - 1 } & { \ddots } & { \ddots } & { } \\ { } & { \ddots } & { \ddots } & { - 1 } \\ { } & { } & { - 1 } & { 2 } \end{array} \right| ( n \geq 2 )
$$
的值为(\quad)。
\begin{solution}
把第 2,3,$\dots$,$n$列都加到第1列上,然后按第1列展开:
\begin{align*}
D _ { n } &= \left| \begin{array} { r r r r r r r r } { 1 } & { - 1 } & { 0 } & { 0 } & { \cdots } & { 0 } & { 0 } & { 0 } \\ { 0 } & { 2 } & { - 1 } & { 0 } & { \cdots } & { 0 } & { 0 } & { 0 } \\ { 0 } & { - 1 } & { 2 } & { - 1 } & { \cdots } & { 0 } & { 0 } & { 0 } \\ { \vdots } & { \vdots } & { \vdots } & { \vdots } & { } & { \vdots } & { \vdots } & { \vdots } \\ { 0 } & { 0 } & { 0 } & { 0 } & { \cdots } & { - 1 } & { 2 } & { - 1 } \\ { 1 } & { 0 } & { 0 } & { 0 } & { \cdots } & { 0 } & { - 1 } & { 2 } \end{array} \right|\\
&=1 \cdot D _ { n - 1 } + ( - 1 ) ^ { n + 1 } 1 \cdot ( - 1 ) ^ { n - 1 }\\
&=D_{n-1}+1.
\end{align*}
由此看出,$D _ { 1 } , D _ { 2 } , \dots , D _ { n }$是首项为2、公差为1的等差数列。\\
因此\[
D _ { n } = 2 + ( n - 1 ) \cdot 1 = n + 1 .
\]
\end{solution}

\item
若 $P$ 为 $5$ 阶正交矩阵,则 $|E - P^2|=$(\quad)。
\begin{solution}
因为 $P$ 是正交矩阵,所以$P^2$是正交矩阵。又因为$P^2$是5阶矩阵,所以$P^2$必有特征值 $1$(不然,$(-1)^5=|P^2|=1$,矛盾)。因此$|E-P^2|=0$。
\end{solution}

\item
在空间直角坐标系 $xOy$ 中,向量 $\boldsymbol{\boldsymbol{\alpha}} _ { 1 } = \left( a _ { 11 } , a _ { 12 } , a _ { 13 } \right) ,  \boldsymbol{\boldsymbol{\alpha}} _ { 2 } = \left( a _ { 21 } , a _ { 22 } , a _ { 23 } \right) ,  \boldsymbol{\boldsymbol{\alpha}} _ { 3 } = \left( a _ { 31 } , a _ { 32 } , a _ { 33 } \right)$ 共面的充要条件为(\quad)。
\begin{solution}
$$
\left| \begin{matrix}
	a_{11}&		a_{12}&		a_{13}\\
	a_{21}&		a_{22}&		a_{23}\\
	a_{31}&		a_{32}&		a_{33}\\
\end{matrix} \right|=0$$
\end{solution}

\item
设 
$$
S = \left( \begin{array} { c c } { 0 } & { E _ { 2 } } \\ { - E _ { 2 } } & { 0 } \end{array} \right) , V = \left\{ X \in R ^ { 4 \times 4 } | S X + X ^ { T } S = 0 \right\},
$$
则 $V$ 作为数域 $\mathbb{R}$ 上的线性空间,其维数为(\quad)。

\begin{solution}
设\[
X=\left( \begin{matrix}
	X_{11}&		X_{12}\\
	X_{21}&		X_{22}\\
\end{matrix} \right) ,
\]
则由$
SX+X^TS=0
$
得\[
\left( \begin{matrix}
	0&		X_{11}+X_{12}\\
	-X_{11}-X_{22}&		0\\
\end{matrix} \right)=0.
\]
因此\[
X=\left( \begin{matrix}
	X_{11}&		X_{12}\\
	X_{21}&		-X_{11}\\
\end{matrix} \right), X_{11}, X_{12}, X_{21}\in \mathbb{R}^{2\times 2}.
\]
所以 $V$ 的维数为 $3$。
\end{solution}

\item
设$A , B \in R ^ { m \times n }$,则矩阵方程$A X = B$有解的充要条件为(\quad)。
\begin{solution}
${\rm rank}(A)={\rm rank}(A\ B)$。考虑\[
A(X_1, X_2, \dots, X_n)=(\boldsymbol{\beta}_1, \boldsymbol{\beta}_2, \dots, \boldsymbol{\beta}_n).
\]
\end{solution}

\item
若$3$阶方阵$A=(a_{ij})_{3\times 3}$的特征值为 $1$,$2$,$3$,则$\sum _ { i = 1 } ^ { 3 } \left( \sum _ { j = 1 } ^ { 3 } a _ { i j } a _ { j i } \right) =$(\quad)。
\begin{solution}
$\sum _ { i = 1 } ^ { 3 } \left( \sum _ { j = 1 } ^ { 3 } a _ { i j } a _ { j i } \right) = {\rm tr}(A^2)=1+4+9=14$。
\end{solution}

\item
令$A \in \mathbb{C} ^ { 5 \times 5 } ,  A ^ { 3 } = E ,  1 = 5 - r ( E - A )$,则${\rm tr} \left( E + A + A ^ { 2 } \right) =$(\quad)。
\begin{solution}
因为 $r(E-A)=4$,所以 $1$ 是 $A$ 的一重特征值。因为 $A^3=E$,所以 $A$ 是幂等矩阵,所以 $A$ 的特征值为 $1$(1重),$0$(4重),所以 $E+A+A^2$ 的特征值为 $3$(1重),$1$(4重),因此\[
{\rm tr}(E+A+A^2)=3+1+1+1+1=7.
\]
\end{solution}

\item
特征值为1,1,1,1的一切$4\times4$阶复数矩阵在复数域内按相似可分(\quad)类。
\begin{solution}
任取一个满足已知条件的复数矩阵$A$。\\
若$A$的最小多项式为$\lambda-1$,则$A-I=0$,因此$A=I$。\\
若$A$的最小多项式为$(\lambda-1)^2$,则$A$的Jordan标准形的主对角元为1的Jordan块的总数$N_1$为:\[
N_1=4-{\rm rank}(A-I),
\]
其中$1$级Jordan块$J_1(1)$的个数$N_1(1)$为:\[
N_1(1)={\rm rank}(A-I)^2+{\rm rank}(A-I)^0-2{\rm rank}(A-I)=4-2{\rm rank}(A-I),
\]
$2$级Jordan块$J_2(1)$的个数$N_1(2)$为:\[
N_1(2)={\rm rank}(A-I)^3+{\rm rank}(A-I)^1-2{\rm rank}(A-I)^2={\rm rank}(A-I).
\]
因此,此时$A$的Jordan标准形只可能有以下两种情况:\[
\left( \begin{matrix}
	1&		&		&		\\
	&		1&		&		\\
	&		&		1&		1\\
	&		&		&		1\\
\end{matrix} \right) ,\left( \begin{matrix}
	1&		1&		&		\\
	&		1&		&		\\
	&		&		1&		1\\
	&		&		&		1\\
\end{matrix} \right) .
\]
若$A$的最小多项式为$(\lambda-1)^3$,则$A$的Jordan标准形的主对角元为1的Jordan块的总数$N_1$为:\[
N_1=4-{\rm rank}(A-I),
\]
其中$1$级Jordan块$J_1(1)$的个数$N_1(1)$为:\[
N_1(1)={\rm rank}(A-I)^2+{\rm rank}(A-I)^0-2{\rm rank}(A-I)={\rm rank}(A-I)^2+4-2{\rm rank}(A-I),
\]
$2$级Jordan块$J_2(1)$的个数$N_1(2)$为:\[
N_1(2)={\rm rank}(A-I)^3+{\rm rank}(A-I)^1-2{\rm rank}(A-I)^2={\rm rank}(A-I)-2{\rm rank}(A-I)^2.
\]
$3$级Jordan块$J_3(1)$的个数$N_1(3)$为:\[
N_1(3)={\rm rank}(A-I)^4+{\rm rank}(A-I)^2-2{\rm rank}(A-I)^3={\rm rank}(A-I)^2.
\]
此时只有${\rm rank}(A-I)=2, {\rm rank}(A-I)^2=1$满足要求。因此,此时$A$的Jordan标准形只可能有以下一种情况:\[
\left( \begin{matrix}
	1&		&		&		\\
	&		1&		1&		\\
	&		&		1&		1\\
	&		&		&		1\\
\end{matrix} \right) .
\]
若$A$的最小多项式为$(\lambda-1)^4$,则$A$的Jordan标准形的主对角元为1的Jordan块的总数$N_1$为:\[
N_1=4-{\rm rank}(A-I),
\]
其中$1$级Jordan块$J_1(1)$的个数$N_1(1)$为:\[
N_1(1)={\rm rank}(A-I)^2+{\rm rank}(A-I)^0-2{\rm rank}(A-I)={\rm rank}(A-I)^2+4-2{\rm rank}(A-I),
\]
$2$级Jordan块$J_2(1)$的个数$N_1(2)$为:\[
N_1(2)={\rm rank}(A-I)^3+{\rm rank}(A-I)^1-2{\rm rank}(A-I)^2.
\]
$3$级Jordan块$J_3(1)$的个数$N_1(3)$为:\[
N_1(3)={\rm rank}(A-I)^4+{\rm rank}(A-I)^2-2{\rm rank}(A-I)^3={\rm rank}(A-I)^2-2{\rm rank}(A-I)^3.
\]
$4$级Jordan块$J_4(1)$的个数$N_1(4)$为:\[
N_1(4)={\rm rank}(A-I)^5+{\rm rank}(A-I)^3-2{\rm rank}(A-I)^4={\rm rank}(A-I)^3.
\]
此时只有${\rm rank}(A-I)=3, {\rm rank}(A-I)^2=2, {\rm rank}(A-I)^3=1$满足要求。因此,此时$A$的Jordan标准形只可能有以下一种情况:\[
\left( \begin{matrix}
	1&		1&		&		\\
	&		1&		1&		\\
	&		&		1&		1\\
	&		&		&		1\\
\end{matrix} \right) .
\]
综上,$A$在复数域内按相似可分为5类。
\end{solution}
\begin{remark}
(1) 另一种方法(以$A$的最小多项式为$(\lambda-1)^2$为例):因为$A$的最小多项式是$A$的Jordan标准形中的各个Jordan块的最小多项式(也是特征多项式)的最小公倍式,所以要确定$A$的Jordan标准形,只需要确定$A$的Jordan标准形中的各个Jordan块的最小多项式$(\lambda-1)^{t_i}, i=1, 2, 3, 4$,进而只需要确定次数$t_i, i=1, 2, 3, 4$,$t_i$满足:\[
t_1+t_2+t_3+t_4=2, 0\le t_i\le 2, \max_{1\le i\le 4}t_i=2.
\]
这样的$t_i$只有:$2, 2, 0, 0$,$2, 1, 1, 0$。(0表示没有相应的Jordan块)\\
(2) 因为存在可逆矩阵$P$使得$P(A-I)P^{-1}=J-I$,所以${\rm rank}(A-I)^r={\rm rank}(J-I)^r, 0\le r \le 4$,由$J-I$的形状得:${\rm rank}(A-I)^{r+1}={\rm rank}(A-I)^r-1$。
\end{remark}

\end{enumerate}

\item[二、](15分)
实数域$\mathbb{R}$上的次数不超过2的多项式集合$\mathbb{P}_2[x]$为实数域上的线性空间。取$\mathbb{P}_2[x]$的一个基\[
\boldsymbol{\alpha} _ { 1 } = \left( 1 , x , x ^ { 2 } \right) \left( \begin{array} { c } { 1 } \\ { 1 } \\ { 1 } \end{array} \right) ,  \boldsymbol{\alpha} _ { 2 } = \left( 1 , x , x ^ { 2 } \right) \left( \begin{array} { c } { 1 } \\ { 1 } \\ { 0 } \end{array} \right) ,  \boldsymbol{\alpha} _ { 3 } = \left( 1 , x , x ^ { 2 } \right) \left( \begin{array} { l } { 1 } \\ { 0 } \\ { 0 } \end{array} \right).
\]
设$\sigma$为$\mathbb{P}[x]$中的线性变换,且\[
\sigma \left( \boldsymbol{\alpha} _ { 1 } \right) = \left( 1 , x , x ^ { 2 } \right) \left( \begin{array} { c } { 3 } \\ { 2 } \\ { 1 } \end{array} \right) , \quad \sigma \left( \boldsymbol{\alpha} _ { 2 } \right) = \left( 1 , x , x ^ { 2 } \right) \left( \begin{array} { c } { 3 } \\ { 2 } \\ { - 1 } \end{array} \right) , \sigma \left( \boldsymbol{\alpha} _ { 3 } \right) = \left( 1 , x , x ^ { 2 } \right) \left( \begin{array} { c } { 1 } \\ { 0 } \\ { 1 } \end{array} \right).
\]
(1) 求$\sigma$在基$\boldsymbol{\alpha}_1, \boldsymbol{\alpha}_2, \boldsymbol{\alpha}_3$下的矩阵$A$;\\
(2) 求$\sigma$的特征值和特征向量;\\
(3) 说明$\sigma$可对角化,并求$\mathbb{P}_2[x]$的一个基$\boldsymbol{\beta}_1, \boldsymbol{\beta}_2, \boldsymbol{\beta}_3$使$\sigma$在此基下的矩阵为对角矩阵。
\begin{solution}
(1) 因为\[
\sigma(\boldsymbol{\alpha}_1, \boldsymbol{\alpha}_2, \boldsymbol{\alpha}_3)=(\boldsymbol{\alpha}_1, \boldsymbol{\alpha}_2, \boldsymbol{\alpha}_3)A,
\]
所以\[
(1, x, x^2)\left( \begin{matrix}
	3&		3&		1\\
	2&		2&		0\\
	1&		-1&		1\\
\end{matrix} \right) =\left( 1,x,x^2 \right) \left( \begin{matrix}
	1&		1&		1\\
	1&		1&		0\\
	1&		0&		0\\
\end{matrix} \right) A.
\]
因此\[
A=\left( \begin{matrix}
	1&		1&		1\\
	1&		1&		0\\
	1&		0&		0\\
\end{matrix} \right) ^{-1}\left( \begin{matrix}
	3&		3&		1\\
	2&		2&		0\\
	1&		-1&		1\\
\end{matrix} \right) =\left(
\begin{matrix}
 0 & 0 & 1 \\
 0 & 2 & 0 \\
 1 & 1 & 0 \\
\end{matrix}
\right).
\]
(2) $\sigma$的特征值和特征向量即$A$的特征值和特征向量。$A$的特征多项式为:\[
f(\lambda)=|\lambda I-A|=(\lambda+1)(\lambda-1)(\lambda-2),
\]
因此$A$的特征值为$-1$,$1$,$2$。\\
解线性方程组$(-I-A)x=0$,得一个基础解系:\[
\boldsymbol{\xi}_1=(1, 0, -1)'.
\]
解线性方程组$(I-A)x=0$,得一个基础解系:\[
\boldsymbol{\xi}_2=(1, 0, 1)'.
\]
解线性方程组$(2I-A)x=0$,得一个基础解系:\[
\boldsymbol{\xi}_3=(1, 3, 2)'.
\]
因此$A$的属于特征值$-1$的特征向量为:$k\boldsymbol{\xi}_1, k\in \mathbb{Z}$,$A$的属于特征值$1$的特征向量为:$l\boldsymbol{\xi}_2, l\in \mathbb{Z}$,$A$的属于特征值$2$的特征向量为:$m\boldsymbol{\xi}_3, m\in \mathbb{Z}$。\\
故$\sigma$的特征值为$-1$,$1$,$2$,$\sigma$的属于特征值$-1$的特征向量为:$k\boldsymbol{\xi}_1, k\in \mathbb{Z}$,$\sigma$的属于特征值$1$的特征向量为:$l\boldsymbol{\xi}_2, l\in \mathbb{Z}$,$\sigma$的属于特征值$2$的特征向量为:$m\boldsymbol{\xi}_3, m\in \mathbb{Z}$。\\
(3) 因为$\sigma$有三个互不相同的特征值,所以$\sigma$可对角化。因为\[
\sigma(\boldsymbol{\xi}_1, \boldsymbol{\xi}_2, \boldsymbol{\xi}_3)=(\boldsymbol{\xi}_1, \boldsymbol{\xi}_2, \boldsymbol{\xi}_3)\left( \begin{matrix}
	-1&		0&		0\\
	0&		1&		0\\
	0&		0&		2\\
\end{matrix} \right) ,
\]
且$\boldsymbol{\xi}_1, \boldsymbol{\xi}_2, \boldsymbol{\xi}_3$显然是线性无关的。所以,取$\boldsymbol{\beta}_1=\boldsymbol{\xi}_1, \boldsymbol{\beta}_2=\boldsymbol{\xi}_2, \boldsymbol{\beta}_3=\boldsymbol{\xi}_3$即可。
\end{solution}

\item[三、](15分)
设$V = \left\{ A \in R ^ { n \times n } | \operatorname { tr } ( A ) = 0 \right\} , W = \{ a E | a \in R \}$。\\
(1) 求证$V$是$\mathbb{R}^{n\times n}$的子空间,并求$\dim V$;\\
(2) 求证$\mathbb{R} ^ { n \times n } = V \oplus W$。
\begin{proof}
重复。
\end{proof}

\item[四、](15分)\\
(1) 将二次型$f ( x , y , z ) = - 2 x y + 2 x z + 2 y z$正交标准化;\\
(2) 求三元实函数$f(x, y, z)$在单位球面$x ^ { 3 } + y ^ { 3 } + z ^ { 3 } = 1$上的最大值和最小值,并分别求一个最值点。
\begin{solution}
(1) 二次型$f ( x , y , z )$ 的矩阵为\[
A=\left( \begin{matrix}
	0&		-1&		1\\
	-1&		0&		1\\
	1&		1&		0\\
\end{matrix} \right) .
\]
$A$的特征值为:$-2$(1重),$1$(2重)。\\
解线性方程组$(-2I-A)x=0$,得一个基础解系:\[
\boldsymbol{\xi}_1=(1, 1, -1)'.
\]
解线性方程组$(I-A)x=0$,得一个基础解系:\[
\boldsymbol{\xi}_2=(1, -1, 0)', \boldsymbol{\xi}_3=(1, 0, 1)'.
\]
用Schmidt正交化法将$\boldsymbol{\xi}_1$与$\boldsymbol{\xi}_2$正交化:\begin{align*}
\boldsymbol{\xi}'_2&=\boldsymbol{\xi}_2;\\
\boldsymbol{\xi}'_3&=\boldsymbol{\xi}_3-\frac{(\boldsymbol{\xi}_3, \boldsymbol{\xi}_2)}{(\boldsymbol{\xi}_2, \boldsymbol{\xi}_2)} \boldsymbol{\xi}_2=\left(\frac12, \frac12, 1\right)'
\end{align*}
其中$(\cdot, \cdot)$是欧式空间中的标准内积。将$\boldsymbol{\xi}_1, \boldsymbol{\xi}'_2, \boldsymbol{\xi}'_3$单位化得:
\begin{align*}
\boldsymbol{\eta}_1&=\frac{\boldsymbol{\xi}_1}{|\boldsymbol{\xi}_1|}=\left(\frac{\sqrt{3}}{3}, \frac{\sqrt{3}}{3}, -\frac{\sqrt{3}}{3}\right)';\\
\boldsymbol{\eta}_2&=\frac{\boldsymbol{\xi}'_2}{|\boldsymbol{\xi}'_2|}=\left(\frac{\sqrt{2}}{2}, -\frac{\sqrt{2}}{2}, 0\right)';\\
\boldsymbol{\eta}_3&=\frac{\boldsymbol{\xi}'_3}{|\boldsymbol{\xi}'_3|}=\left(\frac{\sqrt{6}}{6}, \frac{\sqrt{6}}{6}, -\frac{\sqrt{6}}{3}\right)'.
\end{align*}
因此,取\[
T=(\boldsymbol{\eta}_1, \boldsymbol{\eta}_2, \boldsymbol{\eta}_3),
\]
则$T$是正交矩阵,作正交线性变换\[
\left( \begin{array}{c}
	x\\
	y\\
	z\\
\end{array} \right) =T\left( \begin{array}{c}
	x_1\\
	y_1\\
	z_1\\
\end{array} \right) 
\]
则二次型$f(x, y, z)$化为正交标准形$-2x_1^2-2y_1^2+z_1^2$。\\
(2) 
作正交线性变换\[
\left( \begin{array}{c}
	x\\
	y\\
	z\\
\end{array} \right) =T\left( \begin{array}{c}
	x_1\\
	y_1\\
	z_1\\
\end{array} \right) 
\]
后,单位球面$x ^ { 3 } + y ^ { 3 } + z ^ { 3 } = 1$变为单位球面$x_1 ^ { 3 } + y_1 ^ { 3 } + z_1 ^ { 3 } = 1$。
\end{solution}

\item[五、](15分)
设\[
A = \left( \begin{array} { c } { \boldsymbol{\alpha} _ { 1 } } \\ { \vdots } \\ { \boldsymbol{\alpha} _ { m } } \end{array} \right) ,  B = \left( \begin{array} { c } { \boldsymbol{\beta} _ { 1 } } \\ { \vdots } \\ { \boldsymbol{\beta} _ { m } } \end{array} \right) \in R ^ { m \times  n },
\]
求证齐次线性方程组$Ax=0$与$Bx=0$同解的充要条件为行向量组$\boldsymbol{\alpha} _ { 1 } , \dots ,  \boldsymbol{\alpha} _ { \mathrm { m } }$与$\boldsymbol{\beta} _ { 1 } , \dots , \boldsymbol{\beta} _ { \mathrm { m } }$等价。
\begin{proof}
必要性。设齐次线性方程组$Ax=0$的解空间为$W_1$,$Bx=0$的解空间为$W_2$,则有\[
W_1=W_2.
\]
因此\[
{\rm rank}A=n-\dim W_1=n-\dim W_2={\rm rank}B
\]
从而行向量组$\boldsymbol{\alpha} _ { 1 } , \dots ,  \boldsymbol{\alpha} _ { \mathrm { m } }$与$\boldsymbol{\beta} _ { 1 } , \dots , \boldsymbol{\beta} _ { \mathrm { m } }$等价。\\
充分性。任取一个向量$\boldsymbol{\gamma}\in W_1$,有\[
A\boldsymbol{\gamma}=0.
\]
因为行向量组$\boldsymbol{\alpha} _ { 1 } , \dots ,  \boldsymbol{\alpha} _ { \mathrm { m } }$与$\boldsymbol{\beta} _ { 1 } , \dots , \boldsymbol{\beta} _ { \mathrm { m } }$等价,所以存在$m$阶可逆矩阵$P$,使得\[
A=PB.
\]
于是\[
PB\boldsymbol{\gamma}=0,
\]
从而\[
B\boldsymbol{\gamma}=0.
\]
因此\[
W_1\subset W_2.
\]
同理,有\[
W_2\subset W_1.
\]
所以,$W_1=W_2$,于是,齐次线性方程组$Ax=0$与$Bx=0$同解。
\end{proof}

\item[六、](15分)
设$A \in \mathbb{R} ^ { n \times n }$为实对称矩阵,定义
\begin{align*}
\phi _ { A } : \mathbb{R} ^ { n \times n } &\longrightarrow \mathbb{R} ^ { n \times n } \\
X &\longmapsto  A X A ^ { \prime }
\end{align*}
求证$\phi_{A}$为线性空间$\mathbb{R}^{n\times n}$上的可对角化线性变换。
\begin{proof}
因为$A$是实对称矩阵,所以存在正交矩阵$T$,使得\[
T'AT={\rm diag}\{\lambda_1, \lambda_2, \dots, \lambda_n\}.
\]
其中,$\lambda_1, \lambda_2, \dots, \lambda_n$是$A$的全部特征值。因此,有\[
\phi_A(TE_{ij}T')=ATE_{ij}T'A'=\lambda_i\lambda_j TE_{ij}T', i, j, =1, 2, \dots, n.
\]
其中,$E_{ij}, 1\le i, j\le n$是$(i, j)$元素为1,其余元素为0的$n$阶矩阵。于是\[
\phi_A(TE_{11}T', TE_{12}T', \dots, TE_{nn}T')=(TE_{11}T', TE_{12}T', \dots, TE_{nn}T'){\rm diag}\{\lambda_1\lambda_1, \lambda_1\lambda_2, \dots, \lambda_n\lambda_n\}.
\]
因此$\phi_{A}$为线性空间$\mathbb{R}^{n\times n}$上的可对角化线性变换。
\end{proof}

\item[七、](15分)
设$V$为复数域$\mathbb{C}$上的有限维线性空间,$ \boldsymbol{\alpha}$,$\boldsymbol{\beta}$为其上两个可对角化线性变换,且$\boldsymbol{\alpha}\boldsymbol{\beta}=\boldsymbol{\beta}\boldsymbol{\alpha}$,求证:$\boldsymbol{\alpha}$和$\boldsymbol{\beta}$可同时对角化。
\begin{proof}
看 2009 年第六题引理三。
\end{proof}

\item[八、](15 分)
设$f(x, y)$为数域$\mathbb{F}$上的$n$维线性空间$V$上的对称双线性函数,$U$为$V$的子空间,$U ^ { \perp } = \{ v \in V | f ( v , U ) = 0 \}$。若$U \cap U ^ { \perp } = \{ 0 \}$,求证$V = U \oplus U ^ { \perp }$。
\begin{proof}
由题意,只需要证明$V=U+U^{\perp}$。\\
任取$\boldsymbol{\alpha} \in V$,想证存在$\boldsymbol{\alpha} _ { 1 } \in U , \boldsymbol{\alpha} _ { 2 } \in U ^ { \perp }$,使得$\boldsymbol{\alpha} = \boldsymbol{\alpha} _ { 1 } + \boldsymbol{\alpha} _ { 2 }$。
设$\dim U=m$,在$U$中取一个标准正交基$\boldsymbol{\eta} _ { 1 } , \boldsymbol{\eta} _ { 2 } , \dots , \boldsymbol{\eta} _ { m }$,设$\boldsymbol{\alpha} _ { 1 } = \sum _ { i = 1 } ^ { m } k _ { i } \boldsymbol{\eta} _ { i } , k _ { i } ( i = 1,2 , \dots , m )$待定。令$\boldsymbol{\alpha} _ { 2 } =\boldsymbol{\alpha} - \boldsymbol{\alpha} _ { 1 }$,则
\begin{align*}
\boldsymbol{\alpha} _ { 2 } \in U ^ { \perp } &\Longleftrightarrow \left( \boldsymbol{\alpha} _ { 2 } , \boldsymbol{\eta} _ { j } \right) = 0 , \quad j = 1,2 , \dots , m\\
&\Longleftrightarrow \quad \left( \boldsymbol{\alpha} - \boldsymbol{\alpha} _ { 1 } , \boldsymbol{\eta} _ { j } \right) = 0 , \quad j = 1,2 , \dots , m\\
&\Longleftrightarrow \quad ( \boldsymbol{\alpha} , \boldsymbol{\eta} ) = \left( \boldsymbol{\alpha} _ { 1 } , \boldsymbol{\eta} \right) , \quad j = 1,2 , \dots , m\\
&\Longleftrightarrow \left( \boldsymbol{\alpha} , \boldsymbol{\eta} _ { j } \right) = \left( \sum _ { i = 1 } ^ { m } k _ { i } \boldsymbol{\eta} _ { i } , \boldsymbol{\eta} _ { j } \right) , \quad j = 1,2 , \dots , m\\
&\Longleftrightarrow \left( \boldsymbol{\alpha} , \boldsymbol{\eta} _ { j } \right) = \sum _ { i = 1 } ^ { m } k _ { i } \left( \boldsymbol{\eta} _ { i } , \boldsymbol{\eta} _ { b } \right) , \quad j = 1,2 , \dots , m\\
&\Longleftrightarrow \quad \left( \boldsymbol{\alpha} , \boldsymbol{\eta} _ { b } \right) = k , \quad j = 1,2 , \dots , m.
\end{align*}
于是$\boldsymbol{\alpha} _ { 1 } = \sum _ { i = 1 } ^ { m } \left( \boldsymbol{\alpha} , \boldsymbol{\eta} _ { i } \right) \boldsymbol{\eta} _ { i } , \quad \boldsymbol{\alpha} _ { 2 } = \boldsymbol{\alpha} - \boldsymbol{\alpha} _ { 1 }$,则$\boldsymbol{\alpha} = \boldsymbol{\alpha} _ { 1 } + \boldsymbol{\alpha} _ { 2 }$,且$\boldsymbol{\alpha} _ { 1 } \in U , \boldsymbol{\alpha} _ { 2 } \in U ^ { \perp }$。因此$V = U + U ^ { \perp }$。\\
又$U \cap U ^ { \perp } = \{ 0 \}$,所以$V = U \oplus U ^ { \perp }$。
\end{proof}

\end{enumerate}
\endinput