%\chapter{试题及解析}
\section{2004}
\begin{enumerate}[1~]
\renewcommand{\labelenumi}{\textbf{\theenumi. }}
\renewcommand{\Im}{\text{Im }}
\item[一、]
填空题 (每小题 4 分,共 20 分)
\begin{enumerate}[1.~]
\item
使$x^2-2$在$\mathbb{P}[x]$内可约的最小数域为(\quad)。
\begin{solution}
由 $\mathbb{Q}(\sqrt{2})$ 的定义知,包含 $\mathbb{Q}$ 和 $\sqrt{2}$ 的最小数域为 $\mathbb{Q}(\sqrt{2})$。(相关知识可看\cite{qiujin})
\end{solution}

\item
多项式 $x^5-1$ 在多项式环 $\mathbb{Q}[x]$ 中的标准分解式为(\quad)。
\begin{solution}
$(x-1)(x^4+x^3+x^2+x+1)$。
\end{solution}

\item
$n (n\ge2)$ 阶行列式
$$
\left| \begin{array} { c c c c c c } { x } & { y } & { 0 } & { \cdots } & { 0 } & { 0 } \\ { 0 } & { x } & { y } & { \cdots } & { 0 } & { 0 } \\ { \vdots } & { \vdots } & { \vdots } & { \cdots } & { \vdots } & { \vdots } \\ { 0 } & { 0 } & { 0 } & { \cdots } & { x } & { y } \\ { y } & { 0 } & { 0 } & { \cdots } & { 0 } & { x } \end{array} \right|
$$
的值为(\quad)。

\begin{solution}
记原行列式为$D_n$,按第一列展开得\[
D_n=x^{n}+(-1)^{n+1}y^n.
\]
\end{solution}

\item
设$A_m$为$m$阶方阵,$A_n$为$n$阶方阵,$\left|  { A } _ {  { m } } \right| = a$,$\left|  { B } _ {  { n } } \right| = b$,则 $\left| \begin{smallmatrix} { 0 } & { A _ { m } } \\ { B _ { n } } & { 0 } \end{smallmatrix} \right| =$(\quad)。

\begin{solution}
记所求行列式为$D_{m+n}$,按前$m$行展开得\[
D_{m+n}=(-1)^{(1+2+\cdots+m)+[(n+1)+(n+2)+\cdots+(n+m)]}|A_m||B_n|=(-1)^{mn}ab.
\]
\end{solution}

\item
向量组$\boldsymbol{\alpha} _ { 1 } = ( 5,2 , - 3,1 ) ,  \boldsymbol{\alpha} _ { 2 } = ( 4,1 , - 2,3 ) ,  \boldsymbol{\alpha} _ { 3 } = ( 1,1 , - 1 , - 2 ) ,  \boldsymbol{\alpha} _ { 4 } = ( 3,4 , - 1,2 )$的一个极大无关组为(\quad)。
\begin{solution}
记$A=(\boldsymbol{\alpha}'_1, \boldsymbol{\alpha}'_2, \boldsymbol{\alpha}'_3, \boldsymbol{\alpha}'_4)$,对$A$作初等行变换:\[
A=\left( \begin{matrix}
	5&		4&		1&		3\\
	2&		1&		1&		4\\
	3&		-2&		-1&		-1\\
	-1&		3&		-2&		2\\
\end{matrix} \right) \longrightarrow \left(
\begin{array}{cccc}
 1 & 0 & 1 & 0 \\
 0 & 1 & -1 & 0 \\
 0 & 0 & 0 & 1 \\
 0 & 0 & 0 & 0 \\
\end{array}
\right) .
\]
所以$A$的秩为 $3$,向量组$\boldsymbol{\alpha} _ { 1 },  \boldsymbol{\alpha} _ { 2 } ,  \boldsymbol{\alpha} _ { 3 } ,  \boldsymbol{\alpha} _ { 4 } $的一个极大无关组为$\boldsymbol{\alpha} _ { 1 },  \boldsymbol{\alpha} _ { 2 } ,  \boldsymbol{\alpha} _ { 4 } $。
\end{solution}

\item
已知三个 $3$ 维向量$\boldsymbol{\alpha} _ { 1 } ,  \boldsymbol{\alpha} _ { 2 } ,  \boldsymbol{\alpha} _ { 3 }$中$\boldsymbol{\alpha} _ { 1 } ,  \boldsymbol{\alpha} _ { 2 }$线性无关,而矩阵$A = \left( \boldsymbol{\alpha} _ { 1 } , \boldsymbol{\alpha} _ { 2 } , \boldsymbol{\alpha} _ { 3 } \right)$的秩为2,$A^*$为$A$的伴随阵,则齐次线性方程组$A^*x=0$的通解为(\quad)。
\begin{solution}
因为 $A^* A=|A|=0$,所以\[
A\boldsymbol{\alpha}_1=0, A\boldsymbol{\alpha}_2=0, A \boldsymbol{\alpha}_3 = 0.
\]
因为 $\boldsymbol{\alpha}_3$ 可由 $\boldsymbol{\alpha}_1, \boldsymbol{\alpha}_2$ 线性表出,所以齐次线性方程组 $A^*x = 0$ 的基础解系为 $\boldsymbol{\alpha}_1, \boldsymbol{\alpha}_2$。所以通解为 $k_1 \boldsymbol{\alpha}_1 + k_2 \boldsymbol{\alpha}_2, k_1, k_2 \in \mathbb{R}$。
\end{solution}

\item
设$A$为$n$阶方阵,$A^m=0$($m$为一个正整数),则$A$的特征多项式为(\quad)。
\begin{solution}
因为$A^m=0$,所以$x^m$是$A$的零化多项式,从而$A$的极小多项式为$x^k, k\le m, k\in \mathbb{Z}_+$。因此$A$的特征值全为0,从而$A$的特征多项式为$x^n$。
\end{solution}

\item
设3阶实对称阵$A$的特征值$\lambda _ { 1 } = \lambda _ { 2 } = 1$, $\lambda _ { 3 } = - 1$,对应$\lambda _ { 1 } = \lambda _ { 2 } = 1$的特征向量为$\boldsymbol{\alpha} _ { 1 } = ( 2,1,2 ) ^ { \mathrm { T } }$,$\boldsymbol{\alpha} _ { 2 } = ( 1,2 , - 2 ) ^ { \mathrm { T } }$,则对应于$\lambda_3=-1$的特征向量为(\quad)。
\begin{solution}
设$A$对应于$\lambda_3=-1$的特征向量为$(x, y, z)'$。因为实对称矩阵对应于不同特征值的特征向量相互正交,所以\[
\left\{ \begin{array}{l}
	2x+y+2z=0\\
	x+2y-2z=0\\
\end{array} \right. 
\]
解得\[
(x, y, z)'=k(-2, 2, 1)', k\in\mathbb{Z}, k\ne 0.
\]
所以$A$对应于$\lambda_3=-1$的特征向量为$(x, y, z)'=k(-2, 2, 1)', k\in\mathbb{Z}, k\ne 0.$
\end{solution}

\item
设 $A$ 为正交矩阵,且 $|A|=-1$,则 $A$ 必有特征值(\quad)。
\begin{solution}
显然,$-1$。
\end{solution}
\end{enumerate}

\item[二、](15分)
在 $\mathbb{R}^3$ 中,线性变换 $\mathscr{A}$ 定义如下:
\begin{align*}
& { \mathscr{A} } \boldsymbol{\alpha} _ { 1 }  = ( - 5,0,3 )\\
& { \mathscr{A} } \boldsymbol{\alpha} _ { 2 }  = ( 0 , - 1,6 )\\
& { \mathscr{A} } \boldsymbol{\alpha} _ { 1 }  = ( - 5 , - 1,9 )
\end{align*}
其中
\[
\begin{array} { l } { \boldsymbol{\alpha} _ { 1 } = ( - 1,0,2 ) } \\ { \boldsymbol{\alpha} _ { 2 } = ( 0,1,1 ) } \\ { \boldsymbol{\alpha} _ { 1 } = ( 3 , - 1,0 ) } \end{array}.
\]
求 $\mathscr{A}$ 在基$\boldsymbol{\alpha} _ { 1 } ,  \boldsymbol{\alpha} _ { 2 } ,  \boldsymbol{\alpha} _ { 3 }$下的矩阵$A$。
\begin{solution}
因为\[
\mathscr{A}\left( \begin{array}{c}
	\boldsymbol{\alpha} _1\\
	\boldsymbol{\alpha} _2\\
	\boldsymbol{\alpha} _3\\
\end{array} \right) =\left( \begin{array}{c}
	\mathscr{A}\boldsymbol{\alpha} _1\\
	\mathscr{A}\boldsymbol{\alpha} _2\\
	\mathscr{A}\boldsymbol{\alpha} _3\\
\end{array} \right) =A\left( \begin{array}{c}
	\boldsymbol{\alpha} _1\\
	\boldsymbol{\alpha} _2\\
	\boldsymbol{\alpha} _3\\
\end{array} \right) .
\]
所以\[
A=\left( \begin{array}{c}
	\mathscr{A}\boldsymbol{\alpha} _1\\
	\mathscr{A}\boldsymbol{\alpha} _2\\
	\mathscr{A}\boldsymbol{\alpha} _3\\
\end{array} \right) \left( \begin{array}{c}
	\boldsymbol{\alpha} _1\\
	\boldsymbol{\alpha} _2\\
	\boldsymbol{\alpha} _3\\
\end{array} \right)^{-1} =\left( \begin{matrix}
	-5&		0&		3\\
	0&		-1&		6\\
	-5&		-1&		9\\
\end{matrix} \right) \left( \begin{matrix}
	-1&		0&		2\\
	0&		1&		1\\
	3&		-1&		0\\
\end{matrix} \right)^{-1} =\left(
\begin{array}{ccc}
 2 & -1 & -1 \\
 3 & 0 & 1 \\
 5 & -1 & 0 \\
\end{array}
\right).
\]
所以$\mathscr{A}$在基$\boldsymbol{\alpha} _ { 1 } ,  \boldsymbol{\alpha} _ { 2 } ,  \boldsymbol{\alpha} _ { 3 }$下的矩阵$A$为$\left(
\begin{array}{ccc}
 2 & -1 & -1 \\
 3 & 0 & 1 \\
 5 & -1 & 0 \\
\end{array}
\right)$。
\end{solution}

\item[三、](15分)
求正交变换 
$$
\left( \begin{array} { l } { x } \\ { y } \\ { z } \end{array} \right) = P \left( \begin{array} { l } { u } \\ { v } \\ { w } \end{array} \right)
$$
化二次型 $f = x ^ { 2 } + 4 x y + 4 x z + y ^ { 2 } + 4 y z + z ^ { 2 }$ 为标准型。

\begin{solution}
二次型 $f(x, y, z)$ 的矩阵为\[
A=\left( \begin{matrix}
	1&		2&		2\\
	2&		1&		2\\
	2&		2&		1\\
\end{matrix} \right) .
\]
$A$的特征多项式为\[
f(\lambda)=|\lambda I-A|=(\lambda+1)^2(\lambda-5).
\]
所以$A$的特征值为$-1$(二重),$5$(一重)。\\
解线性方程组$(-I-A)x=0$,得一个基础解系:\[
\boldsymbol{\xi}_1=(1, -1, 0)', \boldsymbol{\xi}_2=(1, 0, -1)'.
\]
解线性方程组$(5I-A)x=0$,得一个基础解系:\[
\boldsymbol{\xi}_3=(1, 1, 1)'.
\]
用Schmidt正交化法把$\boldsymbol{\xi}_1, \boldsymbol{\xi}_2$正交化:\begin{align*}
\boldsymbol{\xi}'_1&=\boldsymbol{\xi}_1;\\
\boldsymbol{\xi}'_2&=\boldsymbol{\xi}_2-\frac{(\boldsymbol{\xi}_2, \boldsymbol{\xi}_1)}{(\boldsymbol{\xi}_1, \boldsymbol{\xi}_1)}\boldsymbol{\xi}_1=\left(\frac12, \frac12, -1\right)'.
\end{align*}
把$\boldsymbol{\xi}'_1, \boldsymbol{\xi}'_2, \boldsymbol{\xi}_3$单位化:\begin{align*}
\boldsymbol{\eta}_1&=\frac{\boldsymbol{\xi}'_1}{|\boldsymbol{\xi}'_1|}=\left(\frac{\sqrt{2}}{2}, -\frac{\sqrt{2}}{2}, 0\right)';\\
\boldsymbol{\eta}_2&=\frac{\boldsymbol{\xi}'_2}{|\boldsymbol{\xi}'_2|}=\left(\frac{\sqrt{6}}{6}, \frac{\sqrt{6}}{6}, -\frac{\sqrt{6}}{3}\right)';\\
\boldsymbol{\eta}_3&=\frac{\boldsymbol{\xi}_3}{|\boldsymbol{\xi}_3|}=\left(\frac{\sqrt{3}}{3}, \frac{\sqrt{3}}{3}, \frac{\sqrt{3}}{3}\right)'.\\
\end{align*}
令\[
P=(\boldsymbol{\eta}_1, \boldsymbol{\eta}_2, \boldsymbol{\eta}_3).
\]
则$P$是正交矩阵,且\[
P^T A P={\rm diag}\{2, 2, 5\}.
\]
因此,作正交线性变换\[
\left( \begin{array} { l } { x } \\ { y } \\ { z } \end{array} \right) = P \left( \begin{array} { l } { u } \\ { v } \\ { w } \end{array} \right)
\]
得\[
f=-u^2-v^2+5w^2.
\]
\end{solution}

\item[四、](15分)
设矩阵 
$$
A = \left( \begin{array} { l l l } { 0 } & { 0 } & { 1 } \\ { x } & { 1 } & { y } \\ { 1 } & { 0 } & { 0 } \end{array} \right)
$$
有 3 个线性无关的特征向量,求 $x$ 和 $y$ 应满足的条件。
\begin{solution}
$A$的特征多项式为\[
f(\lambda)=(\lambda-1)^2(\lambda+1).
\]
\begin{align*}
\text{$A$有3个线性无关的特征向量}&\Longleftrightarrow \text{$A$可对角化}\Longleftrightarrow \text{$A$的最小多项式没有重因式}\\
&\Longleftrightarrow \text{$A$的最小多项式为$\lambda-1$或$\lambda+1$或$(\lambda-1)(\lambda+1)$。}
\end{align*}
因为$A\ne I$,$A\ne -I$,所以应该有\[
(A-I)(A+I)=\left( \begin{matrix}
	-1&		0&		1\\
	x&		0&		y\\
	1&		0&		-1\\
\end{matrix} \right) \left( \begin{matrix}
	1&		0&		1\\
	x&		2&		y\\
	1&		0&		1\\
\end{matrix} \right) =\left( \begin{matrix}
	0&		0&		0\\
	x+y&		0&		x+y\\
	0&		0&		0\\
\end{matrix} \right) .
\]
所以$x+y$应满足的条件为$x+y=0$。
\end{solution}

\item[五、](15分)
设$A$为$n$阶方阵,且$A^m=0$($m$为一个大于1的自然数),现令 $e ^ { A } \equiv E _ { n } + A + \frac { 1 } { 2 ! } A ^ { 2 } + \cdots + \frac { 1 } { ( m - 1 ) ! } A ^ { m - 1 }$,求证矩阵 $e^A$ 可逆。

\begin{proof}
由 $A^m=0$ 得\begin{align*}
e^A\cdot e^{-A}&=\left(E _ { n } + A + \frac { 1 } { 2 ! } A ^ { 2 } + \cdots + \frac { 1 } { ( m - 1 ) ! } A ^ { m - 1 }\right)\left(E _ { n } - A + \frac { 1 } { 2 ! } A ^ { 2 } + \cdots + \frac { (-1)^{m-1} } { ( m - 1 ) ! } A ^ { m - 1 }\right)\\
&=E_n.
\end{align*}
所以矩阵$e^A$可逆。
\end{proof}

\item[六、](15分)
设$A$为$n$维数线性空间$V$上的线性变换,且$A^2=A$。求证 $A$ 在 $V$ 的某组基下的矩阵为 $\left(\begin{smallmatrix} { E _ { r } } & { 0 } \\ { 0 } & { 0 } \end{smallmatrix} \right)$,这里 $r = \operatorname{dim} A (V)$。

\begin{proof}
幂等矩阵必可对角化,重题。
\end{proof}

\item[七、](15分)
设$A$,$B$为数域$\mathbb{P}$上的$n$维线性空间$V$上的线性变换,$\lambda _ { 0 }$是$AB$的特征值且$\boldsymbol{\beta}$为相应的特征向量:\\
(1) 若$\lambda _ { 0 } \ne 0$,求证$\lambda _ { 0 }$也是$BA$的特征值,并求相应的一个特征向量;\\
(2) 若$\lambda _ { 0 } = 0$时,$\lambda _ { 0 }$是否也是$BA$的特征值,说明理由。
\begin{proof}
(1) 因为$\lambda _ { 0 }$是$AB$的特征值且$\boldsymbol{\beta}$为相应的特征向量,所以\[
AB\boldsymbol{\beta}=\lambda_0\boldsymbol{\beta}.
\]
因此\[
BAB\boldsymbol{\beta}=\lambda_0B\boldsymbol{\beta}.
\]
于是$\lambda _ { 0 }$也是$BA$的特征值,$BA$的属于特征值$\lambda_0$的一个特征向量为$B\boldsymbol{\beta}$。\\
(2) 是。因为$|BA|=|B||A|=|AB|=0$。
\end{proof}

\item[八、](15 分)
设$V$为复数域上的$n$维线性空间,$\mathscr{A}$,$\mathscr{B}$为$V$上的线性变换,且$\mathscr{A}\mathscr{B}=\mathscr{B}\mathscr{A}$,求证:\\
(1) 若$\lambda$为$\mathscr{A}$的特征值,则$\mathscr{A}$的对应的特征子空间$V_{\lambda}=\{v\in V|\mathscr{A}v=\lambda v\}$为$\mathscr{B}$的不变子空间;\\
(2) $\mathscr{A}$,$\mathscr{B}$至少有一个公共的特征向量。
\begin{proof}
(1) 任取$\boldsymbol{\alpha}\in V_{\lambda}$,有\[
\mathscr{A}\boldsymbol{\alpha}=\lambda\boldsymbol{\alpha}.
\]
又\[
\mathscr{A}\mathscr{B}=\mathscr{B}\mathscr{A}.
\]
所以\[
\mathscr{A}\mathscr{B}\boldsymbol{\alpha}=\mathscr{B}\mathscr{A}\boldsymbol{\alpha}=\lambda \mathscr{B}\boldsymbol{\alpha}.
\]
因此$\mathscr{B}\boldsymbol{\alpha}\in V_{\lambda}$。从而$V_{\lambda}$是$\mathscr{B}$的不变子空间。\\
(2) 因为$V_{\lambda}$是$\mathscr{B}$的不变子空间,所以$\mathscr{B}|V_{\lambda}$是$V_{\lambda}$上的线性变换。由于$V_{\lambda}$是复数域上的线性空间,因此$\mathscr{B}|V_{\lambda}$有特征值。取它的一个特征值$\mu$,则在$V_{\lambda}$中存在非零向量$\boldsymbol{\xi}$,使得$(\mathscr{B}|V_{\lambda})\boldsymbol{\xi}=\mu\boldsymbol{\xi}$,即$\mathscr{B}\boldsymbol{\xi}=\mu\boldsymbol{\xi}$。又有$\mathscr{A}\boldsymbol{\xi}=\lambda\boldsymbol{\xi}$,因此$\boldsymbol{\xi}$是$\mathscr{A}$与$\mathscr{B}$的公共的特征向量。
\end{proof}

\item[九、](15 分)
设$V$为$n$维欧式空间,$V$的保持内积的线性变换称为正交变换,对$V$的任何单位向量$\boldsymbol{\eta}$,线性变换$A _ { \boldsymbol{\eta} }: A _ { \boldsymbol{\eta} } ( \boldsymbol{\alpha} ) = \boldsymbol{\alpha} - 2 ( \boldsymbol{\eta} , \boldsymbol{\alpha} ) \boldsymbol{\eta} ( \boldsymbol{\alpha} \in V )$称为$V$的镜面反射,求证:\\
(1) $A_{\boldsymbol{\eta}}$为正交变换;\\
(2) $A_{\boldsymbol{\eta}}$的行列式为$-1$;\\
(3) 若$\mathscr{A}$为$V$的正交变换,$1$为其特征值,且相应的特征子空间$V_{\lambda}$的维数为$n-1$,则$\mathscr{A}$为$V$的镜面反射。
\begin{proof}
%(1) 引理1:如果$\mathscr{A}$保持向量的内积不变,那么$\mathscr{A}$是$V$上的一个线性变换。
%引理2:如果$\mathscr{A}$是线性变换且保持向量的内积不变,那么$\mathscr{A}$是单射。
%引理3:$n$维欧式空间$V$到自身的一个映射$\mathscr{A}$如果保持向量的内积不变,那么$\mathscr{A}$是正交变换。
%\begin{proof}
%由引理1的证明过程知,只要$\mathscr{A}$保持向量的内积不变,就可得出$\mathscr{A}$是$V$上的一个线性变换。由引理2的证明过程知,只要$\mathscr{A}$是线性变换且保持向量的内积不变,就可得出$\mathscr{A}$是单射。由于$n$维线性空间$V$上的线性变换$\mathscr{A}$如果是单射,那么它必然是满射。从而$\mathscr{A}$是$V$上的一个正交变换。
%\end{proof}
(1)任取$\boldsymbol{\alpha}, \boldsymbol{\beta}\in V$,有\begin{align*}
(A_{\boldsymbol{\eta}}(\boldsymbol{\alpha}), A_{\boldsymbol{\eta}}(\boldsymbol{\beta}))&=(\boldsymbol{\alpha}-2(\boldsymbol{\eta}, \boldsymbol{\alpha})\boldsymbol{\eta}, \boldsymbol{\beta}-2(\boldsymbol{\eta}, \boldsymbol{\beta})\boldsymbol{\eta})\\
&=(\boldsymbol{\alpha}, \boldsymbol{\beta})-2(\boldsymbol{\eta}, \boldsymbol{\beta})(\boldsymbol{\alpha}, \boldsymbol{\eta})-2(\boldsymbol{\eta}, \boldsymbol{\alpha})(\boldsymbol{\eta}, \boldsymbol{\beta})+4(\boldsymbol{\eta}, \boldsymbol{\alpha})(\boldsymbol{\eta}, \boldsymbol{\beta})(\boldsymbol{\eta}, \boldsymbol{\eta})\\
&=(\boldsymbol{\alpha}, \boldsymbol{\beta}).
\end{align*}
(2)因为\[
A_{\boldsymbol{\eta}}(\boldsymbol{\eta})=\boldsymbol{\eta}-2(\boldsymbol{\eta}, \boldsymbol{\eta})\boldsymbol{\eta}=-\boldsymbol{\eta}.
\]
所以$\langle \boldsymbol{\eta} \rangle$是$A_{\boldsymbol{\eta}}$的一个不变子空间,从而存在$\langle \boldsymbol{\eta} \rangle$的正交补空间$\langle \boldsymbol{\eta} \rangle^{\bot}$使得\[
V=\langle \boldsymbol{\eta} \rangle\oplus\langle \boldsymbol{\eta} \rangle^{\bot}.
\]
取$\langle \boldsymbol{\eta} \rangle^{\bot}$的一个标准正交基$\boldsymbol{\eta}_2, \boldsymbol{\eta}_3, \dots, \boldsymbol{\eta}_{n}$,得到$V$的一个标准正交基$\boldsymbol{\eta} \cdot \boldsymbol{\eta} _ { 2 } , \dots , \boldsymbol{\eta} _ { n }$。此时\[
A_{\boldsymbol{\eta}}(\boldsymbol{\eta})=-\boldsymbol{\eta}, A_{\boldsymbol{\eta}}(\boldsymbol{\eta}_i)=\boldsymbol{\eta}_i, i=2, 3, \dots, n.
\]
于是$A_{\boldsymbol{\eta}}$在$V$的标准正交基$\boldsymbol{\eta} , \boldsymbol{\eta} _ { 2 } , \dots , \boldsymbol{\eta} _ { n }$下的矩阵为\[
A=\operatorname { diag } \{ - 1,1 , \dots , 1 \}.
\]
由于$|A|=-1$,所以$A_{\boldsymbol{\eta}}$的行列式为$-1$。\\
(3)$V = V _ { \lambda } \oplus V _ { \lambda } ^ { \perp }$。
由于$\dim V_{\lambda}=n-1$,因此$\dim V_{\lambda}^{\perp}=1$,从而$V_{\lambda}^{\perp}=\langle \boldsymbol{\eta} \rangle$,其中$\boldsymbol{\eta}$是单位向量。由于$\mathscr{A}$的特征子空间$V_{\lambda}$是$\mathscr{A}$的不变子空间,因此
$V_{\lambda}^{\perp}$也是$\mathscr{A}$的不变子空间。从而$\boldsymbol{\eta}$是$\mathscr{A}$的一个特征向量。由于
$\mathscr{A}$的属于$1$的特征子空间$V_{\lambda}$的维数等于$n-1$,且正交变换$\mathscr{A}$的特征值等于$1$或$-1$,因此$\mathscr{A}(\boldsymbol{\eta})=-\boldsymbol{\eta}$。\\
从而\[
\mathscr{A}(\boldsymbol{\eta})=-\boldsymbol{\eta}=\boldsymbol{\eta}-2(\boldsymbol{\eta}, \boldsymbol{\eta})\boldsymbol{\eta}.
\]
在$V_{\lambda}$中取一个基$\boldsymbol{\alpha} _ { 1 } , \boldsymbol{\alpha} _ { 2 } , \dots , \boldsymbol{\alpha} _ { n - 1 }$,则\[
\mathscr{A}(\boldsymbol{\alpha}_i)=\boldsymbol{\alpha}_i=\boldsymbol{\alpha}_i-2(\boldsymbol{\eta}, \boldsymbol{\alpha}_i)\boldsymbol{\eta}, i=1, 2, \dots, n-1.
\]
由于$\boldsymbol{\alpha} _ { 1 } , \boldsymbol{\alpha} _ { 2 } , \dots , \boldsymbol{\alpha} _ { n - 1 } , \boldsymbol{\eta}$是$V$的一个基,因此\[
\mathscr{A}(\boldsymbol{\alpha})=\boldsymbol{\alpha}-2(\boldsymbol{\eta}, \boldsymbol{\alpha})\boldsymbol{\eta}, \text{对任意的$\boldsymbol{\alpha}\in V$}.
\]
从而$\mathscr{A}$是$V$的镜面反射。
\end{proof}
\begin{remark}
镜面反射的另一个定义:设$V$是$n$维欧几里得空间,$\boldsymbol{\eta}$是$V$中一个单位向量,$\mathscr{P}$是$V$在$\langle \boldsymbol{\eta} \rangle$上的正交投影,令\[
\mathscr { A } = \mathscr { I } - 2 \mathscr { P },
\]
则$\mathscr{A}$称为关于超平面$\langle \boldsymbol{\eta} \rangle^{\perp}$的镜面反射。
\end{remark}

\item[十、](10 分)
若$A$为$n$阶方阵,求证$A^n$的秩等于$A^{n+1}$的秩。
\begin{proof}
因为\[
n\ge r(A^0)\ge r(A)\ge r(A^2)\ge \cdots\ge r(A^n)\ge r(A^{n+1})\ge0.
\]
所以由抽屉原理知,存在正整数$m\le n$,使得\[
r(A^m)=r(A^{m+1}).
\]
因此\[
r(A^m)=r(A^{m+1})=\cdots=r(A^n)=r(A^{n+1}).
\]
\end{proof}
\end{enumerate}
\endinput