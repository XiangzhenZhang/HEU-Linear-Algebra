\section{2019}
\begin{enumerate}[1~]
\renewcommand{\labelenumi}{\textbf{\theenumi. }}
\renewcommand{\Im}{\text{Im }}
\item[一、]填空题 
\begin{enumerate}[1.~]
\item
$x^2+x+1$ 除 $x^{1999}+x^{2009}+x^{2019}$ 所得余式为(\ \ \ )。
\begin{solution}
注意到
\begin{align*}
&(x-1)(x^2+x+1)=x^3-1,\\
&(x-1)(x^{1999}+x^{2009}+x^{2019})=(x-1)x^{1999}(1+x^{10}+x^{20})=x^{1999}(x^{30}-1).
\end{align*}
而 $x^3-1$ 的每个根都是 $x^{30}-1$ 的根,所以
\[
x^3-1|x^{30}-1.
\]
所以余式为 $0$。
\end{solution}

\item
已知
$$
\left| \begin{matrix}
	a_{11}&		a_{12}&		a_{13}\\
	a_{21}&		a_{22}&		a_{23}\\
	a_{31}&		a_{32}&		a_{33}\\
\end{matrix} \right|=3,
\left| \begin{matrix}
	b_{11}&		b_{12}\\
	b_{21}&		b_{22}\\
\end{matrix} \right|=2
,
$$
则 
$$
\left| \begin{matrix}
	0&		0&		a_{13}&		a_{14}&		a_{15}\\
	0&		0&		a_{23}&		a_{24}&		a_{25}\\
	0&		0&		a_{33}&		a_{34}&		a_{35}\\
	b_{41}&		b_{42}&		0&		0&		0\\
	b_{51}&		b_{52}&		0&		0&		0\\
\end{matrix} \right|=(\quad).
$$

\begin{solution}
由 Laplace 定理得
\[\left| \begin{matrix}
	0&		0&		a_{13}&		a_{14}&		a_{15}\\
	0&		0&		a_{23}&		a_{24}&		a_{25}\\
	0&		0&		a_{33}&		a_{34}&		a_{35}\\
	b_{41}&		b_{42}&		0&		0&		0\\
	b_{51}&		b_{52}&		0&		0&		0\\
\end{matrix} \right|=\left( -1 \right) ^{3\times 2}\left| \begin{matrix}
	a_{11}&		a_{12}&		a_{13}\\
	a_{21}&		a_{22}&		a_{23}\\
	a_{31}&		a_{32}&		a_{33}\\
\end{matrix} \right|\left| \begin{matrix}
	b_{11}&		b_{12}\\
	b_{21}&		b_{22}\\
\end{matrix} \right|=6.
\]
\end{solution}

\item
设 
$$AP=PB, B=\left(\begin{matrix}
	1&		0&		0\\
	0&		0&		0\\
	0&		0&		-1\\
\end{matrix} \right), P=\left( \begin{smallmatrix}
	1&		0&		0\\
	2&		-1&		0\\
	2&		1&		1\\
\end{smallmatrix} \right),
$$
则 $A^{2019}=$(\quad)。

\begin{solution}
注意到 $P$ 是可逆矩阵,由定义我们可以计算得出
\[
P^{-1}=\left( \begin{matrix}
	1&		0&		0\\
	2&		-1&		0\\
	-4&		1&		1\\
\end{matrix} \right) .
\]
所以,
\begin{align*}
A^{2019}&=(PBP^{-1})^{2019}=PB^{2019}P^{-1}\\
&=\left( \begin{matrix}
	1&		0&		0\\
	2&		-1&		0\\
	2&		1&		1\\
\end{matrix} \right) \left( \begin{matrix}
	1&		0&		0\\
	0&		0&		0\\
	0&		0&		-1\\
\end{matrix} \right) \left( \begin{matrix}
	1&		0&		0\\
	2&		-1&		0\\
	-4&		1&		1\\
\end{matrix} \right)\\
&=\left( \begin{matrix}
	1&		0&		0\\
	2&		0&		0\\
	6&		-1&		-1\\
\end{matrix} \right).
\end{align*}
\end{solution}

\item
在 $\mathbb{R}^{2\times 2}$ 中,向量组 $\alpha_1=\left( \begin{smallmatrix}
	-3&		1\\
	1&		1\\
\end{smallmatrix} \right)$,$\alpha_2=\left( \begin{smallmatrix}
	1&		-3\\
	1&		1\\
\end{smallmatrix} \right) $,$\alpha_3=\left( \begin{smallmatrix}
	1&		1\\
	-3&		1\\
\end{smallmatrix} \right) $,$\alpha_4=\left( \begin{smallmatrix}
	1&		1\\
	1&		-3\\
\end{smallmatrix} \right) $ 的秩为(\quad)。
\begin{solution}
设存在 $k_1$、$k_2$、$k_3$ 和 $k_4\in\mathbb{R}$,使得
$$
k_1\alpha_1+k_2\alpha_2+k_3\alpha_3+k_4\alpha_4=0,
$$
则有
$$
\left(\begin{matrix}
-3 & 1 & 1 & 1 \\
1 & -3 & 1 & 1 \\
1 & 1 & -3 & 1 \\
1 & 1 & 1 & -3 
\end{matrix}\right)
\left(\begin{matrix}
k_1 \\
k_2 \\
k_3 \\
k_4 
\end{matrix}\right)=0.
$$
因此,$\alpha_1,\alpha_2, \alpha_3, \alpha_4$ 的秩就是矩阵 $\left(\begin{smallmatrix}
-3 & 1 & 1 & 1 \\
1 & -3 & 1 & 1 \\
1 & 1 & -3 & 1 \\
1 & 1 & 1 & -3 
\end{smallmatrix}\right)$ 的秩,即 $3$。
\end{solution}
\end{enumerate}
\end{enumerate}
\endinput