\section{2016}
\begin{enumerate}[1~]
\renewcommand{\labelenumi}{\textbf{\theenumi. }}
\renewcommand{\Im}{\text{Im }}
\item[一、]
填空题 (每小题 4 分,共 20 分)
\begin{enumerate}[1.~]
\item
设$(x-1)^2|Ax^4+Bx^2+1$,则$A+B=(\quad)$。
\begin{solution}
记$f(x)=Ax^4+Bx^2+1$,则$1$是$f(x)$的二重实根,因此,\[
f(1)=0.
\]
故\[
	A+B+1=0
\]
因此 $A+B=-1$。
\end{solution}

\item 
设$f ( x ) = \left| \begin{array} { c c c c } { 1 } & { 1 } & { 1 } & { 1 } \\ { 1 } & { x } & { x ^ { 2 } } & { x ^ { 3 } } \\ { 1 } & { - 2 } & { 4 } & { - 8 } \\ { 1 } & { 3 } & { 9 } & { 27 } \end{array} \right| = 0$的三个根为$x_1$, $x_2$,$x_3$,则$x_1+x_2+x_3=(\quad)$。
\begin{solution}
\begin{align*}
f\left( x \right) &=\left| \begin{matrix}
	1&		1&		1&		1\\
	1&		x&		x^2&		x^3\\
	1&		-2&		4&		-8\\
	1&		3&		9&		27\\
\end{matrix} \right|=\left| \begin{matrix}
	1&		1&		1&		1\\
	0&		x-1&		x^2-1&		x^3-1\\
	0&		-3&		3&		-9\\
	0&		2&		8&		26\\
\end{matrix} \right|\\
&=6\left( x-1 \right) \left| \begin{matrix}
	1&		1&		1&		1\\
	0&		1&		x+1&		x^2+x+1\\
	0&		-1&		1&		-3\\
	0&		1&		4&		13\\
\end{matrix} \right|
=6\left( x-1 \right) \left| \begin{matrix}
	1&		1&		1&		1\\
	0&		1&		x+1&		x^2+x+1\\
	0&		0&		x+2&		\left( x+2 \right) \left( x-1 \right)\\
	0&		0&		-\left( x-3 \right)&		-\left( x-3 \right) \left( x+4 \right)\\
\end{matrix} \right|\\
&=-6\left( x+2 \right) \left( x-1 \right) \left( x-3 \right) \left| \begin{matrix}
	1&		1&		1&		1\\
	0&		1&		x+1&		x^2+x+1\\
	0&		0&		1&		\left( x-1 \right)\\
	0&		0&		1&		\left( x+4 \right)\\
\end{matrix} \right|\\
&=-30\left( x+2 \right) \left( x-1 \right) \left( x-3 \right).
\end{align*}
所以,$f(x)$ 的三个根为 $-2$,$1$,$3$。因此,$x_1+x_2+x_3=2$。
\end{solution}

\item
若方程组
$$
\left\{ \begin{array} { l } { x _ { 1 } - x _ { 2 } = a _ { 1 } }, \\ { x _ { 2 } - x _ { 3 } = a _ { 2 } } ,\\ { x _ { 3 } - x _ { 4 } = a _ { 3 } }, \\ { x _ { 4 } - x _ { 5 } = a _ { 4 } } ,\\ { x _ { 5 } - x _ { 1 } = a _ { 5 } }. \end{array} \right.
$$
有解,则 $a_1 + a_2 + a_3 + a_4 + a_5 = (\ \ \ \ )$。
\begin{solution}
方程组的系数矩阵为\[
A=\left( \begin{matrix}
	1&		-1&		0&		0&		0\\
	0&		1&		-1&		0&		0\\
	0&		0&		1&		-1&		0\\
	0&		0&		0&		1&		-1\\
	-1&		0&		0&		0&		1\\
\end{matrix} \right),\]
增广阵为\[
\tilde{A}=\left( \begin{matrix}
	1&		-1&		0&		0&		0&a_1\\
	0&		1&		-1&		0&		0&a_2\\
	0&		0&		1&		-1&		0&a_3\\
	0&		0&		0&		1&		-1&a_4  \\
	-1&		0&		0&		0&		1&a_5\\
\end{matrix} \right).\]
因为$\mathrm{rank}(A)=5$,所以方程组只有零解,所以$a_1+a_2+a_3+a_4+a_5=0$。
\end{solution}

\item
若$3$阶可逆阵$A$交换$1, 3$行得$B$,则 $AB-I = (\quad)$。
\begin{solution}
???
\end{solution}

\item
设 $A=\left( \begin{smallmatrix}
	1&		2&		3\\
	0&		4&		5\\
	0&		0&		6\\
\end{smallmatrix} \right) $,$f(x) = (x - 1)(x - 2)^2(x - 3)^3(x - 4)^4(x - 5)^5(x - 6)^6 + x + 1$,则 $f(A) = (\ \ \ \ )$。
\begin{solution}
矩阵$A$的特征值分别为 $1, 4, 6$,由 Hamilton-Cayley 定理得
$$(A-I)(A-2I)^2(A-3I)^3(A-4I)^4(A-5I)^5(A-6I)^6 = 0.$$
所以
$$f(A) = A+I=\left( \begin{matrix}
	2&		2&		3\\
	0&		5&		5\\
	0&		0&		7\\
\end{matrix} \right). $$
\end{solution}
\end{enumerate}

\item[二、](15分)
设\[
D_n=\left| \begin{array}{c}
	x\\
	z\\
	z\\
	\vdots\\
	z\\
	z\\
\end{array}\begin{array}{c}
	y\\
	x\\
	z\\
	\vdots\\
	z\\
	z\\
\end{array}\begin{array}{c}
	y\\
	y\\
	x\\
	\vdots\\
	z\\
	z\\
\end{array}\begin{array}{c}
	\cdots\\
	\cdots\\
	\cdots\\
	\vdots\\
	\cdots\\
	\cdots\\
\end{array}\begin{array}{c}
	y\\
	y\\
	y\\
	\vdots\\
	x\\
	z\\
\end{array}\begin{array}{c}
	y\\
	y\\
	y\\
	\vdots\\
	y\\
	x\\
\end{array} \right|.
\]
(1) 当$y=z$时,计算$D_n$。\\
(2) 当$y\ne z$时,计算$D_n$。
\begin{proof}
(1) 证法一:当$x=z$时,\[
D _ { n } 
= \left| \begin{matrix}
	z&		z&		z&		\cdots&		z&		z\\
	z&		z&		z&		\cdots&		z&		z\\
	z&		z&		z&		\cdots&		z&		z\\
	\vdots&		\vdots&		\vdots&		\vdots&		\vdots&		\vdots\\
	z&		z&		z&		\cdots&		z&		z\\
	z&		z&		z&		\cdots&		z&		z\\
\end{matrix} \right|=0.
\]
当$x\ne z$时,
\begin{align*}
D _ { n } 
&= \left| \begin{matrix}
	x&		z&		z&		\cdots&		z&		z\\
	z&		x&		z&		\cdots&		z&		z\\
	z&		z&		x&		\cdots&		z&		z\\
	\vdots&		\vdots&		\vdots&		\vdots&		\vdots&		\vdots\\
	z&		z&		z&		\cdots&		x&		z\\
	z&		z&		z&		\cdots&		z&		x\\
\end{matrix} \right|
=\left| \begin{matrix}
	1&		1&		1&		1&		1&		1&		1\\
	0&		x&		z&		z&		\cdots&		z&		z\\
	0&		z&		x&		z&		\cdots&		z&		z\\
	0&		z&		z&		x&		\cdots&		z&		z\\
	0&		\vdots&		\vdots&		\vdots&		\vdots&		\vdots&		\vdots\\
	0&		z&		z&		z&		\cdots&		x&		z\\
	0&		z&		z&		z&		\cdots&		z&		x\\
\end{matrix} \right|\\
&=\left| \begin{matrix}
	1&		1&		1&		1&		1&		1&		1\\
	-z&		x-z&		0&		0&		\cdots&		0&		0\\
	-z&		0&		x-z&		0&		\cdots&		0&		0\\
	-z&		0&		0&		x-z&		\cdots&		0&		0\\
	-z&		\vdots&		\vdots&		\vdots&		\vdots&		\vdots&		\vdots\\
	-z&		0&		0&		0&		\cdots&		x-z&		0\\
	-z&		0&		0&		0&		\cdots&		0&		x-z\\
\end{matrix} \right|\\
&=\left| \begin{matrix}
	1&		1&		1&		1&		1&		1&		1\\
	-z&		x-z&		0&		0&		\cdots&		0&		0\\
	-z&		0&		x-z&		0&		\cdots&		0&		0\\
	-z&		0&		0&		x-z&		\cdots&		0&		0\\
	-z&		\vdots&		\vdots&		\vdots&		\vdots&		\vdots&		\vdots\\
	-z&		0&		0&		0&		\cdots&		x-z&		0\\
	-z&		0&		0&		0&		\cdots&		0&		x-z\\
\end{matrix} \right|\\
&=\left| \begin{matrix}
	1+n\left( \frac{z}{x-z} \right)&		1&		1&		1&		1&		1&		1\\
	0&		x-z&		0&		0&		\cdots&		0&		0\\
	0&		0&		x-z&		0&		\cdots&		0&		0\\
	0&		0&		0&		x-z&		\cdots&		0&		0\\
	0&		\vdots&		\vdots&		\vdots&		\vdots&		\vdots&		\vdots\\
	0&		0&		0&		0&		\cdots&		x-z&		0\\
	0&		0&		0&		0&		\cdots&		0&		x-z\\
\end{matrix} \right|
\end{align*}
\begin{align*}
\quad&=\left[ 1+n\left( \frac{z}{x-z} \right) \right] \left( x-z \right) ^n=\left[ x+\left( n-1 \right) z \right] \left( x-z \right) ^{n-1}.
\end{align*}
$x=z$时,上式仍然成立。因此,$D_n=\left[ x+\left( n-1 \right) z \right] \left( x-z \right) ^{n-1}$。\\
证法二:\begin{align*}
D_n
&=\left| \begin{matrix}
	x&		z&		z&		\cdots&		z&		z\\
	z&		x&		z&		\cdots&		z&		z\\
	z&		z&		x&		\cdots&		z&		z\\
	\vdots&		\vdots&		\vdots&		\vdots&		\vdots&		\vdots\\
	z&		z&		z&		\cdots&		x&		z\\
	z&		z&		z&		\cdots&		z&		x\\
\end{matrix} \right|\\
&=\left| \begin{matrix}
	x+\left( n-1 \right) z&		x+\left( n-1 \right) z&		x+\left( n-1 \right) z&		\cdots&		x+\left( n-1 \right) z&		x+\left( n-1 \right) z\\
	z&		x&		z&		\cdots&		z&		z\\
	z&		z&		x&		\cdots&		z&		z\\
	\vdots&		\vdots&		\vdots&		\vdots&		\vdots&		\vdots\\
	z&		z&		z&		\cdots&		x&		z\\
	z&		z&		z&		\cdots&		z&		x\\
\end{matrix} \right|\\
&=\left[ x+\left( n-1 \right) z \right] \left| \begin{matrix}
	1&		1&		1&		\cdots&		1&		1\\
	z&		x&		z&		\cdots&		z&		z\\
	z&		z&		x&		\cdots&		z&		z\\
	\vdots&		\vdots&		\vdots&		\vdots&		\vdots&		\vdots\\
	z&		z&		z&		\cdots&		x&		z\\
	z&		z&		z&		\cdots&		z&		x\\
\end{matrix} \right|\\
&=\left[ x+\left( n-1 \right) z \right] \left| \begin{matrix}
	1&		1&		1&		\cdots&		1&		1\\
	0&		x-z&		0&		\cdots&		0&		0\\
	0&		0&		x-z&		\cdots&		0&		0\\
	\vdots&		\vdots&		\vdots&		\vdots&		\vdots&		\vdots\\
	0&		0&		0&		\cdots&		x-z&		0\\
	0&		0&		0&		\cdots&		0&		x-z\\
\end{matrix} \right|\\
&=\left[ x+\left( n-1 \right) z \right] \left( x-z \right) ^{n-1}.
\end{align*}
(2) \begin{align}\label{Dn1}
D_n
&=\left| \begin{matrix}
	x&		y&		y&		\cdots&		y&		y\\
	z&		x&		y&		\cdots&		y&		y\\
	z&		z&		x&		\cdots&		y&		y\\
	\vdots&		\vdots&		\vdots&		\vdots&		\vdots&		\vdots\\
	z&		z&		z&		\cdots&		x&		y\\
	z&		z&		z&		\cdots&		z&		x\\
\end{matrix} \right|\\
&=(x-y)D_{n-1}+y(x-z)^{n-1}\quad (n\ge 2).
\end{align}
由对称性,
\begin{equation}\label{Dn2}
D _ { n } = \left| A ^ { \prime } \right| = ( x - z ) D _ { n - 1 } + z ( x - y ) ^ { n - 1 } \quad ( n \ge 2 ).
\end{equation}
由 \eqref{Dn1} 和  \eqref{Dn2} 得,\[
D_n=\frac{y(x-z)^n-z(x-y)^n}{y-z}\quad (n\ge 2).
\]
容易验证上式在$n=1$时也成立。因此,\[
D_n=\frac{y(x-z)^n-z(x-y)^n}{y-z}.
\]
\end{proof}

\item[三、](15分)
设 $\boldsymbol { \boldsymbol{\alpha} }_1, \boldsymbol { \boldsymbol{\alpha} }_2, \dots , \boldsymbol { \boldsymbol{\alpha} }_n$ 为一组 $n$ 维向量, 求证 $\boldsymbol { \boldsymbol{\alpha} }_1, \boldsymbol { \boldsymbol{\alpha} }_2, \dots ,\boldsymbol { \boldsymbol{\alpha} }_n$ 线性无关的充分必要条件为:任意 $n$ 维向量均可由 $\boldsymbol { \boldsymbol{\alpha} }_1, \boldsymbol { \boldsymbol{\alpha} }_2, \dots ,\boldsymbol { \boldsymbol{\alpha} }_n $ 线性表示。
\begin{proof}
必要性。设$\boldsymbol { \boldsymbol{\alpha} } _ { 1 } , \boldsymbol { \boldsymbol{\alpha} } _ { 2 } , \dots , \boldsymbol { \boldsymbol{\alpha} } _ { n }$线性无关。在$\mathbb{K}^n$中任取一个向量$\boldsymbol{\boldsymbol{\beta}}$。由于$\mathbb{K}^n$中任意$n+1$个向量都线性相关,所以向量组$\boldsymbol { \boldsymbol{\alpha} } _ { 1 } , \boldsymbol { \boldsymbol{\alpha} } _ { 2 } , \dots , \boldsymbol { \boldsymbol{\alpha} } _ { n } , \boldsymbol { \boldsymbol{\beta} }$必线性相关。从而$\boldsymbol { \boldsymbol{\beta} }$可以由$\boldsymbol { \boldsymbol{\alpha} } _ { 1 } , \boldsymbol { \boldsymbol{\alpha} } _ { 2 } , \dots , \boldsymbol { \boldsymbol{\alpha} } _ { n } $线性表出。\\
充分性。设$\mathbb{K}^n$中任一向量都可以由$\boldsymbol { \boldsymbol{\alpha} } _ { 1 } , \boldsymbol { \boldsymbol{\alpha} } _ { 2 } , \dots , \boldsymbol { \boldsymbol{\alpha} } _ { n } $线性表出。则$\boldsymbol{\varepsilon} _ { 1 } , \boldsymbol{\varepsilon} _ { 2 } , \dots , \boldsymbol{\varepsilon} _ { n }$可以由$\boldsymbol { \boldsymbol{\alpha} } _ { 1 } , \boldsymbol { \boldsymbol{\alpha} } _ { 2 } , \dots , \boldsymbol { \boldsymbol{\alpha} } _ { n } $线性表出。因此
\[
{\rm rank}\{\boldsymbol{\varepsilon} _ { 1 } , \boldsymbol{\varepsilon} _ { 2 } , \dots , \boldsymbol{\varepsilon} _ { n }\}< {\rm rank}\{\boldsymbol { \boldsymbol{\alpha} } _ { 1 } , \boldsymbol { \boldsymbol{\alpha} } _ { 2 } , \dots , \boldsymbol { \boldsymbol{\alpha} } _ { n } \}.
\]
由于$\boldsymbol{\varepsilon} _ { 1 } , \boldsymbol{\varepsilon} _ { 2 } , \dots , \boldsymbol{\varepsilon} _ { n }$线性无关,因此${\rm rank}\{\boldsymbol{\varepsilon} _ { 1 } , \boldsymbol{\varepsilon} _ { 2 } , \dots , \boldsymbol{\varepsilon} _ { n }\}=n$。从而${\rm rank}\{\boldsymbol { \boldsymbol{\alpha} } _ { 1 } , \boldsymbol { \boldsymbol{\alpha} } _ { 2 } , \dots , \boldsymbol { \boldsymbol{\alpha} } _ { n }\}=n$。
于是$\boldsymbol { \boldsymbol{\alpha} } _ { 1 } , \boldsymbol { \boldsymbol{\alpha} } _ { 2 } , \dots , \boldsymbol { \boldsymbol{\alpha} } _ { n }$线性无关。
\end{proof}

\item[四、](15分)
设 $A, B, C, D $均为 $n$ 阶方阵,且 $|A| \ne 0, AC = CA$。求证:\[
\left| \begin{matrix}
	A&		B\\
	C&		D\\
\end{matrix} \right|=|AD-CB|.
\]
\begin{proof}
\[
\left| \begin{matrix}
	A&		B\\
	C&		D\\
\end{matrix} \right| = 
\left| \begin{matrix}
	A&		B\\
	0&		D-CA^{-1}B\\
\end{matrix} \right| = 
|A||D-CA^{-1}B|=|AD-CB|.
\]
\end{proof}

\item[五、](15分)
设$A=\left( \begin{smallmatrix}
	2&		1&		0\\
	0&		2&		1\\
	0&		0&		2\\
\end{smallmatrix} \right) $,求证:\[
A^n=\left( \begin{matrix}
	2^n&		2^{n-1}n&		2^{n-3}n\left( n-1 \right)\\
	0&		2^n&		2^{n-1}n\\
	0&		0&		2^n\\
\end{matrix} \right) .
\]

\begin{proof}
因为
\[
A=2I+J,
\]
其中$J=\left( \begin{matrix}
	0&		1&		0\\
	0&		0&		1\\
	0&		0&		0\\
\end{matrix} \right) $。\\
所以
\begin{align*}
A^n&=(2I+J)^n=\sum_{r=0}^n{\binom{n}{r} \left( 2I \right) ^{n-r}J^r}=\binom{n}{0} \left( 2I \right) ^nJ^0+\binom{n}{1} \left( 2I \right) ^{n-1}J^1+\binom{n}{2} \left( 2I \right) ^{n-2}J^2\\
&=\left( \begin{matrix}
	2^n&		2^{n-1}n&		2^{n-3}n\left( n-1 \right)\\
	0&		2^n&		2^{n-1}n\\
	0&		0&		2^n\\
\end{matrix} \right) .
\end{align*}
\end{proof}

\item[六、](15分)
设$A$为正交阵。\\
(1) 求证:对任意的$n$维向量$X$,有$\lVert AX \rVert =\lVert X \rVert $。\\
(2) 若$\lambda$为$A$的一个特征值,求证:$|\lambda|=1$。
\begin{proof}
(1) 对任意的$n$维向量$X$,\[
\lVert AX \rVert =\sqrt{(AX, AX)}=\sqrt{X'A'AX}=\sqrt{X'X}=\sqrt{(X, X)}=\lVert X \rVert 
\]
(2) 取$A$对应于特征值$\lambda$的特征向量$X$,由(1)得\[
|\lambda|\lVert X\rVert=\lVert\lambda X\rVert=\lVert AX \rVert=\lVert X \rVert.
\]
因为$X\ne 0$,所以$|\lambda|=1$。
\end{proof}

\item[七、](15分)
设 $V$ 为 $\mathbb{R}$上的 $3$ 维线性空间,$\boldsymbol { \boldsymbol{\alpha} }_1, \boldsymbol { \boldsymbol{\alpha} }_2, \boldsymbol { \boldsymbol{\alpha} }_3$ 是空间$ V $的一组基,$\boldsymbol { \boldsymbol{\beta} }_1 =\boldsymbol { \boldsymbol{\alpha} }_1 +\boldsymbol { \boldsymbol{\alpha} }_2, \boldsymbol { \boldsymbol{\beta} }_2 = \boldsymbol { \boldsymbol{\alpha} }_2 +\boldsymbol { \boldsymbol{\alpha} }_3, \boldsymbol { \boldsymbol{\beta} }_3 =\boldsymbol { \boldsymbol{\alpha} }_3 + \boldsymbol { \boldsymbol{\alpha} }_1$。\\
(1) 求证:$\boldsymbol { \boldsymbol{\beta} }_1, \boldsymbol { \boldsymbol{\beta} }_2, \boldsymbol { \boldsymbol{\beta} }_3$ 也是空间 $V$ 的基。\\
(2) 求基 $\boldsymbol { \boldsymbol{\beta} }_1, \boldsymbol { \boldsymbol{\beta} }_2, \boldsymbol { \boldsymbol{\beta} }_3$ 到 $\boldsymbol { \boldsymbol{\alpha} }_1, \boldsymbol { \boldsymbol{\alpha} }_2, \boldsymbol { \boldsymbol{\alpha} }_3$ 的过渡矩阵。\\
(3) 求 $\boldsymbol{\boldsymbol{\gamma}} = 3\boldsymbol { \boldsymbol{\alpha} }_1 + \boldsymbol { \boldsymbol{\alpha} }_2 - 4\boldsymbol { \boldsymbol{\alpha} }_3 $在基 $\boldsymbol { \boldsymbol{\beta} }_1, \boldsymbol { \boldsymbol{\beta} }_2, \boldsymbol { \boldsymbol{\beta} }_3$ 下的坐标。
\begin{proof}
(1) 只需证明$\boldsymbol { \boldsymbol{\beta} }_1, \boldsymbol { \boldsymbol{\beta} }_2, \boldsymbol { \boldsymbol{\beta} }_3$是线性无关的。\\
设$k_1\boldsymbol { \boldsymbol{\beta} }_1+k_2\boldsymbol { \boldsymbol{\beta} }_2+k_3\boldsymbol { \boldsymbol{\beta} }_3=0$,则\[
k_1(\boldsymbol { \boldsymbol{\alpha} }_1+\boldsymbol { \boldsymbol{\alpha} }_2)+k_2(\boldsymbol { \boldsymbol{\alpha} }_2+\boldsymbol { \boldsymbol{\alpha} }_3)+k_3(\boldsymbol { \boldsymbol{\alpha} }_3+\boldsymbol { \boldsymbol{\alpha} }_1)=(k_1+k_3)\boldsymbol { \boldsymbol{\alpha} }_1 + (k_1+k_2)\boldsymbol { \boldsymbol{\alpha} }_2 + (k_2+k_3)\boldsymbol { \boldsymbol{\alpha} }_3=0.
\]
因为$\boldsymbol { \boldsymbol{\alpha} }_1, \boldsymbol { \boldsymbol{\alpha} }_2, \boldsymbol { \boldsymbol{\alpha} }_3$ 是空间$ V $的一组基,所以\[
\left\{ \begin{array}{l}
	k_1+k_3=0\\
	k_1+k_2=0\\
	k_2+k_3=0\\
\end{array} \right. 
\]
解得$k_1 = k_2 = k_3 = 0$。所以$\boldsymbol { \boldsymbol{\beta} }_1, \boldsymbol { \boldsymbol{\beta} }_2, \boldsymbol { \boldsymbol{\beta} }_3$是线性无关的。所以$\boldsymbol { \boldsymbol{\beta} }_1, \boldsymbol { \boldsymbol{\beta} }_2, \boldsymbol { \boldsymbol{\beta} }_3$是空间 $V$ 的基。\\
(2) 因为\[
\left( \boldsymbol { \boldsymbol{\beta} }_1, \boldsymbol { \boldsymbol{\beta} }_2, \boldsymbol { \boldsymbol{\beta} }_3 \right) =\left( \boldsymbol { \boldsymbol{\alpha} } _1,\boldsymbol { \boldsymbol{\alpha} } _2,\boldsymbol { \boldsymbol{\alpha} } _3 \right) \left( \begin{matrix}
	1&		0&		1\\
	1&		1&		0\\
	0&		1&		1\\
\end{matrix} \right) ,
\]
所以\[
\left( \boldsymbol { \boldsymbol{\alpha} } _1,\boldsymbol { \boldsymbol{\alpha} } _2,\boldsymbol { \boldsymbol{\alpha} } _3 \right) =\left( \boldsymbol { \boldsymbol{\beta} }_1, \boldsymbol { \boldsymbol{\beta} }_2, \boldsymbol { \boldsymbol{\beta} }_3 \right) \left( \begin{matrix}
	1&		0&		1\\
	1&		1&		0\\
	0&		1&		1\\
\end{matrix} \right) ^{-1}=\left( \boldsymbol { \boldsymbol{\beta} }_1, \boldsymbol { \boldsymbol{\beta} }_2, \boldsymbol { \boldsymbol{\beta} }_3 \right) \left( \begin{matrix}
	\frac{1}{2}&		\frac{1}{2}&		-\frac{1}{2}\\
	-\frac{1}{2}&		\frac{1}{2}&		\frac{1}{2}\\
	\frac{1}{2}&		-\frac{1}{2}&		\frac{1}{2}\\
\end{matrix} \right) .
\]
所以基 $\boldsymbol { \boldsymbol{\beta} }_1, \boldsymbol { \boldsymbol{\beta} }_2, \boldsymbol { \boldsymbol{\beta} }_3$ 到 $\boldsymbol { \boldsymbol{\alpha} }_1, \boldsymbol { \boldsymbol{\alpha} }_2, \boldsymbol { \boldsymbol{\alpha} }_3$ 的过渡矩阵为\[
\left( \begin{matrix}
	\frac{1}{2}&		\frac{1}{2}&		-\frac{1}{2}\\
	-\frac{1}{2}&		\frac{1}{2}&		\frac{1}{2}\\
	\frac{1}{2}&		-\frac{1}{2}&		\frac{1}{2}\\
\end{matrix} \right).
\]
(3) \[
\boldsymbol{\boldsymbol{\gamma}}=\left( \boldsymbol { \boldsymbol{\alpha} } _1,\boldsymbol { \boldsymbol{\alpha} } _2,\boldsymbol { \boldsymbol{\alpha} } _3 \right) \left( \begin{array}{c}
	3\\
	1\\
	-4\\
\end{array} \right) =\left( \boldsymbol { \boldsymbol{\beta} }_1, \boldsymbol { \boldsymbol{\beta} }_2, \boldsymbol { \boldsymbol{\beta} }_3 \right) \left( \begin{matrix}
	\frac{1}{2}&		\frac{1}{2}&		-\frac{1}{2}\\
	-\frac{1}{2}&		\frac{1}{2}&		\frac{1}{2}\\
	\frac{1}{2}&		-\frac{1}{2}&		\frac{1}{2}\\
\end{matrix} \right) \left( \begin{array}{c}
	3\\
	1\\
	-4\\
\end{array} \right) =\left( \boldsymbol { \boldsymbol{\beta} }_1, \boldsymbol { \boldsymbol{\beta} }_2, \boldsymbol { \boldsymbol{\beta} }_3 \right) \left( \begin{array}{c}
	4\\
	-3\\
	-1\\
\end{array} \right) .
\]
所以$\boldsymbol{\boldsymbol{\gamma}}$在基 $\boldsymbol { \boldsymbol{\beta} }_1, \boldsymbol { \boldsymbol{\beta} }_2, \boldsymbol { \boldsymbol{\beta} }_3$ 下的坐标为$(4, -3, -1)'$。
\end{proof}

\item[八、]
(15 分)用正交线性替换化二次型为标准型,并判断\[
f(x_1, x_2, x_3) = 1
\]
为何种几何曲面?其中 $f(x_1, x_2, x_3) = 5x^2_1 + 5x^2_2 + 3x^2_3 - 2x_1x_2 + 6x_1x_3 - 6x_2x_3$。
\begin{proof}
二次型$f(x_1, x_2, x_3)$的矩阵为\[A=
\left( \begin{matrix}
	5&		-1&		3\\
	-1&		5&		-3\\
	3&		-3&		3\\
\end{matrix} \right) .
\]
\[
|\lambda I-A|=\left| \begin{matrix}
	\lambda -5&		1&		-3\\
	1&		\lambda -5&		3\\
	-3&		3&		\lambda -3\\
\end{matrix} \right|=\lambda \left( \lambda -4 \right) \left( \lambda -9 \right) .
\]
所以$A$的全部特征值为$0$,$4$,$9$。\\
解线性方程组$(0I-A)X=0$,得一个基础解系:\[
\boldsymbol{\xi}_1=(-1, 1, 2)'.
\]
解线性方程组$(4I-A)X=0$,得一个基础解系:\[
\boldsymbol{\xi}_2=(1, 1, 0)'.
\]
解线性方程组$(9I-A)X=0$,得一个基础解系:\[
\boldsymbol{\xi}_3=(7, -1, 1)'.
\]
将$\boldsymbol{\xi}_1, \boldsymbol{\xi}_2, \boldsymbol{\xi}_3$分别单位化得:\begin{align*}
\eta_1&=\frac{\boldsymbol{\xi}_1}{|\boldsymbol{\xi}_1|}=\left(\frac{-\sqrt{6}}{6}, \frac{\sqrt{6}}{6}, \frac{\sqrt{6}}{3}\right)';\\
\eta_2&=\frac{\boldsymbol{\xi}_2}{|\boldsymbol{\xi}_2|}=\left(\frac{\sqrt{2}}{2}, \frac{\sqrt{2}}{2}, 0\right)';\\
\eta_3&=\frac{\boldsymbol{\xi}_3}{|\boldsymbol{\xi}_3|}=\left(\frac{7\sqrt{51}}{51}, \frac{-\sqrt{51}}{51}, \frac{\sqrt{51}}{51}\right)'.
\end{align*}
令\[
T=\left( \eta _1,\eta _2,\eta _3 \right) =\left( \begin{matrix}
	-\frac{\sqrt{6}}{6}&		\frac{\sqrt{2}}{2}&		\frac{7\sqrt{51}}{51}\\
	\frac{\sqrt{6}}{6}&		\frac{\sqrt{2}}{2}&		-\frac{\sqrt{51}}{51}\\
	\frac{\sqrt{6}}{3}&		0&		\frac{\sqrt{51}}{51}\\
\end{matrix} \right),
\]
则$T$是正交阵,且$T^{-1}AT={\rm diag}\{0, 4, 9\}$。\\
令\begin{equation}\label{transformation}
\left( \begin{array}{c}
	x_1^*\\
	x_2^*\\
	x_3^*\\
\end{array} \right) =T\left( \begin{array}{c}
	x_1\\
	x_2\\
	x_3\\
\end{array} \right) .
\end{equation}
则\[
f(x_1, x_2, x_3)=4{x_2^*}^2+9{x_3^*}^2.
\]
下面判断曲面类型。作直角坐标变换  \eqref{transformation},则原二次曲面 $5 x _ { 1 } ^ { 2 } + 5 x _ { 2 } ^ { 2 } + 3 x _ { 3 } ^ { 2 } - 2 x _ { 1 } x _ { 2 } + 6 x _ { 1 } x _ { 3 } - 6 x _ { 2 } x _ { 3 }=1$在新的直角坐标系中的方程为:\[
4{x_2^*}^2+9{x_3^*}^2=1.
\]
由此看出,这是椭圆柱面。
\end{proof}

\item[九、]
(15 分)设 $f(x)$ 是一个整系数多项式,$f(0)$ 与 $f(1)$ 均为奇数,求证:$f(x)$ 没有整数根。
\begin{proof}
假如 $f(x)$ 有一个整数根$b$,则$x-b$是本原多项式,且$x-b$是$f(x)$的一个因式。又由于$f(x)$也是本原多项式,因此存在整系数多项式$h(x)$,使得\[
f(x)=(x-b)h(x).
\]
$x$分别用$0$和$1$代入,从上式得\[
f(0)=(-b)h(0), (1)=(1-b)h(1).\]
由于$-b$和$-b+1$必有一个是偶数,因此$f(0)$和$f(1)$必有一个是偶数。这与已知条件矛盾,所以$f(x)$没有整数根。
\end{proof}

\item[十、]
(10 分)设 $V$ 为 $\mathbb{R}$ 上的 $n$ 维线性空间,$\mathscr{A}$ 为 $V$ 上的一个线性变换,$\Im(\mathscr{A})$ 与$\ker\mathscr{A}$分别为线性变换$\mathscr{A}$的值域和核空间,求证:\[
\Im\mathscr{A}\oplus\ker\mathscr{A}= V
\]
的充分必要条件为:\[
\ker\mathscr{A} = \ker\mathscr{A}^2.
\]
\begin{proof}
必要性。易知$\ker\mathscr{A}\subset\ker\mathscr{A}^2$,下证$\ker\mathscr{A}^2\subset\ker\mathscr{A}$。\\
任取向量$\boldsymbol{\alpha} \in \ker\mathscr{A}^2$,有
$$\mathscr{A}^2\boldsymbol{\alpha}=0,$$
因此\[
\mathscr{A}\boldsymbol{\alpha}\in\ker\mathscr{A}.
\]
又\[
\mathscr{A}\boldsymbol{\alpha}\in\Im\mathscr{A},
\]
且$\Im\mathscr{A}+\ker\mathscr{A}$是直和,所以
$$\mathscr{A}\boldsymbol{\alpha}=0,$$
即\[
\boldsymbol{\alpha}\in\ker\mathscr{A}.
\]
故$\ker\mathscr{A}^2\subset\ker\mathscr{A}$。\\
充分性。只需证$\text{Im}\mathscr{A}+\ker\mathscr{A}$是直和。\\
任取一个向量$\boldsymbol{\alpha}\in \Im\mathscr{A}\bigcap\ker\mathscr{A}$,则$\mathscr{A}\boldsymbol{\alpha}=0$,且存在一个向量$\boldsymbol{\beta}\in V$,使得$\boldsymbol{\alpha}=\mathscr{A}\boldsymbol{\beta}$。因此$\mathscr{A}^2\boldsymbol{\beta}=0$,即$\boldsymbol{\beta}\in\ker\mathscr{A}^2$。因为$\ker\mathscr{A}=\ker\mathscr{A}^2$,所以$\boldsymbol{\beta}\in\ker\mathscr{A}$,即$\mathscr{A}\boldsymbol{\beta}=0$。因此$\boldsymbol{\alpha}=0$。故$\text{Im}\mathscr{A}+\ker\mathscr{A}$是直和。

\end{proof}
\end{enumerate}
\endinput