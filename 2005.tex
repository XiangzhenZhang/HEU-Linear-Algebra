\section{2005}
\begin{enumerate}[1~]
\renewcommand{\labelenumi}{\textbf{\theenumi. }}
\renewcommand{\Im}{\text{Im }}
\renewcommand{\u}{\boldsymbol{u}}
\item[一、]
填空题 (每小题 4 分,共 20 分)
\begin{enumerate}[1.~]
\item
若$\mathbb{P}$为同时包含$\mathbb{Q}$和$\sqrt[3]{2}$的最小数域,则$\mathbb{P}$作为$\mathbb{Q}$上的线性空间的维数是(\quad)。
\begin{solution}
\begin{theorem}
设$\mathbb{K}/\mathbb{F}$是域扩张,$\boldsymbol{\alpha} \in \mathbb{K}$且$\boldsymbol{\alpha}$是$\mathbb{F}$上的代数元。如果$\boldsymbol{\alpha}$在$\mathbb{F}$上的极小多项式$m(x)$的次数为$n$,那么\[
[\mathbb{F}(\boldsymbol{\alpha}):\mathbb{F}]=n,
\]
并且$1, \boldsymbol{\alpha}, \boldsymbol{\alpha}^2, \dots, \boldsymbol{\alpha}^{n-1}$是$\mathbb{F}(\boldsymbol{\alpha})/\mathbb{F}$的一个基。
\end{theorem}
\begin{proof}
假如$1, \boldsymbol{\alpha}, \boldsymbol{\alpha}^2, \dots, \boldsymbol{\alpha}^{n-1}$在$\mathbb{F}$上线性相关,则有$\mathbb{F}$中不全为0的元素$b_0, b_1, b_2, \dots, b_{n-1}$,使得\[
b_0+b_1\boldsymbol{\alpha}+b_2\boldsymbol{\alpha}^2+\cdots+b_{n-1}\boldsymbol{\alpha}^{n-1}=0.
\]
令$g(x)=b_0+b_1x+b_2x^2+\cdots+b_{n-1}x^{n-1}$,则$g(\boldsymbol{\alpha})=0$,且$g(x)\ne 0$。这与$m(x)$是$\boldsymbol{\alpha}$在$\mathbb{F}$上的极小多项式矛盾。因此$1, \boldsymbol{\alpha}, \boldsymbol{\alpha}^2, \dots, \boldsymbol{\alpha}^{n-1}$在$\mathbb{F}$上线性无关。\\
又由于\[
\mathbb{F}(\boldsymbol{\alpha})=\mathbb{F}[\boldsymbol{\alpha}]=\{c_0+c_1\boldsymbol{\alpha}+c_2\boldsymbol{\alpha}^2+\cdots+c_{n-1}\boldsymbol{\alpha}^{n-1}|c_i\in\mathbb{F}, i=0, 1, 2, \dots, n-1\},
\]
因此$1, \boldsymbol{\alpha}, \boldsymbol{\alpha}^2, \dots, \boldsymbol{\alpha}^{n-1}$是$\mathbb{F}(\boldsymbol{\alpha})$的一个基。从而\[
[\mathbb{F}(\boldsymbol{\alpha}):\mathbb{F}]=n.
\]
\end{proof}
因为$\mathbb{P}$为同时包含$\mathbb{Q}$和$\sqrt[3]{2}$的最小数域,所以$\mathbb{P}=\mathbb{Q}(\sqrt[3]{2})=\{a+b\sqrt[3]{2}|a, b\in\mathbb{Q}\}$。在定理中令$\mathbb{K}=\mathbb{Q}(\sqrt[3]{2})$,$\mathbb{F}=\mathbb{Q}$,$\boldsymbol{\alpha}=\sqrt[3]{2}$,则由$
\sqrt[3]{2}\text{在$\mathbb{Q}$上的极小多项式为}x^3-2
$
得\[
[\mathbb{Q}(\sqrt[3]{2}):\mathbb{Q}]=3.
\]
所以$\mathbb{P}$作为$\mathbb{Q}$上的线性空间的维数是3。(相关知识可看\cite{qiujin})
\end{solution}

\item
多项式$x ^ { 7 } + 2 x ^ { 6 } + 6 x ^ { 2 } + 2$在复数域内所有根之间的关系是(\quad)。
\begin{solution}
设\[
f(x)=x ^ { 7 } + 2 x ^ { 6 } + 6 x ^ { 2 } + 2,
\]
则\[
f^{\prime} (x)=7x^6+12x^5+12x.
\]
显然$f^{\prime}(x)\nmid f(x)$,所以$(f'(x), f(x))=1$,因此$f(x)$无重根,故多项式$x ^ { 7 } + 2 x ^ { 6 } + 6 x ^ { 2 } + 2$在复数域内的所有根互不相等。
\end{solution}

\item
行列式 
$$
\left| \begin{array} { c c c c } { 1 + a } & { 1 } & { 1 } & { 1 } \\ { 1 } & { 1 + b } & { 1 } & { 1 } \\ { 1 } & { 1 } & { 1 + c } & { 1 } \\ { 1 } & { 1 } & { 1 } & { 1 + d } \end{array} \right|
$$
的值为(\quad)。
\begin{solution}
%拆。
%\begin{align*}
%\left| \begin{array} { c c c c } { 1 + a } & { 1 } & { 1 } & { 1 } \\ { 1 } & { 1 + b } & { 1 } & { 1 } \\ { 1 } & { 1 } & { 1 + c } & { 1 } \\ { 1 } & { 1 } & { 1 } & { 1 + d } \end{array} \right|
%&=\left| \begin{array} { c c c c } { 1  } & { 1 } & { 1 } & { 1 } \\ { 1 } & { 1 + b } & { 1 } & { 1 } \\ { 1 } & { 1 } & { 1 + c } & { 1 } \\ { 1 } & { 1 } & { 1 } & { 1 + d } \end{array} \right|+\left| \begin{array} { c c c c } {  a } & { 1 } & { 1 } & { 1 } \\ { 1 } & { 1 + b } & { 1 } & { 1 } \\ { 1 } & { 1 } & { 1 + c } & { 1 } \\ { 1 } & { 1 } & { 1 } & { 1 + d } \end{array} \right|\\
%&=\underset{\text{有2列全为1或有3列全为1或有4列全为}1}{\underbrace{0}}\\
%&+\left( \left| \begin{matrix}
%	1&		1&		1&		1\\
%	1&		b&		1&		1\\
%	1&		1&		c&		1\\
%	1&		1&		1&		d\\
%\end{matrix} \right|+\left| \begin{matrix}
%	a&		1&		1&		1\\
%	1&		1&		1&		1\\
%	1&		1&		c&		1\\
%	1&		1&		1&		d\\
%\end{matrix} \right|+\left| \begin{matrix}
%	a&		1&		1&		1\\
%	1&		b&		1&		1\\
%	1&		1&		1&		1\\
%	1&		1&		1&		d\\
%\end{matrix} \right|+\left| \begin{matrix}
%	a&		1&		1&		1\\
%	1&		b&		1&		1\\
%	1&		1&		c&		1\\
%	1&		1&		1&		1\\
%\end{matrix} \right| \right) +\left| \begin{matrix}
%	a&		1&		1&		1\\
%	1&		b&		1&		1\\
%	1&		1&		c&		1\\
%	1&		1&		1&		d\\
%\end{matrix} \right|\\
%&=0+((b-1)(c-1)(d-1)-(a-1)(c-1)(d-1)+(a-1)(b-1)(d-1)-(a-1)(b-1)(c-1))+
%\end{align*}
若$a=0$,易得行列式值为$bcd$;\\
若$b=0$,易得行列式值为$acd$;\\
若$c=0$,易得行列式值为$abd$;\\
若$d=0$,易得行列式值为$abc$;\\
若$a, b, c, d$都不为0,则
\begin{align*}
&\left| \begin{array} { c c c c } { 1 + a } & { 1 } & { 1 } & { 1 } \\ { 1 } & { 1 + b } & { 1 } & { 1 } \\ { 1 } & { 1 } & { 1 + c } & { 1 } \\ { 1 } & { 1 } & { 1 } & { 1 + d } \end{array} \right|
=\left| \begin{matrix}
	1+a&		1&		1&		1\\
	1&		1+b&		1&		1\\
	1&		1&		1+c&		1\\
	1&		1&		1&		1+d\\
\end{matrix} \right|\xlongequal{\text{将第一行的}\left( -1 \right) \text{倍加到其余各行}}{}\\
&\left| \begin{matrix}
	1+a&		1&		1&		1\\
	-a&		b&		0&		0\\
	-a&		0&		c&		0\\
	-a&		0&		0&		d\\
\end{matrix} \right|\xlongequal[\text{将第二列的}\left( \frac{a}{b} \right) \text{倍加到第一列,将第三列的}\left( \frac{a}{c} \right) \text{倍加到第一列,将第四列的}\left( \frac{a}{d} \right) \text{倍加到第一列}]{}\\
&\left| \begin{matrix}
	1+a+\frac{a}{b}+\frac{a}{c}+\frac{a}{d}&		1&		1&		1\\
	0&		b&		0&		0\\
	0&		0&		c&		0\\
	0&		0&		0&		d\\
\end{matrix} \right|=abcd+abc+abd+acd+bcd.
\end{align*}
当$a=0$或$b=0$或$c=0$或$d=0$时,上式依然成立,因此行列式为$abcd+abc+abd+acd+bcd$。
\end{solution}

\item
设$n$阶方阵$A$的秩为${r}$,且满足$A^2=A$,则$|2E-A|=$(\quad)。
\begin{solution}
因为$A$是幂等矩阵,${\rm rank}(A)=r$,所以存在$n$阶可逆矩阵$P$,使得\[
PAP^{-1}=\left( \begin{matrix}
	E_r&		\\
	&		0\\
\end{matrix} \right):=\Lambda.
\]
所以\[
|2E-A|=|2E-\Lambda|=2^{n-r}.
\]
\end{solution}

\item
向量组$\boldsymbol{\alpha} _ { 1 } ,  \boldsymbol{\alpha} _ { 2 } ,  \boldsymbol{\alpha} _ { 3 } ,  \boldsymbol{\alpha} _ { 4 } \in \mathbb{R} ^ { 4 }$线性无关,则向量组$\boldsymbol{\alpha} _ { 1 } + \boldsymbol{\alpha} _ { 2 } ,  \boldsymbol{\alpha} _ { 2 } + \boldsymbol{\alpha} _ { 3 } ,  \boldsymbol{\alpha} _ { 3 } + \boldsymbol{\alpha} _ { 4 } ,  \boldsymbol{\alpha} _ { 4 } + \boldsymbol{\alpha} _ { 1 }$的
线性相关性为(\quad)。
\begin{solution}
因为\[
(\boldsymbol{\alpha} _ { 1 } + \boldsymbol{\alpha} _ { 2 } , \boldsymbol{\alpha} _ { 2 } + \boldsymbol{\alpha} _ { 3 } , \boldsymbol{\alpha} _ { 3 } + \boldsymbol{\alpha} _ { 4 } ,  \boldsymbol{\alpha} _ { 4 } + \boldsymbol{\alpha} _ { 1 })=(\boldsymbol{\alpha} _ { 1 } , \boldsymbol{\alpha} _ { 2 } , \boldsymbol{\alpha} _ { 3 } , \boldsymbol{\alpha} _ { 4 })\left( \begin{matrix}
	1&		&		&		1\\
	1&		1&		&		\\
	&		1&		1&		\\
	&		&		1&		1\\
\end{matrix} \right) ,
\]
\[
\left| \begin{matrix}
	1&		&		&		1\\
	1&		1&		&		\\
	&		1&		1&		\\
	&		&		1&		1\\
\end{matrix} \right|=0.
\]
所以,向量组$\boldsymbol{\alpha} _ { 1 } + \boldsymbol{\alpha} _ { 2 } ,  \boldsymbol{\alpha} _ { 2 } + \boldsymbol{\alpha} _ { 3 } ,  \boldsymbol{\alpha} _ { 3 } + \boldsymbol{\alpha} _ { 4 } ,  \boldsymbol{\alpha} _ { 4 } + \boldsymbol{\alpha} _ { 1 }$线性相关。
\end{solution}

\item
设$A$、$B$为$m\times n$矩阵,则两个齐次线性方程组$AX =0$和$BX=0$同解的充分必要条件为$ r(A)=r(B)=$(\quad)。
\begin{solution}
显然,\[
r(A)=r(B)=r\left(\left[ \begin{array}{c}
	A\\
	B\\
\end{array} \right] \right).
\]
\end{solution}

\item
设$\mathscr{A}$为$3$维线性空间$V$中的线性变换,则$\dim \mathscr{A}(V)$与$\dim\Ker\mathscr{A}$的关系是(\quad)。
\begin{solution}
因为$V$是有限维的线性空间,所以$\mathscr{A}(V)$和$\Ker\mathscr{A}$都是有限维的。由于\[
V/\Ker\mathscr{A}\cong\mathscr{A}(V).
\]
所以$\mathscr{A}(V)$是有限维的,且\[
\dim V-\dim\Ker\mathscr{A}=\dim\mathscr{A}V.
\]
因此,$\dim\mathscr{A}(V)+\dim\Ker\mathscr{A}=3$。
\end{solution}

\item
设$A$为$n$阶方阵,则$A$与$A^T$的关系是(\quad)。
\begin{solution}
\begin{lemma}
设$J_n(a)$是对角元素为$a$的$n$级Jordan块,则$J_n(a)\sim J_n(a)'$。
\end{lemma}
\begin{subproof}
由于\[
\left( \begin{array} { c c c c c } { 0 } & { 0 } & { \cdots } & { 0 } & { 1 } \\ { 0 } & { 0 } & { \cdots } & { 1 } & { 0 } \\ { \vdots } & { \vdots } & { } & { \vdots } & { \vdots } \\ { 0 } & { 1 } & { \cdots } & { 0 } & { 0 } \\ { 1 } & { 0 } & { \cdots } & { 0 } & { 0 } \end{array} \right) \left( \begin{array} { c } { \boldsymbol{\gamma} _ { 1 } } \\ { \boldsymbol{\gamma} _ { 2 } } \\ { \vdots } \\ { \boldsymbol{\gamma} _ { n } } \end{array} \right) = \left( \begin{array} { c } { \boldsymbol{\gamma} _ { n } } \\ { \boldsymbol{\gamma} _ { n - 1 } } \\ { \vdots } \\ { \boldsymbol{\gamma} _ { 2 } } \\ { \boldsymbol{\gamma} _ { 1 } } \end{array} \right),
\]
\[
\left( \boldsymbol { \boldsymbol{\alpha} } _ { 1 } , \boldsymbol { \boldsymbol{\alpha} } _ { 2 } , \dots , \boldsymbol { \boldsymbol{\alpha} } _ { n } \right) \left( \begin{array} { c c c c c } { 0 } & { 0 } & { \cdots } & { 0 } & { 1 } \\ { 0 } & { 0 } & { \cdots } & { 1 } & { 0 } \\ { \vdots } & { \vdots } & { } & { \vdots } & { \vdots } \\ { 0 } & { 1 } & { \cdots } & { 0 } & { 0 } \\ { 1 } & { 0 } & { \cdots } & { 0 } & { 0 } \end{array} \right) = \left( \boldsymbol { \boldsymbol{\alpha} } _ { n } , \boldsymbol { \boldsymbol{\alpha} } _ { n - 1 } , \dots , \boldsymbol { \boldsymbol{\alpha} } _ { 2 } , \boldsymbol { \boldsymbol{\alpha} } _ { 1 } \right),
\]
因此\[
\left( \begin{array} { c c c c c } { 0 } & { 0 } & { \cdots } & { 0 } & { 1 } \\ { 0 } & { 0 } & { \cdots } & { 1 } & { 0 } \\ { \vdots } & { \vdots } & { } & { \vdots } & { \vdots } \\ { 0 } & { 1 } & { \cdots } & { 0 } & { 0 } \\ { 1 } & { 0 } & { \cdots } & { 0 } & { 0 } \end{array} \right) \left( \begin{array} { c c c c c } { 0 } & { 0 } & { \cdots } & { 0 } & { 1 } \\ { 0 } & { 0 } & { \cdots } & { 1 } & { 0 } \\ { \vdots } & { \vdots } & { } & { \vdots } & { \vdots } \\ { 0 } & { 1 } & { \cdots } & { 0 } & { 0 } \\ { 1 } & { 0 } & { \cdots } & { 0 } & { 0 } \end{array} \right) = \left( \begin{array} { c c c c c } { 1 } & { 0 } & { \cdots } & { 0 } & { 0 } \\ { 0 } & { 1 } & { \cdots } & { 0 } & { 0 } \\ { \vdots } & { \vdots } & { } & { \vdots } & { \vdots } \\ { 0 } & { 0 } & { \cdots } & { 1 } & { 0 } \\ { 0 } & { 0 } & { \cdots } & { 0 } & { 1 } \end{array} \right) = I
\]
\[
\left( \begin{array} { c c c c c } { 0 } & { 0 } & { \cdots } & { 0 } & { 1 } \\ { 0 } & { 0 } & { \cdots } & { 1 } & { 0 } \\ { \vdots } & { \vdots } & { } & { \vdots } & { \vdots } \\ { 0 } & { 1 } & { \cdots } & { 0 } & { 0 } \\ { 1 } & { 0 } & { \cdots } & { 0 } & { 0 } \end{array} \right)^{-1}
\left( \begin{array} { c c c c c c } { a } & { 1 } &0& { \cdots } & { 0 } & { 0 } \\ { 0 } & { a } &1& { \cdots } & { 0 } & { 0 } \\ { \vdots } & { \vdots } &{ \vdots } & { } & { \vdots } & { \vdots } \\ { 0 } & { 0 } &0& { \cdots } & { a } & { 1 } \\ { 0 } & { 0 } &0& { \cdots } & { 0 } & { a } \end{array} \right)
= \left( \begin{array} { c c c c c } { 0 } & { 0 } & { \cdots } & { 0 } & { 1 } \\ { 0 } & { 0 } & { \cdots } & { 1 } & { 0 } \\ { \vdots } & { \vdots } & { } & { \vdots } & { \vdots } \\ { 0 } & { 1 } & { \cdots } & { 0 } & { 0 } \\ { 1 } & { 0 } & { \cdots } & { 0 } & { 0 } \end{array} \right) 
\]
\[
= \left( \begin{array} { c c c c c } { 0 } & { 0 } & { \cdots } & { 0 } & { 1 } \\ { 0 } & { 0 } & { \cdots } & { 1 } & { 0 } \\ { \vdots } & { \vdots } & { } & { \vdots } & { \vdots } \\ { 0 } & { 1 } & { \cdots } & { 0 } & { 0 } \\ { 1 } & { 0 } & { \cdots } & { 0 } & { 0 } \end{array} \right) \left( \begin{array} { c c c c c c } { 0 } & { 0 } & { \cdots } & { 0 } & { 1 } & { a } \\ { 0 } & { 0 } & { \cdots } & { 1 } & { a } & { 0 } \\ { \vdots } & { \vdots } & { } & { \vdots } & { \vdots } \\ { 1 } & { a } & { \cdots } & { 0 } & { 0 } & { 0 } \\ { a } & { 0 } & { \cdots } & { 0 } & { 0 } & { 0 } \end{array} \right) = \left( \begin{array} { c c c c c c } { a } & { 0 } & { \cdots } & { 0 } & { 0 } & { 0 } \\ { 1 } & { a } & { \cdots } & { 0 } & { 0 } & { 0 } \\ { \vdots } & { \vdots } & { } & { \vdots } & { \vdots } & { \vdots } \\ { 0 } & { 0 } & { \cdots } & { 1 } & { a } & { 0 } \\ { 0 } & { 0 } & { \cdots } & { 0 } & { 1 } & { a } \end{array} \right).
\]
从而$J_n(a)\sim J_n(a)'$。
\end{subproof}
\begin{lemma}
任一 $n$ 级复矩阵 $A$ 与 $A'$ 相似。
\end{lemma}
\begin{subproof}
由于$A$是复数域上的矩阵,因此$A$有Jordan 标准形\[
J = 
\operatorname { diag } \left\{ J _ { r _ { 1 } } \left( 
\lambda _ { 1 } \right) , J _ { r _ { 2 } } \left( \lambda _ { 
2 } \right) , \cdots , J _ { r _ { m } } \left( \lambda _ { m 
} \right) \right\},
\]
其中$\lambda _ { 1 } , \lambda _ { 2 } , 
\dots , \lambda _ { m }$可能有相同的,$r _ { 1 } + r _ { 2 
} + \dots + r _ { m } = n $。据引理1得,$J _ { r _ { i } } 
\left( \lambda _ { i } \right) \sim J _ { r _ { i } } \left( 
\lambda _ { i } \right) ^ { \prime }$,且可逆矩阵$P _ { i } 
= \left\{ \boldsymbol{\varepsilon} _ { r _ { i } } , \boldsymbol{\varepsilon} _ { r _ { i 
} } - 1 , \dots , \boldsymbol{\varepsilon} _ { 2 } , \boldsymbol{\varepsilon} _ { 1 } 
\right\}$使得$P _ { i } ^ { - 1 } J _ { r _ { i } } \left( 
\lambda _ { i } \right) P _ { i } = J _ { r _ { i } } \left( 
\lambda _ { i } \right) ^ { \prime }$。令$P = 
\operatorname { diag } \left\{ P _ { 1 } , P _ { 2 } , \dots 
, P _ { m } \right\}$,则$P ^ { - 1 } J P = J ^ { \prime }
$。从而$J \sim J ^ { \prime }$。由于$A\sim J$,因此
$A'\sim J'$。从而$A\sim A'$。
\end{subproof}
\begin{lemma}
数域$\mathbb{K}$上$n$级矩阵$A$与$B$相似当且仅当把它们看成复矩阵后相似。
\end{lemma}
\begin{subproof}
设数域$\mathbb{K}$上$n$级矩阵$A$的不变因子为$d _ { 1 } ( \lambda ) , d _ { 2 } ( \lambda ) , \dots , d _ { n } ( \lambda )$。把$A$看成复矩阵后,不变因子仍然是$d _ { 1 } ( \lambda ) , d _ { 2 } ( \lambda ) , \dots , d _ { n } ( \lambda )$。因此
\begin{align*}
&\text{数域$\mathbb{K}$上的$n$级矩阵$A$与$B$相似}\\
\Longleftrightarrow &\text{$A$与$B$有相同的不变因子}\\
\Longleftrightarrow &\text{把$A$与$B$看成复矩阵后相似。}
\end{align*}
\end{subproof}
由引理知,$A$与$A^T$相似。
\end{solution}

\item
设$A$为$n$阶实反对称阵,$\boldsymbol{x}$是非零的$n$维列向量,则$\boldsymbol{x}^TA\boldsymbol{x}$为(\quad)。
\begin{solution}
因为\[
\boldsymbol{x}^TA\boldsymbol{x}=(\boldsymbol{x}^TA\boldsymbol{x})^T=-\boldsymbol{x}^TA\boldsymbol{x}.
\]
所以$\boldsymbol{x}^TA\boldsymbol{x}=0$。
\end{solution}

\item
$A$为$n$阶是正交阵,$\lambda$为其特征值,则$\left| \lambda ^ { - 1 } E - A \right| =$(\quad)。
\begin{solution}
\begin{align*}
|\lambda^{-1}E-A|&=|\lambda^{-1}AA^T-A|=\frac{1}{\lambda^n}|AA^T-\lambda A|=\frac{1}{\lambda^n}|A||A^T-\lambda E|\\
&=\frac{(-1)^n}{\lambda^n}|\lambda E-A^T|=0.
\end{align*}
\end{solution}
\end{enumerate}

\item[二、](15分)
在$\mathbb{R}^3$中,线性变换$\mathscr{A}$定义为
$$
\left\{ \begin{array} { l } { A \boldsymbol{\beta} _ { 1 } = ( 1,0,0 ) } \\ { A \boldsymbol{\beta} _ { 2 } = ( 3,3,2 ) } \\ { A \boldsymbol{\beta} _ { 3 } = ( 3,3,1 ) } \end{array} \right.,
$$
其中 
$$
\left\{ \begin{array} { l } { \boldsymbol{\beta} _ { 1 } = ( 1,0,0 ) } \\ { \boldsymbol{\beta} _ { 2 } = ( 1,1,0 ) } \\ { \boldsymbol{\beta} _ { 3 } = ( 1,1,1 ) } \end{array} \right..
$$

(1) 求 $\mathscr{A}$ 在基 $\boldsymbol{\beta} _ { 1 } ,  \boldsymbol{\beta} _ { 2 } ,  \boldsymbol{\beta} _ { 3 }$ 下的矩阵 $B$;

(2) 求 $\mathscr{A}$ 的特征值与特征向量。

\begin{solution}
(1) 因为\[
\mathscr{A}\left( \begin{array}{c}
	\boldsymbol{\beta} _1\\
	\boldsymbol{\beta} _2\\
	\boldsymbol{\beta} _3\\
\end{array} \right) =\left( \begin{array}{c}
	\mathscr{A}\boldsymbol{\beta} _1\\
	\mathscr{A}\boldsymbol{\beta} _2\\
	\mathscr{A}\boldsymbol{\beta} _3\\
\end{array} \right) =B\left( \begin{array}{c}
	\boldsymbol{\beta} _1\\
	\boldsymbol{\beta} _2\\
	\boldsymbol{\beta} _3\\
\end{array} \right) .
\]
所以\[
B=\left( \begin{array}{c}
	\mathscr{A}\boldsymbol{\beta} _1\\
	\mathscr{A}\boldsymbol{\beta} _2\\
	\mathscr{A}\boldsymbol{\beta} _3\\
\end{array} \right) ^{-1}\left( \begin{array}{c}
	\boldsymbol{\beta} _1\\
	\boldsymbol{\beta} _2\\
	\boldsymbol{\beta} _3\\
\end{array} \right) =\left( \begin{matrix}
	1&		0&		0\\
	3&		3&		2\\
	3&		3&		1\\
\end{matrix} \right) \left( \begin{matrix}
	1&		0&		0\\
	1&		1&		0\\
	1&		1&		1\\
\end{matrix} \right)^{-1} =\left(
\begin{array}{ccc}
 1 & 0 & 0 \\
 0 & 1 & 2 \\
 0 & 2 & 1 \\
\end{array}
\right).
\]
(2) $B$的特征多项式为\[
f(\lambda)=|\lambda I-B|=(x+1)(x-1)(x-3).
\]
所以$B$的特征值为$-1$,$1$,3。\\
解线性方程组$(-I-A)x=0$,得一个基础解系:\[
\boldsymbol{\xi}_1=(0, -1, 1)'.\]
解线性方程组$(\frac13I-A)x=0$,得一个基础解系:\[
\boldsymbol{\xi}_2=(1, 0, 0)'.\]
解线性方程组$(-I-A)x=0$,得一个基础解系:\[
\boldsymbol{\xi}_3=(0, 1, 1)'.\]
因此,$\mathscr{A}$的特征值为$-1$,$1$,3。$\mathscr{A}$的属于特征值$-1$的特征向量为 $k(0, -1, 1)', k\in\mathbb{R}, k\ne 0$,$\mathscr{A}$的属于特征值 $1$ 的特征向量为$l(1, 0, 0)', l\in\mathbb{R}, l\ne 0$,$\mathscr{A}$ 的属于特征值 $3$ 的特征向量为 $m(0, 1, 1)', m\in\mathbb{R}, m\ne 0$。
\end{solution}

\item[三、](15分)
设 $\mathscr{A}$ 为数域 $\mathbb{K}$ 上的 2 维线性空间$V$上的非零的幂零线性变换,求证在 $V$ 的某个基下 $\mathscr{A}$ 的矩阵为 $\left( \begin{smallmatrix} {0} & { 0 } \\ { 1 } & { 0 } \end{smallmatrix} \right)$。
\begin{proof}
任取 $\boldsymbol{\alpha} \in V$,先证 $\boldsymbol{\alpha}, \mathscr{A}\boldsymbol{\alpha}$ 是线性无关的。

设\[
k_1\boldsymbol{\alpha}+k_2\mathscr{A}\boldsymbol{\alpha}=0, k_1, k_2\in \mathbb{K}.
\]
因为 $\mathscr{A}$ 是幂零线性变换,所以 $\mathscr{A}^2=0$,所以\[
k_1\mathscr{A}\boldsymbol{\alpha}=0.
\]
所以$k_1=0$,从而$k_2=0$,所以$\boldsymbol{\alpha}, \mathscr{A}\boldsymbol{\alpha}$是线性 无关的。于是$\boldsymbol{\alpha}, \mathscr{A}\boldsymbol{\alpha}$是$V$的一个基。\\
因为\[
\mathscr{A}(\boldsymbol{\alpha}, \mathscr{A}\boldsymbol{\alpha})=(\boldsymbol{\alpha}, \mathscr{A}\boldsymbol{\alpha})\left( \begin{array} { l l } { 0 } & { 0 } \\ { 1 } & { 0 } \end{array} \right).
\]
所以$\mathscr{A}$在$\boldsymbol{\alpha}, \mathscr{A}\boldsymbol{\alpha}$下的矩阵为$\left( \begin{smallmatrix} { 0 } & { 0 } \\ { 1 } & { 0 } \end{smallmatrix} \right)$。
\end{proof}

\item[四、](15分)
用二次型的理论求三元实函数$f ( x , y , z ) = 2 x ^ { 2 } + 2 x y + 2 x z + 2 y ^ { 2 } + 2 y z + 2 z ^ { 2 }$在单位球面上的最大值和最小值,并求最大值点和最小值点。
\begin{solution}
先用正交变换化二次型为标准型。\\
二次型的矩阵为\[
A=\left( \begin{matrix}
	2&		1&		1\\
	1&		2&		1\\
	1&		1&		2\\
\end{matrix} \right).
\]
矩阵$A$的特征多项式为\[
f(\lambda)=|\lambda E-A|=(x-1)^2(x-4).
\]
所以矩阵$A$的特征值为 $1$($2$ 重),$4$($1$ 重)。\\
解线性方程组$(E-A)X=0$,得一个基础解系\[
\boldsymbol{\xi}_1 = (1, 0, -1)', \boldsymbol{\xi}_2 = (1, -1, 0)'.
\]
解线性方程组$(4E-A)X=0$,得一个基础解系\[
\boldsymbol{\xi}_3 = (1, 1, 1)'.
\]
用 Schmidt 正交化法把$\boldsymbol{\xi}_1, \boldsymbol{\xi}_2$化为正交向量:\begin{align*}
\boldsymbol{\xi}_1'&=(1, 0, -1)',\\
\boldsymbol{\xi}_2'&=\boldsymbol{\xi}_2-\frac{(\boldsymbol{\xi}_2, \boldsymbol{\xi}_1')}{(\boldsymbol{\xi}_1', \boldsymbol{\xi}_1')}\boldsymbol{\xi}_1'=\left(\frac12, -\frac32, \frac12\right)'.
\end{align*}
把$\boldsymbol{\xi}_1', \boldsymbol{\xi}_2', \boldsymbol{\xi}_3$单位化得\begin{align*}
\boldsymbol{\eta}_1&=\frac{\boldsymbol{\xi}_1'}{|\boldsymbol{\xi}_1'|}=\left(\frac{\sqrt{2}}{2}, 0, -\frac{\sqrt{2}}{2}\right)',\\
\boldsymbol{\eta}_2&=\frac{\boldsymbol{\xi}_2'}{|\boldsymbol{\xi}_2'|}=\left(\frac{\sqrt{11}}{11}, \frac{-3\sqrt{11}}{11}, \frac{\sqrt{11}}{11}\right)',\\
\boldsymbol{\eta}_3&=\frac{\boldsymbol{\xi}_3}{|\boldsymbol{\xi}_3|}=\left(\frac{\sqrt{3}}{3}, \frac{\sqrt{3}}{3}, \frac{\sqrt{3}}{3}\right)'.
\end{align*}
令$T=(\boldsymbol{\eta}_1, \boldsymbol{\eta}_2, \boldsymbol{\eta}_3)$,则$T$是正交矩阵,且\[
TAT'={\rm diag}\{1, 1, 4\}.
\]
因此,作正交变换\[
\left\{ \begin{array}{l}
	x=\frac{\sqrt{2}}{2}x'+\frac{\sqrt{11}}{11}y'+\frac{\sqrt{3}}{3}z',\\
	y=-\frac{3\sqrt{11}}{11}y'+\frac{\sqrt{3}}{3}z',\\
	z=-\frac{\sqrt{2}}{2}x'+\frac{\sqrt{11}}{11}y'+\frac{\sqrt{3}}{3}z'.\\
\end{array} \right. 
\]
后,原二次型$f(x, y, z)$在新的坐标系下的表达式为\[
f(x, y, z)={x^{\prime}}^2+{y^{\prime}}^2+4{z^{\prime}}^2.
\]
而单位球面经过正交变换后仍是单位球面,因此只需求${x^{\prime}}^2+{y^{\prime}}^2+4{z^{\prime}}^2$在单位球面上的最大值。\\
因为
\[
{x^{\prime}}^2+{y^{\prime}}^2+{z^{\prime}}^2=1,
\]
所以
\[
{x^{\prime}}^2+{y^{\prime}}^2+4{z^{\prime}}^2=1+3{z^{\prime}}^2.
\]
因此$f(x, y, z)$的最大值为$4$,此时$(x', y', z')=(0, 0, 1)\text{或}(0, 0, -1)$,于是$(x, y, z)'=T(x', y', z')'=\left(\frac{\sqrt{3}}{3}, \frac{\sqrt{3}}{3}, \frac{\sqrt{3}}{3}\right)'\text{或}\left(\frac{\sqrt{3}}{3}, \frac{\sqrt{3}}{3}, -\frac{\sqrt{3}}{3}\right)'$;最小值为$1$,此时$(x', y', z')=(x', y', 0), x'^2+y'^2=1$,于是$(x, y, z)'=T(x', y', z')'=\left( \frac{\sqrt{2}}{2}x'+\frac{\sqrt{11}}{11}y',-\frac{3\sqrt{11}}{11}y',-\frac{\sqrt{2}}{2}x'+\frac{\sqrt{11}}{11}y' \right)' $,其中$x', y'$可取满足$x'^2+y'^2=1$的一切实数。\\
综上所述,$f(x, y, z)$在单位球面上的最大值为$4$,最小值为$1$,最大值点为\[
(x, y, z)=\left(\frac{\sqrt{3}}{3}, \frac{\sqrt{3}}{3}, \frac{\sqrt{3}}{3}\right)'\text{或}\left(\frac{\sqrt{3}}{3}, \frac{\sqrt{3}}{3}, -\frac{\sqrt{3}}{3}\right)',
\]
最小值点为\[
(x, y, z)=\left( \frac{\sqrt{2}}{2}x'+\frac{\sqrt{11}}{11}y',-\frac{3\sqrt{11}}{11}y',-\frac{\sqrt{2}}{2}x'+\frac{\sqrt{11}}{11}y' \right),
\]
其中$x', y'$可取满足$x'^2+y'^2=1$的一切实数。
\end{solution}

\item[五、](15分)
设$A$、$B$为$n$维向量空间$V$上的线性变换,且$AB=A+B$。\\
(1) 求证$A-E$与$B-E$都可逆;\\
(2) 求证$AB=BA$。
\begin{proof}
(1) 因为$AB=A+B$,所以$(A-E)(B-E)=E$,所以$A-E$与$B-E$都可逆。\\
(2) 因为$(A-E)(B-E)=E$,所以$(B-E)(A-E)=E$,所以$BA=B+A$,所以$AB=BA$。
\end{proof}

\item[六、](15分)
设$A$为$n$维线性空间$V$上的线性变换,求证$V =\mathscr{A}(V)\oplus \Ker \mathscr{A}$的充分必要条件是$\operatorname { Ker } \mathscr{A} ^ { 2 } = \mathrm { Ker } \mathscr{A}$。
\begin{proof}
必要性。显然$\Ker \mathscr{A} \subset \Ker \mathscr{A}^2$,下证$\Ker \mathscr{A}^2 \subset \Ker \mathscr{A}$。\\
任取一个向量$\boldsymbol{\alpha}\in\Ker\mathscr{A}^2$,有\[
\mathscr{A}^2\boldsymbol{\alpha}=0.
\]
从而$\mathscr{A}\boldsymbol{\alpha}\in\Ker\mathscr{A}$。\\
由于$V=\mathscr{A}(V)\oplus \Ker \mathscr{A}$,所以\[
\mathscr{A}(V)\cap \Ker \mathscr{A}=0.
\]
因为$\mathscr{A}\boldsymbol{\alpha}\in\mathscr{A}V$,所以\[
\mathscr{A}\boldsymbol{\alpha}\in\mathscr{A}V \cap \Ker\mathscr{A}=0.
\]
从而$\boldsymbol{\alpha}\in\Ker\mathscr{A}$。于是$\operatorname { Ker } \mathscr{A} ^ { 2 } = \mathrm { Ker } \mathscr{A}$。\\
充分性。由于\[
V / \mathrm { Ker } \mathscr { A } \cong  \mathscr { A }V.
\]
所以\[
\dim V=\dim \mathscr{A}V+\dim\Ker\mathscr{A}.
\]
任取一个向量$\boldsymbol{\alpha}\in\mathscr{A}V\cap\Ker\mathscr{A}$,有\[
\mathscr{A}\boldsymbol{\alpha}=0, \text{存在}\boldsymbol{\beta}\in V, \text{使得} \boldsymbol{\alpha}=\mathscr{A}\boldsymbol{\beta}.
\]
因此\[
\mathscr{A}^2\boldsymbol{\beta}=\mathscr{A}\boldsymbol{\alpha}=0.
\]
因为$\Ker\mathscr{A}^2=\Ker\mathscr{A}$,所以\[
\boldsymbol{\alpha}=\mathscr{A}\boldsymbol{\beta}=0.
\]
从而$\mathscr{A}V\cap\Ker\mathscr{A}=0$,于是$V=\mathscr{A}V\oplus\Ker\mathscr{A}$。
\end{proof}

\item[七、](15分)
设$A$为$n$阶实对称阵,$\lambda _ { 1 } , \dots , \lambda _ {  { m } }$为$A$的一切不同特征值;若非零$n$维实列向量$\boldsymbol{\beta}$与特征值$\lambda _ { 1 } , \dots , \lambda _ {  { m } - 1 }$的特征向量都正交,求证$\boldsymbol{\beta}$为对应特征值$\lambda_m$的特征向量。
\begin{proof}
因为$A$是实对称矩阵,所以$A$可以对角化。于是
\[
V=V_{\lambda_{1}}\oplus V_{\lambda_{2}}\oplus\cdots\oplus V_{\lambda_{m}}.
\]
设\[
\boldsymbol{\beta}=\boldsymbol{\beta}_1+\boldsymbol{\beta}_2+\cdots+\boldsymbol{\beta}_m, \boldsymbol{\beta}_i\in V_{\lambda_{i}}, i=1, 2, \dots, m.
\]
则\[
(\boldsymbol{\beta}, \boldsymbol{\beta}_i)=0, i=1, 2, \dots, m-1.
\]
又$V_{\lambda_{i}}\perp V_{\lambda_{j}}, i, j=1, 2, \dots, m, i\ne j$,所以\[
\boldsymbol{\beta}=\boldsymbol{\beta}_m.
\]
因此$\boldsymbol{\beta}$为对应特征值$\lambda_m$的特征向量。
\end{proof}

\item[八、](15 分)
设$A$为$n$阶实对称阵,且对任何非零$n$维实列向量$x$有$x^TAx\ne 0$,求证$A$正定或负定。
\begin{proof}
反证,假设$A$既不是正定矩阵也不是负定矩阵。因为$A$为$n$阶实对称阵,所以存在可逆矩阵$P$,使得\[
PAP^{\prime}=\left( \begin{matrix}
	E_s&		&		\\
	&		-E_t&		\\
	&		&		0\\
\end{matrix} \right) .
\]
取$\x=\left( 1,\underset{s-1\text{个}}{\underbrace{0,\dots ,0}},1,0\dotsc ,0 \right) $,$\y=P\x$,则$\y\ne0$,且$\y^T A \y=0$,矛盾。
\end{proof}

\item[九、](15 分)
设$V$为复数域上的$n$维线性空间,$\mathscr{A}$为$V$上的线性变换,$\lambda$为一个复数;令$V _ { \lambda } = \{ v \in V | ( A - \lambda E ) ^ { k } v = 0 ( \exists k \geq 1 ) \}$。\\
(1) 求证$\lambda$为$A$的特征值的充分必要条件为$V _ { \lambda } \ne \{ 0 \}$;\\
(2) 求证$V_{\lambda}$为$\mathscr{A}$的不变子空间;\\
(3) 求证当$V _ { \lambda } \ne \{ 0 \}$时,$\mathscr{A}$在$V_{\lambda}$上没有与$\lambda$不同的特征值。
\begin{proof}
(1) 必要性。设$\lambda$为$A$的特征值,$\boldsymbol{\alpha}$为$A$的属于$\lambda$的特征向量,则\[
(A-\lambda E)\boldsymbol{\alpha}=0.
\]
因此,$V_{\lambda}\ne \{0\}$。\\
充分性。设$V _ { \lambda } \ne \{ 0 \}$,则存在$k\ge 1$,使得\[
\{ v \in V | ( A - \lambda E ) ^ { k } v = 0 ( \exists k \ge 1 ) \}\ne 0.
\]
因此存在非零向量$\boldsymbol{\alpha}\in V$,使得\[
(A-\lambda E)(A-\lambda E)^{k-1} \boldsymbol{\alpha}=0.
\]
显然$(A-\lambda E)^{k-1} \boldsymbol{\alpha}\ne 0$,所以$\lambda$为$A$的特征值。

(2) 显然 $V_{\lambda}$ 是 $\mathscr{A}$ 的子空间,下证它是 $\mathscr{A}$ 的不变子空间。为此,先证对任意的$ k\in\mathbb{N}^*$, $\mathscr{A}$与$(\mathscr{A}-\lambda E)^k$ 可交换。

$k=1$时,显然。

假设$\mathscr{A}(\mathscr{A}-\lambda E)^{k-1}=(\mathscr{A}-\lambda E)^{k-1}\mathscr{A}$,则\[
\mathscr{A}(\mathscr{A}-\lambda E)^{k}=\mathscr{A}(\mathscr{A}-\lambda E)^{k-1}(\mathscr{A}-\lambda E)=(\mathscr{A}-\lambda E)^{k-1}\mathscr{A}(\mathscr{A}-\lambda E)=(\mathscr{A}-\lambda E)^{k}\mathscr{A}.
\]
由数学归纳法原理得,$\mathscr{A}(\mathscr{A}-\lambda E)^{k}=(\mathscr{A}-\lambda E)^{k}\mathscr{A}$对任意$k\in\mathbb{N}^*$都成立。

任取一个向量$\boldsymbol{\alpha}\in V_{\lambda}$,有\[
(\mathscr{A}-\lambda E)^{k}(\mathscr{A}\boldsymbol{\alpha})=\mathscr{A}(\mathscr{A}-\lambda E)^{k}\boldsymbol{\alpha}=0.
\]
因此$\mathscr{A}\boldsymbol{\alpha}\in V_{\lambda}$,所以$V_{\lambda}$为$\mathscr{A}$的不变子空间。

(3) 先证明$\lambda$是$\mathscr{A}$在$V_{\lambda}$上的特征值。这只需证明$\Ker(\mathscr{A}-\lambda E)\subset V_{\lambda}$。注意到对任意的$k\ge 1$,\[
\Ker(\mathscr{A}-\lambda E)\subset \Ker(\mathscr{A}-\lambda E)^k.
\]
因此,$\Ker(\mathscr{A}-\lambda E)\subset V_{\lambda}$。\\
再设存在复数$\mu\ne\lambda$是$\mathscr{A}$在$V_{\lambda}$上的特征值。因为对任意的$k\ge 1$,$x-\mu$与$(x-\lambda)^k$互素,因此存在$f(x), g(x)\in\mathbb{C}[x]$,使得\[
f(x)(x-\mu)+g(x)(x-\lambda)^k=1.
\]
任取一个$\mathscr{A}$的属于$\mu$的特征向量$\u$,由 Hamilton-Cayley 定理得\[
\u=f(\mathscr{A})(\mathscr{A}-\mu E)\u+g(\mathscr{A})(\mathscr{A}-\lambda E)\u=0+0=0.
\]
这与$\u$是特征向量矛盾。因此,$\mathscr{A}$在$V_{\lambda}$上没有与$\lambda$不同的特征值。
\end{proof}

\item[十、](12 分)
若$A$,$B$为$m\times n$矩阵,试用三种不同的方法证明(每个方法 $4$ 分)$r ( A + B ) \le r ( A ) + r ( B ) $。
\begin{proof}
证法一。设$A=(\boldsymbol{\alpha}_1, \boldsymbol{\alpha}_2, \dots, \boldsymbol{\alpha}_n)$,$B=(\boldsymbol{\beta}_1, \boldsymbol{\beta}_2, \dots, \boldsymbol{\beta}_n)$,其中$\boldsymbol{\alpha}_i, \boldsymbol{\beta}_i, i=1, 2, \dots, n$是$m$维列向量。则\[
A+B=(\boldsymbol{\alpha}_1+\boldsymbol{\beta}_1, \boldsymbol{\alpha}_2+\boldsymbol{\beta}_2, \dots, \boldsymbol{\alpha}_n+\boldsymbol{\beta}_n).
\]
因为向量组 $\boldsymbol{\alpha}_1+\boldsymbol{\beta}_1, \boldsymbol{\alpha}_2+\boldsymbol{\beta}_2, \dots, \boldsymbol{\alpha}_n+\boldsymbol{\beta}_n$ 可以由向量组 $\boldsymbol{\alpha}_1, \boldsymbol{\alpha}_2, \dots, \boldsymbol{\alpha}_n, \boldsymbol{\beta}_1, \boldsymbol{\beta}_2, \dots, \boldsymbol{\beta}_n$ 线性表出,所以向量组$\boldsymbol{\alpha}_1+\boldsymbol{\beta}_1, \boldsymbol{\alpha}_2+\boldsymbol{\beta}_2, \dots, \boldsymbol{\alpha}_n+\boldsymbol{\beta}_n$的秩小于等于向量组$\boldsymbol{\alpha}_1, \boldsymbol{\alpha}_2, \dots, \boldsymbol{\alpha}_n, \boldsymbol{\beta}_1, \boldsymbol{\beta}_2, \dots, \boldsymbol{\beta}_n$的秩,因此$r(A+B)\le r(A)+r(B)$。\\
证法二。
\end{proof}
\end{enumerate}
\endinput