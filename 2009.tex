\section{2009}
\begin{enumerate}[1~]
\renewcommand{\labelenumi}{\textbf{\theenumi. }}
\renewcommand{\Im}{\text{Im }}
\item[一、]
填空题 (每小题 4 分,共 20 分)
\begin{enumerate}[1.~]
\item
满足 $\mathbb{Q} \subset \mathbb{F} \subset \mathbb{Q}[\sqrt{2}]$的数域 $\mathbb{F}$ 有(\quad)。
\begin{solution}
由 $ \mathbb{Q}(\sqrt{2})$ 的定义知,$ \mathbb{Q}(\sqrt{2})$ 是包含 $ \mathbb{Q}$ 和 $\sqrt{2}$ 的最小数域,因此 $\mathbb{F}$ 为 $\mathbb{Q}$或$ \mathbb{Q}(\sqrt{2})$。

\end{solution}

\item
有理数域上以 $\sqrt { 2 } + \sqrt { 3 }$ 为根首项系数为 1 的不可约多项式是(\quad)。
\begin{solution}
设$t=\sqrt{2}+\sqrt{3}$,则$t^2=(\sqrt{2}+\sqrt{3})^2=5+2\sqrt{6}$,$t^4=(5+2\sqrt{6})^2=49+20\sqrt{6}$,于是$t^4-10t^2+1=0$。从而$t=\sqrt{2}+\sqrt{3}$是$f(x)=x^4-10x^2+1$的一个实根。把$f(x)$看作实数域上的多项式进行分解:\begin{align*}
f(x)&=x^4-10x^2+1=x^4+2x^2+1-12x^2=(x^2+1)^2-(2\sqrt{3}x)^2\\
&=(x^2+2\sqrt{3}x+1)(x^2-2\sqrt{3}x+1)\\
&=[x-(\sqrt{2}-\sqrt{3})][x+(\sqrt{2}+\sqrt{3})][x-(\sqrt{2}+\sqrt{3})][x+(\sqrt{2}-\sqrt{3})].
\end{align*}
于是根据唯一因式分解定理得,$f(x)=x^4-10x^2+1$在$\mathbb{Q}[x]$上没有一次因式,也没有二次因式。从而$x^4-10x^2+1$在$\mathbb{Q}$上不可约。因此有理数域上以 $\sqrt { 2 } + \sqrt { 3 }$ 为根首项系数为 1 的不可约多项式是$x^4-10x^2+1$。
\end{solution}

\item
$n$ 阶行列式
$$\left| \begin{array} { c c c c } { 1 + x _ { 1 } ^ { 2 } } & { x _ { 2 } x _ { 1 } } & { \cdots } & { x _ { n } x _ { 1 } } \\ { x _ { 1 } x _ { 2 } } & { 1 + x _ { 2 } ^ { 2 } } & { \cdots } & { x _ { n } x _ { 2 } } \\ { \vdots } & { \vdots } & { } & { \vdots } \\ { x _ { 1 } x _ { n } } & { x _ { 2 } x _ { n } } & { \cdots } & { 1 + x _ { n } ^ { 2 } } \end{array} \right|$$
的值为 (\quad)。
\begin{solution}
\begin{align*}
&\left| \begin{array} { c c c c } { 1 + x _ { 1 } ^ { 2 } } & { x _ { 2 } x _ { 1 } } & { \cdots } & { x _ { n } x _ { 1 } } \\ { x _ { 1 } x _ { 2 } } & { 1 + x _ { 2 } ^ { 2 } } & { \cdots } & { x _ { n } x _ { 2 } } \\ { \vdots } & { \vdots } & { } & { \vdots } \\ { x _ { 1 } x _ { n } } & { x _ { 2 } x _ { n } } & { \cdots } & { 1 + x _ { n } ^ { 2 } } \end{array} \right|
=\left|E_n+\left( \begin{array}{c}
	x_1\\
	x_2\\
	\vdots\\
	x_n\\
\end{array} \right) \left( \begin{matrix}
	x_1&		x_2&		\cdots&		x_n\\
\end{matrix} \right) \right|\\
&=\left|1+\left( \begin{matrix}
	x_1&		x_2&		\cdots&		x_n\\
\end{matrix} \right) \left( \begin{array}{c}
	x_1\\
	x_2\\
	\vdots\\
	x_n\\
\end{array} \right) \right|=1+\sum_{i=1}^n x_i^2.
\end{align*}
\end{solution}

\item
若 $A$ 为可逆矩阵,则 $(A^{-1})^* = (\quad)A$。
\begin{solution}
$(A^{-1})^*=(A^*)^{-1}=|A^*|^{-1}(A^*)^*=||A|A^{-1}|^{-1}A=|A|^{1-n}A$。
\end{solution}

\item
若 $V_1, V_2$ 为 $3$ 维线性空间中两个不同的 $2$ 维子空间,则 $\dim(V_1 + V_2) =$(\quad)。
\begin{solution}
由题意,有\[
\dim V_1=\dim V_2=2, \dim (V_1\cap V_2)=1.
\]
所以\[
\dim(V_1+V_2)=\dim V_1+\dim V_2-\dim(V_1\cap V_2)=3.
\]
\end{solution}

\item
令 $A \in \mathbb { R } ^ { n \times n } , A ^ { n - 1 } \neq 0 , A ^ { n } = 0$,则 $V = \{ f ( A ) | f ( x ) \in \mathbb { R } [ x ] \}$ 作为实数域上的线性空间其维数为 (\quad)。
\begin{solution}
易知$I, A, A^2, \dots, A^{n-1}$是$V$的一个基,故$\dim V=n$。
\end{solution}

\item
$A \in \mathbb { R } ^ { m \times n }$,则线性方程组 $Ax = b$ 对任何向量$b \in \mathbb { R } ^ { m }$都有解的充要条件为( \quad) 。
\begin{solution}
\begin{align*}
&\text{线性方程组 $Ax = b$ 对任何向量$b \in \mathbb { R } ^ { m }$都有解}\\
&\Longleftrightarrow \text{对任何向量$b \in \mathbb { R } ^ { m }$,${\rm rank}A={\rm rank}(A|b)$}\\
&\Longleftrightarrow \text{对任何向量$b \in \mathbb { R } ^ { m }$,$b$可由$A$的列向量线性表出。}\\
&\Longleftrightarrow \text{$\mathbb{R}^{m}\subset \mathcal{R}(A)$,其中$\mathcal{R}(A)$表示由$A$的列向量生成的线性空间。}\\
&\Longleftrightarrow \text{$\dim \mathcal{R}(A)\ge\dim \mathbb{R}^m=m$。}\\
&\Longleftrightarrow \text{$A$的列秩大于等于$m$。}\\
&\Longleftrightarrow \text{$A$的秩大于等于$m$。}
\end{align*}
由题意,${\rm rank}A\le m$,所以\begin{align*}
\text{线性方程组 $Ax = b$ 对任何向量$b \in \mathbb { R } ^ { m }$都有解}
&\Longleftrightarrow \text{${\rm rank}A=m$。}
\end{align*}
\end{solution}

\item
若 $A$ 为 $3$ 阶实对称阵,其特征值为 $-3, 1, 4$,则当 $t$ 满足( \quad)  时,$tE + A$ 正定。
\begin{solution}
$tE+A$的特征值为$t-3, t+1, t+4$,因此\begin{align*}
tE+A\text{正定}&\Longleftrightarrow t-3>0\text{且}t+1>0\text{且}t+4>0\Longleftrightarrow t>3\text{且}t>-1\text{且}t>-4 \\
&\Longleftrightarrow t>3.
\end{align*}
因此当$t>3$时,$tE+A$正定。
\end{solution}

\item
令 $A\in \mathbb{R}^{4\times 4}$ 的特征值为 $1, 2, 3, 4$,则 ${\rm tr}(A^2) =$ (\quad)。
\begin{solution}
因为$A$的特征值为$1, 2, 3, 4$,所以$A^2$的特征值为$1, 4, 9, 16$,所以${\rm tr}(A^2)=30$。
\end{solution}

\item
一切 $n$ 阶幂等阵 $(A^2 = A)$ 在复数域内按相似可分为 ( \quad) 类。
\begin{solution}
(复数域是多余的)因为幂等矩阵必可对角化,所以\[
A\sim \left\{ \begin{array}{l}
	\text{diag}\{E_r,0_{n-r}\},\ 1\le r=\text{rank}A\le n, \\
	0,\ \text{rank}A=0.\\
\end{array} \right. 
\]
所以一切 $n$ 阶幂等阵 $(A^2 = A)$ 在复数域内按相似可分为 $n+1$类。
\end{solution}

\end{enumerate}

\item[二、](15分)
设 $V$ 为实数域 $\mathbb{R}$ 上的 $5$ 维线性空间,$\mathscr{A}$ 为其上的线性变换,且 $\mathscr{A}$ 在基 $\boldsymbol{\varepsilon} _ { 1 } , \boldsymbol{\varepsilon} _ { 2 } , \boldsymbol{\varepsilon} _ { 3 } , \boldsymbol{\varepsilon} _ { 4 } , \boldsymbol{\varepsilon} _ { 5 }$之下的矩阵为\[
A=\left( \begin{matrix}
	&		&		&		&		1\\
	&		&		&		1&		\\
	&		&		1&		&		\\
	&		1&		&		&		\\
	1&		&		&		&		\\
\end{matrix} \right) .
\]
(1) 求 $V$ 的另一组基 $\boldsymbol{\alpha} _ { 1 } , \boldsymbol{\alpha} _ { 2 } , \boldsymbol{\alpha} _ { 3 } , \boldsymbol{\alpha} _ { 4 } , \boldsymbol{\alpha} _ { 5 }$,使 $\mathscr{A}$ 在此基下的矩阵为对角阵。\\
(2) 求$A^n$。
\begin{solution}
(1) 因为\[
\mathscr{A}(\boldsymbol{\varepsilon} _ { 1 } , \boldsymbol{\varepsilon} _ { 2 } , \boldsymbol{\varepsilon} _ { 3 } , \boldsymbol{\varepsilon} _ { 4 } , \boldsymbol{\varepsilon} _ { 5 })=(\boldsymbol{\varepsilon} _ { 1 } , \boldsymbol{\varepsilon} _ { 2 } , \boldsymbol{\varepsilon} _ { 3 } , \boldsymbol{\varepsilon} _ { 4 } , \boldsymbol{\varepsilon} _ { 5 }) \left( \begin{matrix}
	&		&		&		&		1\\
	&		&		&		1&		\\
	&		&		1&		&		\\
	&		1&		&		&		\\
	1&		&		&		&		\\
\end{matrix} \right).
\]
所以\[
\left\{ \begin{array}{l}
	\mathscr{A}\boldsymbol{\varepsilon} _1=\boldsymbol{\varepsilon} _5,\\
	\mathscr{A}\boldsymbol{\varepsilon} _2=\boldsymbol{\varepsilon} _4,\\
	\mathscr{A}\boldsymbol{\varepsilon} _3=\boldsymbol{\varepsilon} _3,\\
	\mathscr{A}\boldsymbol{\varepsilon} _4=\boldsymbol{\varepsilon} _2,\\
	\mathscr{A}\boldsymbol{\varepsilon} _5=\boldsymbol{\varepsilon} _1.\\
\end{array} \right. 
\]
因此\[
\left\{ \begin{array}{l}
	\mathscr{A}(\boldsymbol{\varepsilon} _1+\boldsymbol{\varepsilon}_5)=(\boldsymbol{\varepsilon}_1+\boldsymbol{\varepsilon} _5),\\
	\mathscr{A}(\boldsymbol{\varepsilon} _2+\boldsymbol{\varepsilon}_4)=(\boldsymbol{\varepsilon}_2+\boldsymbol{\varepsilon} _4),\\
	\mathscr{A}\boldsymbol{\varepsilon} _3=\boldsymbol{\varepsilon} _3,\\
	\mathscr{A}(\boldsymbol{\varepsilon} _1-\boldsymbol{\varepsilon}_5)=-(\boldsymbol{\varepsilon}_1-\boldsymbol{\varepsilon} _5),\\
	\mathscr{A}(\boldsymbol{\varepsilon} _2-\boldsymbol{\varepsilon}_4)=-(\boldsymbol{\varepsilon}_2-\boldsymbol{\varepsilon} _4).\\
\end{array} \right. 
\]
于是\[
\mathscr{A}(\boldsymbol{\varepsilon} _1+\boldsymbol{\varepsilon}_5, \boldsymbol{\varepsilon} _2+\boldsymbol{\varepsilon}_4, \boldsymbol{\varepsilon} _3, \boldsymbol{\varepsilon} _1-\boldsymbol{\varepsilon}_5, \boldsymbol{\varepsilon} _2-\boldsymbol{\varepsilon}_4)=(\boldsymbol{\varepsilon} _1+\boldsymbol{\varepsilon}_5, \boldsymbol{\varepsilon} _2+\boldsymbol{\varepsilon}_4, \boldsymbol{\varepsilon} _3, \boldsymbol{\varepsilon} _1-\boldsymbol{\varepsilon}_5, \boldsymbol{\varepsilon} _2-\boldsymbol{\varepsilon}_4)\left( \begin{matrix}
	1&		&		&		&		\\
	&		1&		&		&		\\
	&		&		1&		&		\\
	&		&		&		-1&		\\
	&		&		&		&		-1\\
\end{matrix} \right).
\]
因此,取$(\boldsymbol{\alpha} _ { 1 } , \boldsymbol{\alpha} _ { 2 } , \boldsymbol{\alpha} _ { 3 } , \boldsymbol{\alpha} _ { 4 } , \boldsymbol{\alpha} _ { 5 })=(\boldsymbol{\varepsilon} _1+\boldsymbol{\varepsilon}_5, \boldsymbol{\varepsilon} _2+\boldsymbol{\varepsilon}_4, \boldsymbol{\varepsilon} _3, \boldsymbol{\varepsilon} _1-\boldsymbol{\varepsilon}_5, \boldsymbol{\varepsilon} _2-\boldsymbol{\varepsilon}_4)$即可。\\
(2) 记\[
B=(\boldsymbol{\varepsilon} _ { 1 } , \boldsymbol{\varepsilon} _ { 2 } , \boldsymbol{\varepsilon} _ { 3 } , \boldsymbol{\varepsilon} _ { 4 } , \boldsymbol{\varepsilon} _ { 5 }),
\]
\[
C=(\boldsymbol{\varepsilon} _1+\boldsymbol{\varepsilon}_5, \boldsymbol{\varepsilon} _2+\boldsymbol{\varepsilon}_4, \boldsymbol{\varepsilon} _3, \boldsymbol{\varepsilon} _1-\boldsymbol{\varepsilon}_5, \boldsymbol{\varepsilon} _2-\boldsymbol{\varepsilon}_4).
\]
由(1)知\[
B^{-1}C\left( \begin{matrix}
	1&		&		&		&		\\
	&		1&		&		&		\\
	&		&		1&		&		\\
	&		&		&		-1&		\\
	&		&		&		&		-1\\
\end{matrix} \right)C^{-1}B=A.
\]
因此\[
A^n=B^{-1}C\left( \begin{matrix}
	1&		&		&		&		\\
	&		1&		&		&		\\
	&		&		1&		&		\\
	&		&		&		(-1)^n&		\\
	&		&		&		&		(-1)^n\\
\end{matrix} \right)C^{-1}B.
\]
又\[
C=B\left( \begin{matrix}
	1&		&		&		1&		\\
	&		1&		&		&		1\\
	&		&		1&		&		\\
	&		1&		&		&		-1\\
	1&		&		&		-1&		\\
\end{matrix} \right).
\]
所以\begin{align*}
A^n&=\left( \begin{matrix}
	1&		&		&		1&		\\
	&		1&		&		&		1\\
	&		&		1&		&		\\
	&		1&		&		&		-1\\
	1&		&		&		-1&		\\
\end{matrix} \right)
\left( \begin{matrix}
	1&		&		&		&		\\
	&		1&		&		&		\\
	&		&		1&		&		\\
	&		&		&		(-1)^n&		\\
	&		&		&		&		(-1)^n\\
\end{matrix} \right)
\left( \begin{matrix}
	1&		&		&		1&		\\
	&		1&		&		&		1\\
	&		&		1&		&		\\
	&		1&		&		&		-1\\
	1&		&		&		-1&		\\
\end{matrix} \right)^{-1}\\
&=\frac12\left( \begin{matrix}
	1+(-1)^n&		&		&		&1+(-1)^{n+1}		\\
	&		1+(-1)^n&		&1+(-1)^{n+1}		&		\\
	&		&		1&		&		\\
	&		1+(-1)^{n+1}&		&1+(-1)^n		&		\\
	1+(-1)^{n+1}&		&		&		&1+(-1)^n		\\
\end{matrix} \right).
\end{align*}

\end{solution}

\item[三、](15分)
设 $A \in \mathbb { R } ^ { n \times n } , R ( A ) = \{ A x | x \in \mathbb { R } ^ { n } \} , N ( A ) = \left\{ x \in \mathbb { R } ^ { n } | A x = 0 \right\}$。若 $A$ 与 $A^2$ 有相同的秩。求证:\\
(1) 齐次线性方程组 $AX = 0$ 和 $A^2X = 0$ 同解。\\
(2) $\mathbb { R } ^ { n } = R ( A ) \oplus N ( A )$。
\begin{solution}
(回忆错了。)\\
(1) 显然$AX=0$的解是$A^2X=0$的解,下证$A^2X=0$的解也是$AX=0$的解。\\
因为${\rm rank}(A^2)={\rm rank}A$,所以存在可逆矩阵$P$,使得\[
PA^2=A.
\]
任取$A^2X=0$的一个解$B$,有\[
A^2B=0,
\]
从而\[
PA^2B=0,
\]
于是\[
AB=0.
\]
因此$B$是$AX=0$的一个解。\\
因此齐次线性方程组 $AX = 0$ 和 $A^2X = 0$ 同解。\\
(2) (补充条件:$A=A^2$)任取一个向量$\boldsymbol{\alpha}\in \mathbb{R}^n$,有\[
\boldsymbol{\alpha}={A}\boldsymbol{\alpha}+\boldsymbol{\alpha}-{A}\boldsymbol{\alpha}.
\]
因为\[
A(\boldsymbol{\alpha}-{A}\boldsymbol{\alpha})=0,
\]
所以$
\boldsymbol{\alpha}-{A}\boldsymbol{\alpha}\in N(A).
$
又$A\boldsymbol{\alpha}\in R(A)$,所以\[
\mathbb { R } ^ { n } = R ( A ) + N ( A ).
\]
任取一个向量$\boldsymbol{\beta}\in R(A)\cap N(A)$,有\[
A\boldsymbol{\beta}=0, \text{存在向量$\boldsymbol{\gamma}\in \mathbb{R}^n$,使得$A\boldsymbol{\gamma}=\boldsymbol{\beta}$。}
\]
因此\[
A\boldsymbol{\gamma}=A^2\boldsymbol{\gamma}=A\boldsymbol{\beta}=0,
\]
于是\[
\boldsymbol{\beta} = 0.
\]
从而$R(A)\cap N(A)=0$。\\
因此\[
\mathbb { R } ^ { n } = R ( A ) \oplus N ( A ).
\]
\end{solution}

\item[四、](15分)
设 $V$ 为数域 $\mathbb{C}$ 上的 $n$ 维线性空间 $(n \ge 2)$,$\mathscr{A}$ 为其上的线性变换,$\mathscr{A}^{n-1}\ne 0, \mathscr{A}^{n} = 0$。 求证:$\mathscr{A}$ 在 $V$的某个基下的矩阵为\[
\left( \begin{array} { c c c c c } { 0 } & { } & { } & { } & { } \\ { 1 } & { \ddots } & { } & { } & { } \\ { } & { \ddots } & { \ddots } & { } & { } \\ { } & { } & { 1 } & { 0 } & { } \\ { } & { } & { } & { 1 } & { 0 } \end{array} \right).
\]
\begin{proof}
\begin{lemma}
对任意的$\boldsymbol{\alpha}\in V$,$\boldsymbol{\alpha}\ne0$,$\boldsymbol{\alpha}, \mathscr{A}\boldsymbol{\alpha}, \mathscr{A}^2\boldsymbol{\alpha}, \dots, \mathscr{A}^{n-1}\boldsymbol{\alpha}$线性无关,从而是$V$的一个基。
\end{lemma}
\begin{subproof}
任取$\boldsymbol{\alpha}\in V$,$\boldsymbol{\alpha}\ne0$,设
\begin{equation}\label{k0alpha}
k_0\boldsymbol{\alpha}+k_1\mathscr{A}\boldsymbol{\alpha}+k_2\mathscr{A}^2\boldsymbol{\alpha}+\cdots+k_{n-1}\mathscr{A}^{n-1}\boldsymbol{\alpha}=0, k_0, k_1, k_2, \dots, k_{n-1}\in \mathbb{C}.
\end{equation}
\ref{k0alpha}式两边同时用$\mathscr{A}^{n-1}$作用之,得到\[
k_0\mathscr{A}^{n-1}\boldsymbol{\alpha}=0.
\]
从而$k_0=0$。于是\ref{k0alpha}式变为\begin{equation}\label{k1alpha}
k_1\mathscr{A}\boldsymbol{\alpha}+k_2\mathscr{A}^2\boldsymbol{\alpha}+\cdots+k_{n-1}\mathscr{A}^{n-1}\boldsymbol{\alpha}=0.
\end{equation}
\\
\ref{k1alpha}式两端同时用$\mathscr{A}^{n-2}$作用之,得到\[
k_1\mathscr{A}^{n-1}\boldsymbol{\alpha}=0.
\]

从而$k_1=0$,于是\ref{k1alpha}式变为\begin{equation}\label{k2alpha}
k_2\mathscr{A}^2\boldsymbol{\alpha}+\cdots+k_{n-1}\mathscr{A}^{n-1}\boldsymbol{\alpha}=0.
\end{equation}
仿照上面的步骤,依次用$\mathscr{A}^{n-3}$,$\mathscr{A}^{n-4}$,$\dots$,$\mathscr{A}$作用之,可得\[
k_0=k_1=k_2=\cdots=k_{n-1}=0.
\]
因此$\boldsymbol{\alpha}, \mathscr{A}\boldsymbol{\alpha}, \mathscr{A}^2\boldsymbol{\alpha}, \dots, \mathscr{A}^{n-1}\boldsymbol{\alpha}$线性无关,断言成立。
\end{subproof}
因为\[
\mathscr{A}(\boldsymbol{\alpha}, \mathscr{A}\boldsymbol{\alpha}, \mathscr{A}^2\boldsymbol{\alpha}, \dots, \mathscr{A}^{n-1}\boldsymbol{\alpha})=(\boldsymbol{\alpha}, \mathscr{A}\boldsymbol{\alpha}, \mathscr{A}^2\boldsymbol{\alpha}, \dots, \mathscr{A}^{n-1}\boldsymbol{\alpha})\left( \begin{matrix} { 0 } & { } & { } & { } & { } \\ { 1 } & { \ddots } & { } & { } & { } \\ { } & { \ddots } & { \ddots } & { } & { } \\ { } & { } & { 1 } & { 0 } & { } \\ { } & { } & { } & { 1 } & { 0 } \end{matrix} \right).
\]
所以 $\mathscr{A}$ 在基 $\boldsymbol{\alpha}, \mathscr{A}\boldsymbol{\alpha}, \mathscr{A}^2\boldsymbol{\alpha}, \dots, \mathscr{A}^{n-1}\boldsymbol{\alpha}$ 下的矩阵为 $\left( \begin{smallmatrix} { 0 } & { } & { } & { } & { } \\ { 1 } & { \ddots } & { } & { } & { } \\ { } & { \ddots } & { \ddots } & { } & { } \\ { } & { } & { 1 } & { 0 } & { } \\ { } & { } & { } & { 1 } & { 0 } \end{smallmatrix} \right)$。
\end{proof}

\item[五、](15分)
(1) 求证任何一个正定矩阵 $A = B^2$,$B$ 也为正定矩阵。\\
(2) 求证任何一个可逆实矩阵 $A = QP$,$Q$ 为正定矩阵,$P$ 为正交阵。
\begin{proof}
(1) 参见2010年第六题。\\
(2) 
因为$A$是$n$级实可逆矩阵,所以$AA'$是正定矩阵。由(1)得,存在正定矩阵$Q$,使得\[
AA'= Q^2.
\]
从而$A=Q^2{A'}^{-1}$。记$P=Q{A'}^{-1}$。由于\begin{align*}
P'P&=A^{-1}Q'Q{A'}^{-1}=A^{-1}Q^2{A'}^{-1}=A^{-1}AA'{A'}^{-1}\\
&=E_n.
\end{align*}
因此$P$是正交矩阵。从上述$A$的表示式得,$A=QP$。
\end{proof}
\begin{remark}
注意极分解定理。
\begin{theorem}[极分解定理]
对于任一实可逆矩阵$A$,一定存在一个正交矩阵$T$和两个正定矩阵$S_1$,$S_2$,使得\[
A=TS_1= S_2T,
\]
并且这两种分解的每一种都是唯一的。
\end{theorem}
\end{remark}

\item[六、](15分)
设$\mathbb{F}$ 为数域, $A , B \in \mathbb { F } ^ { n 
\times n } ( n \geq 1 ) : A + B = E _ { n } , A B = B A , A ^ 
{ 2 } = A , B ^ { 2 } = B$。求证存在一个可逆矩阵 $P$ 使得\[
P ^ { - 1 } A P = \left( \begin{matrix}
	E_s&		\\
	&		0\\
\end{matrix} \right)  , 
P ^ { - 1 } B P = \left( \begin{matrix}
	0&		\\
	&		E_t\\
\end{matrix} \right) .
\]
这里 $s + t = n$。
\begin{proof}
首先证明四个引理。
\begin{lemma}
幂等矩阵一定可对角化,并且如果$n$级幂等矩阵$A$的秩为$r (r>0)$,那么\[
A \sim \left( \begin{array} { l l } { I _ { r } } & { 0 } \\ { 0 } & { 0 } \end{array} \right).
\]
\end{lemma}
\begin{subproof}
若$r=n$,则$A$可逆。从$A^2=A$得出,$A=I$,结论显然成立。

若$r=0$,则$A=0$。结论也成立。下面设$0<r<n$。

注意到:当$0<r<n$时,幂等矩阵$A$的全部特征值是$0, 1$。

对于特征值$0$,齐次线性方程组$(0I-A)X=0$的解空间$W_0$的维数等于$n-{\rm rank}(-A)=n-r$。

由于$A$是幂等矩阵,因此${\rm rank}(A)+{\rm rank}(I-A)=n$。从而${\rm rank}(I-A)=n-r$。

对于特征值$1$,齐次线性方程组$(I-A)X=0$的解空间$W_1$的维数等于$n-{\rm rank}(I-A)=n-(n-r)=r$。因此\[
\operatorname { dim } W _ { 0 } + \operatorname { dim } W _ { 1 } = ( n - r ) + r = n.
\]
从而$A$可对角化。$A$的相似标准形中,特征值$1$在主对角线上出现的次数等于$W_1$的维数$r$,特征值$0$在主对角线上出现的次数等于$W_0$的维数$n-r$。因此\[
A \sim \left( \begin{array} { l l } { I _ { r } } & { 0 } \\ { 0 } & { 0 } \end{array} \right).
\]
\end{subproof}

\begin{lemma}
数域$\mathbb{F}$上的幂等矩阵的秩等于它的迹。
\end{lemma}
\begin{subproof}
设$A$是数域$\mathbb{F}$上$n$级幂等矩阵,且${\rm rank}(A)=r>0$。则据引理1得\[
A \sim \left( \begin{array} { l l } { I _ { r } } & { 0 } \\ { 0 } & { 0 } \end{array} \right).
\]
由于相似的矩阵有相等的迹,因此\[
\operatorname { tr } ( A ) = \operatorname { tr } \left( \begin{array} { c c } { I _ { r } } & { 0 } \\ { 0 } & { 0 } \end{array} \right) = r = \operatorname { rank } ( A ).
\]
若$A=0$,则${\rm tr}(0)=0={\rm rank}(0)$。
\end{subproof}

\begin{lemma}
 如果域$\mathbb{F}$上的$n$级矩阵$A$与$B$都是可对角化的,且$AB=BA$,那么存在域$\mathbb{F}$上一个$n$级可逆矩阵$S$,使得$S^{-1}AS$与$S^{-1}BS$都为对角矩阵。
\end{lemma}
\begin{subproof}
已知$A$可对角化,设$A$的所有不同的特征值为$\lambda _ { 1 } , \lambda _ { 2 } , \dots , \lambda _ { s }$,
其中$\lambda_i$的重数为$n_i, i=1, 2, \dots, s$。
于是存在域 $\mathbb{F}$ 上的一个 $n$ 级可逆矩阵 $P$,
使得 $P ^ { - 1 } A P = \operatorname { diag } \left\{ \lambda _ { 1 } I _ { n _ { 1 }} , \lambda _ { 2 } I _ { n _ { 2 } } , \dots , \lambda _ { s } I _ { n _ { s } } \} \right.$。

记 $D=P^{-1}AP$,令 $G=P^{-1}BP$,由于 $AB=BA$,因此\[
D G = \left( P ^ { - 1 } A P \right) \left( P ^ { - 1 } B P \right) = P ^ { - 1 } A B P = P ^ { - 1 } B A P = \left( P ^ { - 1 } B P \right) \left( P ^ { - 1 } A P \right) = G D.
\]

由于 $D = \operatorname { diag } \left\{ \lambda _ { 1 } I _{ n _ { 1 } } , \lambda _ { 2 } I _ { n _ { 2 } } , \dots , 
\lambda , I _ { n _ { n } } \right\}$,且$\lambda _ { 1 } , 
\lambda _ { 2 } , \dots , \lambda _ { s }$,两两不等,因此
$G = \operatorname { diag } \left\{ B _ { 1 } , B _ { 2 } , 
\dots , B _ { s } \right\}$,其中 $B_i$ 是 $n_i$ 级矩阵,
$i=1, 2, \dots, s$。由于 $B$ 可对角化,因此 $G=P^{-1}BP$ 也可对角化。
从而 $G$ 的最小多项式 $m_G(x)$在$\mathbb{F}[x]$中可以分解
成不同的一次因式的乘积。设$B_i$的最小多项式为$m_i(\lambda), i=1, 2, \dots, s$,则\[
m _ { G } ( \lambda ) = \left[ m _ { 1 } ( \lambda ) , m _ { 2 } ( \lambda ) , \dots , m _ { s } ( \lambda ) \right].
\]
于是$m _ { i } ( \lambda ) \left| m _ { G } ( \lambda )
\right.$,从而$m_i(\lambda)$在$\mathbb{F}[\lambda]$中也
可分解成不同的一次因式的乘积,因此$B_i$可对角化,于是存在域$
\mathbb{F}$上 $n$ 级可逆矩阵$Q$,使得$Q_i^{-1}B_iQ_i$为
对角矩阵,$i=1, 2, \dots, s$。
令\[
 { Q } = \operatorname { diag } \left\{  { Q } _ { 1 } ,  { Q } _ { 2 } , \dots ,  { Q } \right\},
\]
则$Q ^ { - 1 } G Q = \operatorname { diag } \left\{ Q _ { 1 } ^ { - 1 } B _ { 1 } Q _ { 1 } , Q _ { 2 } ^ { - 1 } B _ { 2 } Q _ { 2 } , \dots , Q ^ { - 1 } B , Q _ { 3 } \right\}$为对角矩阵。令$S=PQ$,则\begin{align*}
&S ^ { - 1 } B S = Q ^ { - 1 } P ^ { - 1 } B P Q = Q ^ { - 1 } G Q\\
&S ^ { - 1 } A S = Q ^ { - 1 } P ^ { - 1 } A P Q = Q ^ { - 1 } \operatorname { diag } \left\{ \lambda _ { 1 } I _ { n _ { 1 } } , \lambda _ { 2 } I _ { n _ { 2 } } , \dots , \lambda _ { i } I _ { n _ { 1 } } \right\} Q\\
&= \operatorname { diag } \left\{ Q _ { 1 } ^ { - 1 } \left( \lambda _ { 1 } I _ { n _ { 1 } } \right) Q _ { 1 } , Q _ { 2 } ^ { - 1 } \left( \lambda _ { 2 } I _ { n _ { 2 } } \right) Q _ { 2 } , \dots , Q _ { s } ^ { - 1 } \left( \lambda _ { s } I _ { n _ { j } } \right) Q _ { s } \right\}\\
&= \operatorname { diag } \left\{ \lambda _ { 1 } I _ { n _ { 1 } } , \lambda _ { 2 } I _ { n _ { 2 } } , \dots , \lambda , I _ { n } \right\}.
\end{align*}
于是$S^{-1}BS$和$S^{-1}AS$都是对角矩阵。
\end{subproof}

\begin{lemma}\label{midengchongyao}
设$V$是域$\mathbb{F}$上$n$维线性空间,$\mathscr{A}$是$V$上的一个线性变换。证明:$\mathscr{A}$是幂等变换的充分必要条件是\[
\operatorname { rank } ( \mathscr { A } ) + \operatorname { rank } ( \mathscr { A } - \mathscr { I } ) = n.
\]
\end{lemma}
\begin{subproof}
$V$上的线性变换$\mathscr{A}$是幂等变换$\Longleftrightarrow \mathscr{A}(\mathscr{A}-\mathscr{I})=0$.\\
考虑域$\mathbb{F}$上的一元多项式$f(x)=x(x-1)$,由于$(x, x-1)=1$,因此,Ker $f ( \mathscr { A } ) = \operatorname { Ker } \mathscr { A } \oplus \operatorname { Ker } ( \mathscr { A } - \mathscr { I } )$,从而\begin{align*}
\text{$\mathscr{A}$是幂等变换}&\Longleftrightarrow
\mathscr{A}(\mathscr{A}-\mathscr{I})=0\\
&\Longleftrightarrow f(\mathscr{A})=0\\
&\Longleftrightarrow V = \mathrm { Ker } \mathscr { A } \oplus \mathrm { Ker } ( \mathscr { A } - \mathscr { I } )\\
&\Longleftrightarrow \operatorname { dim } V = \operatorname { dim } ( \operatorname { Ker } \mathscr { A } ) + \operatorname { dim } ( \operatorname { Ker } ( \mathscr { A } - \mathscr { I } ) )\\
&\Longleftrightarrow \operatorname { rank } ( \mathscr { A } ) + \operatorname { rank } ( \mathscr { A } - \mathscr { I } ) = n.
\end{align*}
其中倒数第二个$\Longleftrightarrow$的“$\Longleftarrow$”理由是: ${\rm Ker}\mathscr { A } + \operatorname { Ker } ( \mathscr { A } - \mathscr { I } )$是直和,因此$\dim({\rm Ker} \mathscr{A}\oplus \operatorname { Ker } ( \mathscr { A } - \mathscr { I } ) = \operatorname { dim } ( \mathrm { Ker } \mathscr { A } ) + \operatorname { dim } ( \operatorname { Ker } ( \mathscr { A } - \mathscr { I } ) )$。于是$\operatorname { dim } V = \operatorname { dim } ( \operatorname { Ker } \mathscr { A } \oplus \operatorname { Ker } ( \mathscr { A } - \mathscr { I } ) )$。
从而$V = \operatorname { Ker } \mathscr { A } \oplus \operatorname { Ker } ( \mathscr { A } - \mathscr { I } )$。
\end{subproof}

回到原题。因为 $A$ 和 $B$ 都是幂等矩阵,所以 $A$ 和 $B$ 都可对角化,又因为 $AB = BA$,所以 $A$ 和 $B$ 可以同时对角化。故存在可逆矩阵 $P$,使得\[
P ^ { - 1 } A P = \left( \begin{matrix}
	E_s&		\\
	&		0\\
\end{matrix} \right)  , 
P ^ { - 1 } B P = \left( \begin{matrix}
	0&		\\
	&		E_t\\
\end{matrix} \right) .
\]
其中 $s = {\rm rank}(A)$,$t = {\rm rank}(B)$。 由引理\ref{midengchongyao} 知\[
{\rm rank}(A) + {\rm rank}(B) = {\rm rank}(A) + {\rm rank}(E_n - A) = n.
\]
所以 $s + t = n$。证毕。
\end{proof}

\item[七、](15分)
设 $A, B$ 为 $n$ 阶方阵,$A$ 为 $n$ 阶幂零阵,求证\[
| A + B | = | B |.
\]
\begin{proof}
因为$A$是$n$阶幂零矩阵,所以\[
A^{n-1}\ne 0, A^n=0.
\]
于是\[
A^{n-1}(A+B)=A^n+A^{n-1}B=A^{n-1}B.
\]
因此\[
|A^{n-1}||A+B|=|A^{n-1}||B|.
\]
故\[
| A + B | = | B |.
\]
\end{proof}

\item[八、](15 分)
设 $A \in \mathbb { C } ^ { n \times n }$ 可逆. 求证矩阵方程 $A X A ^ { T } - X = 0$ 仅有零解的充要条件为 $A$ 的任何两个特征值的乘积不为 1。
\begin{proof}
\begin{lemma}
设$A$,$B$是$n$级复矩阵。则矩阵方程$AX-XB=0$只有零解的充分必要条件是:$A$与$B$没有公共的特征值。
\end{lemma}
\begin{subproof}
必要性。假设$A$与$B$有公共的特征值$\lambda$,则存在$\boldsymbol { \boldsymbol{\alpha} } \in \mathbb { C } ^ { n }$且$\boldsymbol{\alpha} \neq 0$使得$A \boldsymbol{\alpha} =\lambda_0 \boldsymbol {\boldsymbol{\alpha}}$
由于
\[
\left| \lambda _ { 0 } I - B ^ { \prime } \right| = \left| \left( \lambda _ { 0 } I - B \right) ^ { \prime } \right| = \left| \lambda _ { 0 } I - B \right| = 0,
\]
因此$\lambda_0$也是$B’$的一个特征值,从而存在$\boldsymbol { \beta } \in \mathbb { C } ^ { n }$且$\boldsymbol { \beta }\ne 0$,使得$B ^ { \prime } \boldsymbol { \beta } = \lambda _ { 0 } \boldsymbol { \beta }$。于是$\boldsymbol { \beta } ^ { \prime } B = \lambda _ { 0 } \boldsymbol { \beta } ^ { \prime }$。
从而
\[
A \left( \boldsymbol{\alpha}\boldsymbol{ \beta} ^ { \prime } \right) - \left( \boldsymbol{\alpha}\boldsymbol{ \beta} ^ { \prime } \right) B = \lambda _ { 0 }
\boldsymbol{\alpha} \boldsymbol{\beta} ^ { \prime } - 
\boldsymbol{\alpha} \lambda _ { 0 } \boldsymbol{\beta} ^ 
{ \prime } = 0.
\]
因此$ \boldsymbol{\alpha}  \boldsymbol{ \beta } ^ { \prime }$是矩阵方程$AX-XB=0$的一个解。\\
由于$\boldsymbol{\alpha} \neq 0 , \boldsymbol{\beta} \neq 
0$,因此$\boldsymbol{\alpha}$的某个分量$a_i\ne0$,$B$的
某个分量$b_i\ne 0$。从而$\boldsymbol{\alpha\beta}'$的$(i, j)
$元$a_ib_j\ne 0$,因此$\boldsymbol{\alpha\beta}'\ne0$,于
是$\boldsymbol{\alpha\beta}'$是$AX-XB=0$的一个非零解。\\
充分性。设$A$与$B$没有公共的特征值,则$A$的特征多项式
$f(\lambda)$与$B$的特征多项式$g(\lambda)$没有公共的复根,
从而$f(\lambda)$与$g(\lambda)$在$\mathbb{C}[\lambda]
$中互素。于是存在$u ( \lambda ) , v ( \lambda ) \in \mathbb{ 
C } [ \lambda ]$,使得$u ( \lambda ) f ( \lambda ) + v ( 
\lambda ) g ( \lambda ) = 1$。$\lambda$用$A$代入,得
\[
u ( A ) f ( A ) + v ( A ) g ( A ) = I.
\]
由于$f(A)=0$,因此$g(A)$可逆。设$n\times m$级复矩阵$
\mathbb{C}$是矩阵方程$AX-XB=0$的一个解,则$AC=CB$。设
\[
g ( \lambda ) = \lambda ^ { m } + b _ { m - 1 } \lambda ^ 
{ m - 1 } + \cdots + b _ { 1 } \lambda + b _ { 0 },
\]
则
\begin{align*}
g ( A ) C & = \left( A ^ { m } + b _ { m - 1 } A ^ { m - 1 } 
+ \cdots + b _ { 1 } A + b _ { 0 } I \right) C \\ 
& = C \left( B ^ { m } + b _ { m - 1 } B ^ { m - 1 } + 
\cdots + b _ { 1 } B + b _ { 0 } I \right) \\
&=C \cdot g ( B ) = C \cdot 0 = 0.
\end{align*}
两边左乘$g(A)^{-1}$,即得$C=0$。因此$AX-XB=0$只有零解。
\end{subproof}
由引理知,命题显然成立。
\end{proof}
\end{enumerate}
\endinput